\documentclass[a4paper]{article}
\usepackage{header}
\usepackage{float}
\usepackage{cmap}

\newcommand\enumtocitem[3]{\item\textbf{#1}\addtocounter{#2}{1}\addcontentsline{toc}{#2}{\protect{\numberline{#3}} #1}}
\newcommand\defitem[1]{\enumtocitem{#1}{subsection}{\thesubsection}}
\newcommand\proofitem[1]{\enumtocitem{#1}{subsection}{\thesubsection}}

\newtheorem{theorem1*}{Theorem}

\newtheoremstyle{named}{}{}{}{}{\bfseries}{}{.5em}{Теорема \thmnote{#3}}
\theoremstyle{named}
\newtheorem*{namedtheorem}{Theorem}

\newcommand{\italicbold}[1]{\emph{\textbf{#1}}}

\newlist{colloq}{enumerate}{1}
\setlist[colloq]{label=\textbf{\arabic*.}}

\everymath{\displaystyle}

\renewcommand*{\arraystretch}{1.5}

\title{\HugeМатематический анализ, Коллоквиум 2}
\author{
	Балюк Игорь \\
	\href{https://teleg.run/lodthe}{@lodthe},
    \href{https://github.com/LoDThe/hse-tex}{GitHub} \\
}

\usepackage[yyyymmdd,hhmmss]{datetime}
\settimeformat{xxivtime}
\renewcommand{\dateseparator}{.}
\date{Дата изменения: \today \ в \currenttime}

\begin{document}
    \maketitle

    \tableofcontents

    \newpage

    \section{Вопросы предварительной части коллоквиума}

	Список вопросов предварительной части коллоквиума, ответ на которые	необходим для подготовки к основной части.

    \begin{colloq}
    
    \defitem{Определение непрерывности функции в точке.}

    	Функция $f(x)$ непрерывна в точке $x_0$, если она определена на некоторой окрестности этой точки и $\lim_{x \to x_0} f(x) = f(x_0)$. Другими словами, $A = f(x_0)$ и справедливы следующие определения предела функции в точке $x_0$:

    	\begin{itemize}
    		\item
    		\italicbold{По Коши}:
    		\begin{equation*}
    			\forall \eps > 0 \ \exists \delta > 0: \ \forall x, 0 < |x - x_0| < \delta \implies |f(x) - A| < \eps
    		\end{equation*}


    		\item
    		\italicbold{По Гейне}:
    		\begin{equation*}
    			\forall \{x_n\}: \; x_n \in \overset{\circ}{U}(x_0), \lim_{n \to \infty} x_n = x_0 \implies 
    			\lim_{n \to \infty} f(x_n) = A
    		\end{equation*}
    	\end{itemize}

	\defitem{Точки разрыва, их классификация.}

		Пусть $f(x)$ определена в некоторой окрестности $U_{\delta}(a)$ и функция разрывна в $a$. Тогда этот разрыв является одним из следующих:

		\begin{itemize}
			\item
			\italicbold{Устранимый разрыв}: пределы $f(x)$ справа и слева существуют и равны друг другу, но отличаются от значения функции в исследуемой точке:
			\begin{equation*}
				\lim_{x \to a - 0} f(x) = \lim_{x \to a + 0} f(x) \neq f(a)
			\end{equation*}

			\item
			\italicbold{Неустранимый разрыв первого рода}: пределы $f(x)$ справа и слева существуют, но не равны друг другу

			\item
			\italicbold{Неустранимый разрыв второго рода}: хотя бы один из односторонних пределов $f(x)$ не существует или равен бесконечности.
		\end{itemize}

	\defitem{Теорема о непрерывности сложной функции.}

		\begin{theorem*}
			Пусть функция $g(x)$ непрерывна в точке $a_0$ и функция $f(x)$ непрерывна в точке $b_0~=~g(a_0)$. Тогда функция $f(g(x))$ непрерывна в точке $a_0$.
		\end{theorem*}

	\defitem{Формулировки первой и второй теорем Вейерштрасса.}

		\begin{namedtheorem}[Вейерштрасса (первая)]
			Если функция $f(x)$ непрерывна на отрезке $[a, b]$, то она ограничена на этом отрезке.
		\end{namedtheorem}

		\begin{namedtheorem}[Вейерштрасса (вторая)]
			Непрерывная на отрезке $[a, b]$ функция $f$ достигает на нем своих точных нижней и верхней граней. То есть существуют такие точки $x_1, x_2 \in [a, b]$, так что для любого $x \in [a, b]$, выполняются неравенства:
			\begin{equation*}
				f(x_1) \leq f(x) \leq f(x_2)
			\end{equation*}
		\end{namedtheorem}

	\defitem{Понятие производной функции в точке.}

		Рассмотрим функцию, область определения которой содержит точку $x_0$. Тогда функция $f(x)$ является дифференцируемой в точке $x_0$, и ее производная определяется $f'(x_0)$, если следующий предел существует:
		\begin{equation*}
			f'(x_0) = \lim_{\Delta \to 0} \dfrac{f(x_0 + \Delta) - f(x_0)}{\Delta}
		\end{equation*}

	\defitem{Геометрический и физический смысл производной.}

		\textbf{Геометрический смысл производной}. Производная в точке $x_0$ равна угловому коэффициенту касательной к графику функции $y = f(x)$ в этой точке.

		\textbf{Физический смысл производной}. Если точка движется вдоль оси $OX$ и ее координата изменяется по закону  $x(t)$, то мгновенная скорость точки: $v(t) = x'(t)$.

	\defitem{Уравнение касательной к графику функции в точке.}

		Пусть дана функция $f$, которая в некоторой точке $x_0$ имеет конечную производную $f(x_0)$. Тогда прямая, проходящая через точку $(x_0; f(x_0))$, имеющая угловой коэффициент $f’(x_0)$, называется касательной.

		Итак, пусть дана функция $y = f(x)$, которая имеет производную $y = f’(x)$ на отрезке $[a, b]$. Тогда в любой точке $x_0 \in (a; b)$ к графику этой функции можно провести касательную, которая задается уравнением:
		\begin{equation*}
			y = f'(x_0) \cdot (x - x_0) + f(x_0)
		\end{equation*}

		\href{https://ege-ok.ru/2014/02/16/uravnenie-kasatelnoy}{Подробнее тут}

	\defitem{Понятие дифференцируемости функции в точке.}

		Функция $f(x)$ является дифференцируемой в точке $x_0$ своей области определения $D[f]$, если существует такая константа A, что:
		\[\begin{gathered}
			f(x) = f(x_0) + A(x - x_0) + \os(x - x_0) \\
			\text{и} \\
			A = f'(x_0) = \lim_{\Delta x \to 0} \dfrac{\Delta y}{\Delta x} = \lim_{x \to x_0} \dfrac{f(x) - f(x_0)}{x - x_0}
		\end{gathered}\]

	\defitem{Правила дифференцирования (производная суммы, произведения, частного).}

		Пусть функции $f(x)$ и $g(x)$ имеют производные в точке $x_0$. Тогда,
		\[\begin{gathered}
			(g + f)'(x_0) = g'(x_0) + f'(x_0) \\
			(g \cdot f)'(x_0) = g'(x_0) \cdot f(x_0) + g(x_0) \cdot f'(x_0) \\
		\end{gathered}\]

		Если $g(x_0) \neq 0$, то
		\begin{equation*}
			\left(\dfrac{f}{g}\right)'(x_0) = \dfrac{f'(x_0) \cdot g(x_0) - f(x_0) \cdot g'(x_0)}{g(x_0)^2}
		\end{equation*}

	\defitem{Формула вычисления производной сложной функции.}

		Если $g(x)$ дифференцируема в точке $x_0$ и $f(x)$ дифференцируема в точке $y_0 = g(x_0)$, тогда,
		\begin{equation*}
			(f \circ g)'(x_0) = (f(g(x_0)))' = f'(g(x_0)) \cdot g'(x_0)
		\end{equation*}

	\defitem{Таблица производных основных элементарных функций.}

		{
			\everymath{\textstyle}
			\begin{figure}[H]
				\centering
					\begin{tabular}{| c | c |} \hline
					$f(x) $&$ f'(x) $\\\hline
					$const $&$ 0 $\\\hline
					$x^a $&$ a \cdot x^{a - 1} $\\\hline
					$a^x $&$ a^x \cdot \ln a $\\\hline
					$e^x $&$ e^x $\\\hline
					$\log_a x $&$ \frac{1}{\ln a \cdot x} $\\\hline
					$\ln x $&$ \frac{1}{x} $\\\hline
					$\sin x $&$ \cos x $\\\hline
					$\cos x $&$ -\sin x $\\\hline
					$\tg x $&$ \frac{1}{\cos^2 x} $\\\hline
					$\ctg x $&$ -\frac{1}{\sin^2 x} $\\\hline
				\end{tabular}\quad\quad\quad
				\begin{tabular}{| c | c |} \hline
					$f(x) $&$ f'(x) $\\\hline
					$\arcsin x $&$ \frac{1}{\sqrt{1 - x^2}} $\\\hline
					$\arccos x $&$ -\frac{1}{\sqrt{1 - x^2}} $\\\hline
					$\arctg x $&$ \frac{1}{1 + x^2} $\\\hline
					$\arcctg x $&$ -\frac{1}{1 + x^2} $\\\hline
				\end{tabular}
			\end{figure}

		}


	\defitem{Понятие дифференциала (первого) функции в точке.}

		Функция $f(x)$ является дифференцируемой в точке $x_0$ своей области определения $D[f]$, если существует такая константа A, что:
		\[\begin{gathered}
			f(x) = f(x_0) + A(x - x_0) + \os(x - x_0) \\
			A = f'(x_0) = \lim_{\Delta \to 0} \dfrac{f(x_0 + \Delta) - f(x_0)}{\Delta} \\
		\end{gathered}\]

		Тогда выражение $f'(x_0) dx$ называют дифференциалом функции $f(x)$ в точке $x_0$. Обозначение: $df~=~df(x_0, dx)$. Обратите внимание, что $df$ зависит и от точки, и от $dx$.

	\stepcounter{subsection}
	\stepcounter{colloqi}

	\defitem{Определение локального экстремума. Необходимое условие для внутреннего локального экстремума (теорема Ферма).}

		Точка $x_0$ называется точкой локального максимума (минимума) функции $f$, если существует такая окрестность $U_{\delta} (x_0)$ точки $x_0$, что 
		\begin{equation*}
			\forall x \in U_{\delta}(x_0) \implies f(x) \leq f(x_0) \text{ (для минимума соответственно $f(x) \geq f(x_0)$)}
		\end{equation*}

		$x_0$ называется точкой строгого локального максимума (минимума), если
		\begin{equation*}
			\forall x \in \overset{\circ}{U_{\delta}} (x_0) \implies f(x) < f(x_0) \text{ (для минимума соответственно $f(x) > f(x_0)$)}
		\end{equation*}

		\begin{namedtheorem}[Ферма]
			Если функция имеет в точке локального экстремума производную, то эта производная равна нулю.
		\end{namedtheorem}

	\defitem{Формулы Лагранжа и Коши.}

		\begin{namedtheorem}[Лагранжа. Формула конечных приращений]
			Если функция $f(x)$ непрерывна на отрезке $[a, b]$ и дифференцируема на интервале $(a, b)$, то в этом интервале существует хотя бы одна точка $x_0$, что
			\begin{equation*}
				\dfrac{f(b) - f(a)}{b - a} = f'(x_0)
			\end{equation*}
		\end{namedtheorem}

		\begin{namedtheorem}[Коши. Обобщает формулу конечных приращений Лагранжа.]
			Пусть функции $f(x)$ и $g(x)$ непрерывны на отрезке $[a, b]$ и дифференцируемы на интервале $(a, b)$, причем $g'(x) \neq 0$ при всех $x \in (a, b)$. Тогда в этом интервале существует точка $x = \xi$ такая, что
			\begin{equation*}
				\dfrac{f(b) - f(a)}{g(b) - g(a)} = \dfrac{f'(\xi)}{g'(\xi)}
			\end{equation*}
		\end{namedtheorem}

	\defitem{Многочлен Тейлора и формула Тейлора для функций одной переменной.}

		Предположим, что имеется некоторая функция $f(x)$ и надо исследовать ее поведение в некоторой точке $x_0$ или ее окрестности. Сама функция может быть при этом достаточно сложной, и поэтому непосредственное вычисление $\lim_{x \to x_0} f(x)$ (как пример того, что мы хотим узнать о функции в $x_0$) окажется крайне трудоемким. Идея в том, чтобы найти такой многочлен $P_n(x)$, что $f(x) \sim P_n(x - x_0)$ при $x \to x_0$, а затем исследовать его. Работать с многочленами практически всегда намного проще.

		Предположим пока, что $x_0 = 0$. Тогда $P_n(x) = c_0 + c_1x + c_2x^2 + \dots + c_nx^n$. $P_n(0) = c_0$, а $P_n'(x) = c_1 + 2c_2x + \dots + nc_nx^{n - 1}$, из чего следует, что $c_1 = P_n'(0)$. По аналогии можно получить, что $c_2 = \dfrac{P_n''(0)}{2!}, \dots, c_n = \dfrac{P_n^{(n)}(0)}{n!}$. Т.е. получаем, что $P_n(x) = P_n(0) + \dfrac{P_n'(0)}{1!}x + \dots + \dfrac{P_n^{(n)}(0)}{n!} x^n$.

		Пусть $\exists f^{(n)}(x_0)$, тогда справедлива формула:
		\begin{equation*}
			f(x) = f(x_0) + \dfrac{f'(x_0)}{1!}(x - x_0) + \dfrac{f''(x_0)}{2!}(x - x_0)^2 + \dots + \dfrac{f^{(n)}(x_0)}{n!}(x - x_0)^n + r_n(f, x)
		\end{equation*}

		Эта формула называется \textbf{формулой Тейлора} и обычно записывается в виде:
		\begin{equation*}
			f(x) = \underbrace{\sum_{k = 0}^n \dfrac{f^{(k)}(x_0)}{k!} \cdot (x - x_0)^k}_{\text{многочлен Тейлора}} + \underbrace{r_n(f, x)}_{\text{остаточный член}}
		\end{equation*}


	\defitem{Формулы Маклорена для основных элементарных функций.}

		При $x_0 = 0$ формула Тейлора с остаточным членом в форме Пеано называется формулой Маклорена

		Приведем пример: $f(x) = \sin x$. Вспомним, что
		\begin{equation*}
			f^{(n)}(x) = \sin \left(x + \dfrac{\pi n}{2}\right) \implies f^{(n)}(0) = \sin \left(\dfrac{\pi n}{2}\right) 
			= \begin{cases}
				0, \quad &\text{если $n = 2k$}, \\
				(-1)^k, \quad &\text{если $n = 2k + 1$}
			\end{cases}
		\end{equation*}

		Тогда получаем следующее разложение:
		\begin{equation*}
			\sin x = x - \dfrac{x^3}{3!} + \dfrac{x^5}{5!} - \dots + (-1)^{n - 1} \cdot \dfrac{x^{2n - 1}}{(2n - 1)!} + \os(x^{2n})
		\end{equation*}

		\begin{enumerate}[leftmargin=*]
			\item
			$e^x = 1 + \dfrac{x}{1!} + \dfrac{x^2}{2!} + \dots + \dfrac{x^n}{n!} + \os(x^n), x \to 0$

			\item
			$\ln(1 + x) = x - \dfrac{x^2}{2} + \dfrac{x^3}{3} - \dots + (-1)^{n + 1} \cdot \dfrac{x^n}{n} + \os(x^n), x \to 0$

			\item
			$\underset{\alpha \in \RR}{(1 + x)^{\alpha}} = 1 + \sum_{k = 1}^{n} \dbinom{\alpha}{k} x^k + \os(x^n)$

			Например $(1 + x)^{\frac{1}{3}} - 1 = \dbinom{\frac{1}{3}}{1}x + \dbinom{\frac{1}{3}}{2} x^2 + \os(x^2) = \dfrac{1}{3} x + \dfrac{\frac{1}{3}(\frac{1}{3} - 1)}{2} x^2 + \os(x^2)$

			\item
			$\sin (x) = x - \dfrac{x^3}{3!} + \dfrac{x^5}{5!} - \dots + (-1)^{n - 1} \cdot \dfrac{x^{2n - 1}}{(2n - 1)!} + \os(x^{2n})$

			\item
			$\cos (x) = 1 - \dfrac{x^2}{2!} + \dfrac{x^4}{4!} + \dots + (-1)^{n} \dfrac{x^{2n}}{(2n)!} + \os(x^{2n + 1})$

			\item
			$\tg (x) = x + \dfrac{x^3}{3} + \dfrac{2}{15} x^5 + \dots + \dfrac{B_{2n}(-4)^n(1 - 4^n)}{(2n)!} \cdot x^{2n - 1} + \os(x^{2n - 1})$, где $B_{2n}$ --- числа Бернулли

			Но достаточно помнить, что
			$\tg (x) = x + \dfrac{x^3}{3} + \dfrac{2}{15}x^5 + \os(x^5)$, т.е. общая формула для семинаров \underline{не} нужна

			\item
			$\arcsin (x) = x + \dfrac{x^3}{6} + \dfrac{3}{40} x^5 + \dots + \dfrac{(2n)!}{4^n (n!)^2 (2n + 1)} \cdot x^{2n + 1} + \os(x^{2n + 1})$

			Достаточно знать $\arcsin (x) = x + \dfrac{x^3}{6} + \dfrac{3}{40} x^5 + \os(x^5)$

			\item
			$\arccos (x) = \dfrac{\pi}{2} - \arcsin (x)$

			\item
			$\arctg(x) = x - \dfrac{x^3}{3} + \dfrac{x^5}{5} + \dots + (-1)^{n + 1} \dfrac{x^{2n - 1}}{2n - 1} + \os(x^{2n - 1})$
		\end{enumerate}

	\defitem{Правило Лопиталя.}

		\begin{namedtheorem}[Лопиталя (первое правило)]
			Если функции $f(x)$ и $g(x)$ таковы, что

			\begin{enumerate}
				\item
				$f(x)$ и $g(x)$ дифференцируемы в проколотой окрестности точки $a$

				\item
				$\lim_{x \to a} f(x) = \lim_{x \to a} g(x) = 0$

				\item
				$g'(x) \neq 0$ в окрестности $U$

				\item
				Существует $\lim_{x \to a} \dfrac{f'(x)}{g'(x)}$
			\end{enumerate}

			Тогда существует $\lim_{x \to a} \dfrac{f(x)}{g(x)} = \lim_{x \to a} \dfrac{f'(x)}{g'(x)}$
		\end{namedtheorem}

		\begin{namedtheorem}[Лопиталя (второе правило)]
			Если для функций $f(x)$ и $g(x)$ справедливо следующее:

			\begin{enumerate}
				\item
				$f(x)$ и $g(x)$ дифференцируемы на интервале $(a, b)$

				\item
				$\lim_{x \to a + 0} f(x) = \lim_{x \to a + 0} g(x) = \infty$

				\item
				$g'(x) \neq 0$ при $x \in (a, b)$

				\item
				Существует $\lim_{x \to a + 0} \dfrac{f'(x)}{g'(x)} = A$
			\end{enumerate}

			Тогда $\lim_{x \to a + 0} \dfrac{f(x)}{g(x)} = \lim_{x \to a + 0} \dfrac{f'(x)}{g'(x)} = A$
		\end{namedtheorem}

    \end{colloq}

    \section{Вопросы на знание доказательств}
    \begin{colloq}
    %\setlength\parindent{20pt}

    \proofitem{Определение непрерывности функции в точке. Точки разрыва, их классификация.}

    	Функция $f(x)$ непрерывна в точке $x_0$, если она определена на некоторой окрестности этой точки $\lim_{x \to x_0} f(x) = f(x_0)$. Другими словами, $A = f(x_0)$ и справедливы следующие определения предела функции в точке $x_0$:

    	\begin{itemize}
    		\item
    		\italicbold{По Коши}:
    		\begin{equation*}
    			\forall \eps > 0 \ \exists \delta > 0: \ \forall x, 0 < |x - x_0| < \delta \implies |f(x) - A| < \eps
    		\end{equation*}


    		\item
    		\italicbold{По Гейне}:
    		\begin{equation*}
    			\forall \{x_n\}: \; x_n \in \overset{\circ}{U}(x_0), \lim_{n \to \infty} x_n = x_0 \implies 
    			\lim_{n \to \infty} f(x_n) = A
    		\end{equation*}
    	\end{itemize}

    	\textbf{Классицифкация разрывов:}

    	Пусть $f(x)$ определена в некоторой окрестности $U_{\delta}(a)$ и функция разрывна в $a$. Тогда говорят, что функция имеет

		\begin{itemize}
			\item
			\italicbold{Устранимый разрыв}: пределы $f(x)$ справа и слева существуют и равны друг другу, но отличаются от значения функции в исследуемой точке:
			\begin{equation*}
				\lim_{x \to a - 0} f(x) = \lim_{x \to a + 0} f(x) \neq f(a)
			\end{equation*}

			\item
			\italicbold{Неустранимый разрыв первого рода}: пределы $f(x)$ справа и слева существуют, но не равны друг другу

			\item
			\italicbold{Неустранимый разрыв второго рода}: хотя бы один из односторонних пределов $f(x)$ не существует или равен бесконечности.
		\end{itemize}

	\proofitem{Непрерывность элементарных функций.}

		Многие элементарные функции непрерывны на своей области определения (например, $\sin x, \cos x$). Докажем для некоторых из них

		\[\begin{gathered}
			\lim_{\Delta \to 0} \sin(x_0 + \Delta)
			= \lim_{\Delta \to 0} (\sin x_0 \cos \Delta + \cos x_0 \sin \Delta)
			= \sin x_0 \cdot 1 + \cos x_0 \cdot 0 = \sin x_0
		\end{gathered}\]

		\[\begin{gathered}
			\lim_{\Delta \to 0} \cos(x_0 + \Delta)
			= \lim_{\Delta \to 0} (\cos x_0 \cos \Delta - \sin x_0 \sin \Delta)
			= \cos x_0 \cdot 1 - \sin x_0 \cdot 0 = \cos x_0
		\end{gathered}\]

		Тангенс и котангенс выражаются через синус и косинус. Воспользуемся арифметическими свойствами непрерывных функций. Поскольку синус и косинус определены и непрерывны для всех $x$, то тангенс и котангенс определены и непрерывны для всех $x$, кроме точек, в которых знаменатель обращается в нуль:

		\[\begin{gathered}
			y = \tg x, \quad x \neq \dfrac{\pi}{2} + \pi n, n \in \ZZ \\
			y = \ctg x, \quad x \neq \pi n, n \in \ZZ
		\end{gathered}\]

		\[\begin{gathered}
			\lim_{\Delta \to 0} a^{x_0 + \Delta} = a^{x_0} \cdot a^0 = a^{x_0} \cdot 1 = a^{x_0}
		\end{gathered}\]
		
	\proofitem{Арифметические свойства непрерывных функций.}

		\begin{theorem*}
			Пусть $g(x)$ и $f(x)$ непрерывны в $a$, тогда функции $f \pm g, f \cdot g, \dfrac{f}{g} (g \neq 0)$ также непрерывны в точке $a$.
		\end{theorem*}

		\begin{proof}
			Рассмотрим сумму $(f(x) + g(x))$. Для остальных операций доказательство практически аналогично. По определению $\lim_{x \to a} f(x) = f(a)$ и $\lim_{x \to a} g(x) = g(a)$. Но тогда, используя свойство суммы для пределов, получаем, что $\lim_{x \to a} (f(x) + g(x)) = f(a) + g(a)$, что означает, что $(f(x) + g(x))$ непрерывна в точке $a$.
		\end{proof}

	\proofitem{Теорема о непрерывности сложной функции.}

		\begin{theorem*}
			Если функция $g(t)$ непрерывна в точке $t_0$ и функция $f(x)$ непрерывна в точке $x_0 = g(t_0)$, то $f(g(t))$ непрерывна в $t_0$. 
		\end{theorem*}

		\begin{proof}
			Для доказательства этой теоремы воспользуемся формальным преобразованием двух выражений с кванторами.

			$f(x)$ непрерывна в $x_0$:
			\begin{equation*}
				\forall \eps > 0 \ \exists \delta > 0: \ \forall x: \ 0 < |x - x_0| < \delta \implies |f(x) - f(x_0)| < \eps
			\end{equation*}

			$g(t)$ непрерывна в $t_0$:
			\begin{equation*}
				\forall \delta > 0 \ \exists \mu > 0: \ \forall t: \ 0 < |t - t_0| < \mu \implies |g(t) - g(t_0)| < \delta
			\end{equation*}

			Получается, $f(g(t))$ непрерывна в $t_0$:
			\begin{equation*}
				\forall \eps > 0 \ \exists \mu > 0: \ \forall t: \ 0 < |t - t_0| < \mu \implies |f(g(t)) - f(g(t_0))| < \eps
			\end{equation*}
		\end{proof}

	\proofitem{Свойства функций, непрерывных на отрезке (первая и вторая теоремы Вейерштрасса).}

		\begin{namedtheorem}[Вейерштрасса (первая)]
			Если функция $f(x)$ непрерывна на отрезке $[a, b]$, то она на нём ограничена, то есть $\exists A: \ \forall x \in [a, b] \implies |f(x)| \leq A$
		\end{namedtheorem}

		\begin{proof}
			Докажем от противного.

			Пусть $f$ не ограничена на отрезке $[a, b]$, тогда:

			\[\begin{gathered}
				\forall A > 0 \ \exists x_A \in [a, b]: \ |f(x_A)| > A \\
				A = 1 \implies \exists x_1 \in [a, b]: \ |f(x_1)| > 1 \\
				A = 2 \implies \exists x_2 \in [a, b]: \ |f(x_2)| > 2 \\
				\vdots \\
				A = n \implies \exists x_n \in [a, b]: \ |f(x_n)| > n \\
			\end{gathered}\]

			Получим последовательность $\{x_n\} \subset [a, b]$, то есть последовательность $\{x_n\}$ ограничена.

			По теореме Больцано-Вейерштрасса из неё можно выделить подпоследовательность, которая сходится к точке $c$, то есть
			\begin{equation*}
				\lim_{k \to \infty} x_{n_k} = c
			\end{equation*}

			Тогда $c \in [a, b]$. Но по условию функция непрерывна в точке $c$ и тогда по определению непрерывности в точке по Гейне $\lim_{k \to \infty} f(x_{n_k}) = f(c)$.

			С другой стороны
			\begin{equation*}
				|f(x_{n_k})| > n_k, n_k \geq k \implies \lim_{k \to \infty} f(x_{n_k}) = \infty
			\end{equation*}

			А это противоречит единственности предела.
		\end{proof}

		\begin{namedtheorem}[Вейерштрасса (вторая)]
			Непрерывная на отрезке $[a, b]$ функция $f$ достигает на нем своих точных нижней и верхней граней. То есть существуют такие точки $c_1, c_2 \in [a, b]$, так что для любого $x \in [a, b]$, выполняются неравенства:
			\begin{equation*}
				f(c_2) \leq f(x) \leq f(c_1)
			\end{equation*}
		\end{namedtheorem}

		\begin{proof}
			Докажем $\exists c_1 \in [a, b]: \ f(c_1) = \sup\limits_{x \in [a, b]} f(x)$.

			Пусть $M = \sup\limits_{x \in [a, b]} f(x)$ (существование следует из первой теоремы Вейерштрасса). В силу определения точной верхней грани выполняется условие:
			\begin{equation*}
				\begin{cases}
					\forall x \in [a, b] \implies f(x) \leq M \\
					\forall \eps > 0 \ \exists x_{\eps} \in [a, b]: \ M - \eps < f(x_{\eps}) \\
				\end{cases}
			\end{equation*}

			Полагая $\eps = 1, \dfrac{1}{2}, \dfrac{1}{3}, \dots, \dfrac{1}{n}$ получим последовательность $\{x_n\}$ такую, что для всех $n \in \NN$ выполняются условия $M - \dfrac{1}{n} < f(x_n) \leq M$, откуда $\exists \lim_{n \to \infty} f(x_n)$. Существует подпоследовательность $\{x_{n_k}\}$ последовательности $\{x_n\}$ (она ограничена отрезком $[a, b]$, а значит является ограниченной) и точка $c$ (по теореме Больцано-Вейерштрасса, из последовательности можно выделить подпоследовательность, сходящуюся к точке $c$), такие что $\lim_{k \to \infty} x_{n_k} = c$, где $c \in [a, b]$.

			В силу непрерывности функции $f$ в точке $c$, получаем $\lim_{k \to \infty} f(x_{n_k}) = f(c)$.

			С другой стороны, $\{f(x_{n_k})\}$ --- подпоследовательность последовательности $\{f(x_n)\}$, сходящейся к числу $M$. Поэтому $\lim_{k \to \infty} f(x_{n_k}) = M$.

			В силу единственности предела последовательности заключаем, что $f(c) = M = \sup\limits_{x \in [a, b]} f(x)$.

			Утверждение $\exists c_1 \in [a, b]: \ f(c_1) = \sup\limits_{x \in [a, b]} f(x)$ доказано.

			Аналогично доказывается $\exists c_2 \in [a, b]: \ f(c_2) = \inf\limits_{x \in [a, b]} f(x)$

			Функция непрерывна на интервале может не достигать своих точных граней (требовать непрерывности на сегменте существенно).
		\end{proof}

	\proofitem{Теорема Коши о прохождении непрерывной функции через промежуточные значения.}

		\begin{namedtheorem}[Больцано-Коши (первая), о нулях непрерывной функции]
			Если функция $f(x)$ непрерывна на сегменте $[a, b]$ и на своих концах принимает значение разных знаков, то существует такая точка, принадлежащая этому отрезку, в которой функция обращается в нуль.
		\end{namedtheorem}

		\begin{proof}
			Геометрически очень легко: функция пересечет ось $OX$.

			Алгебраически: разделим отрезок $[a, b]$ точкой $x_0$ на два равных по длине отрезка, тогда либо $f(x_0) = 0$ и, значит, искомая точка $x_0$ найдена, либо $f(x_0) \neq 0$ и тогда на концах одного из полученных промежутков функция $f$ принимает значения разных знаков, точнее, на левом конце значение меньше нуля, на правом --- больше.

			Обозначим этот отрезок $[a_1, b_1]$ и разделим его снова на два равных подлине отрезка и т.д. В результате, либо через конечное число шагов придем к искомой точке $x$, в которой $f(x) = 0$, либо получим последовательность вложенных отрезков $[a_n, b_n]$ по длине стремящихся к нулю и таких, что
			\begin{equation*}
				f(a_n) < 0 < f(b_n)
			\end{equation*}

			Пусть $\gamma$ --- общая точка всех отрезков $[a_n, b_n]$. Тогда $\gamma = \lim_{n \to \infty} a_n = \lim_{n \to \infty} b_n$. Поэтому, в силу непрерывности функции $f$
			\begin{equation*}
				f(\gamma) = \lim_{n \to \infty} f(a_n) = \lim_{n \to \infty} f(b_n)
			\end{equation*}

			Но тогда
			\begin{equation*}
				\lim_{n \to \infty} f(a_n) \leq 0 \leq \lim_{n \to \infty} f(b_n)
			\end{equation*}

			Откуда следует, что $f(\gamma) = 0$.
		\end{proof}

		\begin{namedtheorem}[Больцано-Коши (вторая), о промежуточном значении непрерывных функций]
			Если функция $f$ непрерывна на отрезке $[a, b]$ и $A = f(a) \neq f(b) = B$, число $C \in (A, B)$, тогда 
			существует такая точка $c \in [a, b]$, что $f(c) = C$.

			Другими словами, утверждается, что если непрерывная функция, принимает два значения, то она принимает и любое значение между ними.
		\end{namedtheorem}

		\begin{proof}
			Не нарушая общности будем считать, что $A = f(a) < f(b) = B$. Рассмотри функцию $h(x) = f(x) - C$, непрерывность на отрезке $[a, b]$ которой следует из непрерывности функции $f$. Очевидно что $h(a) = A - C < 0$ и $h(b) = B - C > 0$. Применяем к $h$ первую теорему Больцано-Коши и находим точку $c$, в которой $h(c) = f(c) - C = 0$, то есть $f(c) = C$. Теорема доказана.
		\end{proof}

	\proofitem{Понятие производной функции в точке.}

		Рассмотрим функцию, область определения которой содержит точку $x_0$. Тогда функция $f(x)$ является дифференцируемой в точке $x_0$, и ее производная  $f'(x_0)$ определяется следующей формулой, если существует предел
		\begin{equation*}
			f'(x_0) = \lim_{\Delta \to 0} \dfrac{f(x_0 + \Delta) - f(x_0)}{\Delta}
		\end{equation*}

	\proofitem{Геометрический и физический смысл производной.}

		\textbf{Геометрический смысл производной}. Производная в точке $x_0$ равна угловому коэффициенту касательной к графику функции $y = f(x)$ в этой точке.

		\textbf{Физический смысл производной}. Если точка движется вдоль оси $OX$ и ее координата изменяется по закону  $x(t)$, то мгновенная скорость точки: $v(t) = x'(t)$.

	\proofitem{Уравнение касательной к графику функции в точке.}

		Пусть дана функция $f$, которая в некоторой точке $x_0$ имеет конечную производную $f'(x_0)$. Тогда прямая, проходящая через точку $(x_0; f(x_0))$, имеющая угловой коэффициент $f’(x_0)$, называется касательной.

		Итак, пусть дана функция $y = f(x)$, которая имеет производную $y = f’(x)$ на отрезке $[a, b]$. Тогда в любой точке $x_0 \in (a; b)$ к графику этой функции можно провести касательную, которая задается уравнением:
		\begin{equation*}
			y = f'(x_0) \cdot (x - x_0) + f(x_0)
		\end{equation*}

	\stepcounter{subsection}
	\stepcounter{colloqi}

	\proofitem{Понятие дифференцируемости функции в точке.}

		Функция $f(x)$ является дифференцируемой в точке $x_0$ своей области определения $D[f]$, если существует такая константа A, что:
		\[\begin{gathered}
			f(x) = f(x_0) + A(x - x_0) + \os(x - x_0) \\
			\text{и} \\
			A = f'(x_0) = \lim_{\Delta x \to 0} \dfrac{\Delta y}{\Delta x} = \lim_{x \to x_0} \dfrac{f(x) - f(x_0)}{x - x_0}
		\end{gathered}\]

		\begin{namedtheorem}
			$f(x)$ дифференцируема в точке $x$ только и только тогда, когда $\exists f'(x)$, причем $A = f'(x)$
		\end{namedtheorem}

		\begin{proof}
			Докажем необходимость и достаточность

			\begin{itemize}
				\item
				\textbf{Необходимость}. Пусть $f(x)$ дифференцируема в точке $x$ $\implies \Delta y = A\Delta x + \os(\Delta x) \implies \dfrac{\Delta y}{\Delta x} = A + \dfrac{\os(\Delta x)}{\Delta x} = A + \os(1)$

				Тогда $\lim_{\Delta x \to 0} \dfrac{\Delta y}{\Delta x} = A \implies A = f'(x)$

				\item
				\textbf{Достаточность}. Пусть $\exists f'(x) \implies \exists \lim_{\Delta x \to 0} \dfrac{\Delta y}{\Delta x} = f'(x)$. 
				Рассмотрим $\beta(\Delta x) = \dfrac{\Delta y}{\Delta x} - f'(x)$. $\lim_{\Delta x \to 0} \beta(\Delta x) = 0$, 
				т.е. $\beta(\Delta x) = \os(1) \implies \dfrac{\Delta y}{\Delta x} - f'(x) = \os(1) \implies \Delta y = f'(x) \Delta x + \os(\Delta x)$
			\end{itemize}
		\end{proof}

	\proofitem{Необходимое условие дифференцируемости.}

		\begin{theorem*}
			Если функция $f(x)$ дифференцируема в точке $x_0$, то она непрерывна в этой точке.
		\end{theorem*}

		\begin{proof}
			Пусть функция $y = f(x)$ дифференцируема в точке $x_0$. Тогда её приращение представимо в виде
			\begin{equation*}
				\Delta y = A \Delta x + \os(\Delta x)
			\end{equation*}

			Но тогда при $\Delta x \to 0$ будет $\Delta y \to 0$, а это означает непрерывность функции $y = f(x)$ в точке $x_0$.
		\end{proof}

		Обратите внимание, что из непрерывности не следует дифференцируемости (например, $f(x) = |x|$).

	\proofitem{Правила дифференцирования.}

		\begin{theorem*}
			Если $f(x)$ и $g(x)$ дифференцируемы в точке $x$, то $f \pm g$, $f \cdot g$, $\dfrac{f}{g} (g \neq 0)$ также дифференцируемы в точке $x$, причем $(f \pm g)' = f' \pm g'$, $(f \cdot g)' = f'g + fg'$, $\left(\dfrac{f}{g}\right)' = \dfrac{f'g - fg'}{g^2}$
		\end{theorem*}

		\begin{proof}
			Будем считать, что $\Delta f$ отвечает приращению $f(x)$, $\Delta g$ отвечает приращению $g(x)$, а $\Delta h$ отвечает приращению $h(x)$.

			\begin{enumerate}
				\item
				$h(x) = f(x) \pm g(x)$
				\begin{align*}
					\Delta h 
					= h(x + \Delta x) - h(x)
					&= (f(x + \Delta x) \pm g(x + \Delta x)) - (f(x) \pm g(x)) \\
					&= (f(x + \Delta x) - f(x)) \pm (g(x + \Delta x) - g(x)) \\
					&= \Delta f \pm \Delta g
				\end{align*}

				Таким образом,
				\begin{equation*}
					\dfrac{\Delta h}{\Delta x} = \dfrac{\Delta f}{\Delta x} \pm \dfrac{\Delta g}{\Delta x}
				\end{equation*}

				При $\Delta x \to 0$ существует предел правой части, равный $f'(x) \pm g'(x)$, а значит, существует и предел левой части
				\begin{equation*}
					h'(x) = f'(x) \pm g'(x)
				\end{equation*}


				\item
				$h(x) = f(x) \cdot g(x)$
				\begin{align*}
					\Delta h 
					= h(x + \Delta x) - h(x)
					&= f(x + \Delta x) \cdot g(x + \Delta x) - f(x)g(x) \\
					&=(f(x + \Delta x) \cdot g(x + \Delta x) - f(x + \Delta x) \cdot g(x)) + (f(x + \Delta x) \cdot g(x) - f(x) \cdot g(x))
				\end{align*}

				Далее можно записать
				\begin{equation*}
					\Delta h = f(x + \Delta x) \cdot (g(x + \Delta x) - g(x)) + g(x) \cdot (f(x + \Delta x) - f(x)) = f(x + \Delta x) \cdot \Delta g + g(x) \Delta f
				\end{equation*}

				Таким образом
				\begin{equation*}
					\dfrac{\Delta h}{\Delta x} = f(x + \Delta x) \dfrac{\Delta g}{\Delta x} + g(x) \cdot \dfrac{\Delta f}{\Delta x}
				\end{equation*}

				Возьмем теперь предел правой части при $\Delta x \to 0$. В силу непрерывности $f(x)$ в $x$ (т.к. она дифференцируема в этой точке) $\lim_{\Delta x \to 0} f(x + \Delta x) = f(x)$. Тогда получаем, что
				\begin{equation*}
					h'(x) = f(x) \cdot g'(x) + g(x) \cdot f'(x)
				\end{equation*}
			\end{enumerate}

			\begin{lemma}
				Если $f(x)$ непрерывна в точке $a$ и $f(a) > 0 (f(a) < 0)$, то $\exists \delta > 0: f(x) > 0 (f(x) < 0) \forall x \in U_{\delta}(a)$
			\end{lemma}

			\begin{proof}
				Так как $f(x) \in C(a)$, то $\forall \eps > 0 \ \exists \delta > 0: \ \forall x \in U_{\delta}(a) \implies |f(x) - f(a)| < \eps \iff f(a) - \eps < f(x) < f(a) + \eps$. Положим $\eps = \dfrac{|f(a)|}{2}$, тогда $f(a) - \eps > 0$ при $f(a) > 0$ и $f(a) + \eps < 0$ при $f(a) < 0$. Т.е. левая и правая части неравенства всегда одного знака, значит $\forall x \in U_{\delta}(a)$ выполнено требуемое.
			\end{proof}

			\begin{enumerate}[resume]
				\item
				$h(x) = \dfrac{f(x)}{g(x)}$. По лемме, $g(x) \neq 0$, то $g(x + \Delta x) \neq 0$ для малых $\Delta x$. Тогда
				\begin{align*}
					\Delta h
					= h(x + \Delta x) - h(x)
					&= \dfrac{f(x + \Delta x)}{g(x + \Delta x)} - \dfrac{f(x)}{g(x)} \\
					&= \dfrac{f(x + \Delta x)g(x) - g(x + \Delta x)f(x)}{g(x)g(x + \Delta x)} \\
					&= \dfrac{(f(x + \Delta x)g(x) - f(x)g(x)) - (g(x + \Delta x)f(x) - f(x)g(x))}{g(x)g(x + \Delta x)} \\
					&= \dfrac{g(x)(f(x + \Delta x) - f(x)) - f(x)(g(x + \Delta x) - g(x))}{g(x)g(x + \Delta x)} \\
					&= \dfrac{g(x)\Delta f - f(x) \Delta g}{g(x)g(x + \Delta x)}
				\end{align*}

				Таким образом,
				\begin{equation*}
					\dfrac{\Delta h}{\Delta x} = \dfrac{g(x) \dfrac{\Delta f}{\Delta x} - f(x) \dfrac{\Delta g}{\Delta x}}{g(x)g(x + \Delta x)}
				\end{equation*}

				Снова используя непрерывность и беря предел правой и левой частей, получаем, что
				\begin{equation*}
					h'(x) = \dfrac{g(x)f'(x) - f(x)g'(x)}{g^2(x)}
				\end{equation*}
			\end{enumerate}
		\end{proof}

	\proofitem{Теорема о дифференцируемости и производной сложной функции.}

		\begin{theorem*}
			Пусть функцию $y = y(x)$ от переменной $x$ можно представить как сложную функцию в следующем виде:
			\begin{equation*}
				y(x) = f(u(x))
			\end{equation*}

			где $f(u)$ и $u(x)$ есть некоторые функции. Функция $u = u(x)$ дифференцируема при некотором значении переменной $x$. Функция $f(u)$ дифференцируема при значении переменной $u = u(x)$. Тогда сложная (составная) функция $y = f(u(x))$ дифференцируема в точке $x$ и ее производная определяется по формуле:
			\begin{equation*}
				y'(x) = f'(u) \cdot u'(x)
			\end{equation*}
		\end{theorem*}

		\begin{proof}
			Введем следующие обозначения.

			\[\begin{gathered}
				\Delta u = u(x + \Delta x) - u(x) \\
				\Delta f = f(u + \Delta u) - f(u) = f(u(x + \Delta x)) - f(u(x)) \\
			\end{gathered}\]

			Здесь $\Delta u$ есть функция от переменных $x$ и $\Delta x$, $\Delta f$ есть функция от переменных $u$ и $\Delta u$. Но мы будем опускать аргументы этих функций, чтобы не загромождать выкладки.

			Поскольку функции $u$ и $f$ дифференцируемы в точках $x$ и $u = u(x)$, соответственно, то в этих точках существуют производные этих функций, которые являются следующими пределами:
			\[\begin{gathered}
				u'(x) = \lim_{\Delta x \to 0} \dfrac{\Delta u}{\Delta x} \\
				f'(u) = \lim_{\Delta u \to 0} \dfrac{\Delta f}{\Delta u} \\
			\end{gathered}\]

			Рассмотрим следующую функцию:
			\begin{equation*}
				\eps(\Delta u) = \dfrac{\Delta f}{\Delta u} - f'(u)
			\end{equation*}

			При фиксированном значении переменной $u$, $\eps$ является функцией от $\Delta u$. Очевидно, что
			\begin{equation*}
				\lim_{\Delta u \to 0} \eps(\Delta u) = 0
			\end{equation*}

			Тогда
			\begin{equation*}
				\Delta f = (f'(u) + \eps(\Delta u)) \cdot \Delta u
			\end{equation*}

			Поскольку функция $u(x)$ является дифференцируемой функцией в точке $x$, то она непрерывна в этой точке. Поэтому
			\begin{equation*}
				\lim_{\Delta x \to 0} \Delta u = 0
			\end{equation*}

			Тогда
			\begin{equation*}
				\lim_{\Delta x \to 0} \eps(\Delta u) = \lim_{\Delta u \to 0} \eps(\Delta u) = 0
			\end{equation*}

			Теперь находим производную.
			\begin{align*}
				y'(x) 
				&= \lim_{\Delta x \to 0} \dfrac{\Delta f}{\Delta x} \\
				&= \lim_{\Delta x \to 0} \left(f'(u) \dfrac{\Delta u}{\Delta x} + \eps(\Delta u) \dfrac{\Delta u}{\Delta x}\right) \\
				&= f'(u) \cdot \lim_{\Delta x \to 0} \dfrac{\Delta u}{\Delta x} + \lim_{\Delta x \to 0} \eps(\Delta u) \cdot \lim_{\Delta x \to 0} \dfrac{\Delta u}{\Delta x} \\
				&= f'(u) \cdot u'(x) + 0 \cdot u'(x) \\
				&= f'(u) \cdot u'(x)
			\end{align*}

			Формула доказана.
		\end{proof}

	\stepcounter{subsection}
	\stepcounter{colloqi}

	\proofitem{Таблица производных основных элементарных функций.}

		{
			\everymath{\textstyle}
			\begin{figure}[H]
				\centering
					\begin{tabular}{| c | c |} \hline
					$f(x) $&$ f'(x) $\\\hline
					$const $&$ 0 $\\\hline
					$x^a $&$ a \cdot x^{a - 1} $\\\hline
					$a^x $&$ a^x \cdot \ln a $\\\hline
					$e^x $&$ e^x $\\\hline
					$\log_a x $&$ \frac{1}{\ln a \cdot x} $\\\hline
					$\ln x $&$ \frac{1}{x} $\\\hline
					$\sin x $&$ \cos x $\\\hline
					$\cos x $&$ -\sin x $\\\hline
					$\tg x $&$ \frac{1}{\cos^2 x} $\\\hline
					$\ctg x $&$ -\frac{1}{\sin^2 x} $\\\hline
				\end{tabular}\quad\quad\quad
				\begin{tabular}{| c | c |} \hline
					$f(x) $&$ f'(x) $\\\hline
					$\arcsin x $&$ \frac{1}{\sqrt{1 - x^2}} $\\\hline
					$\arccos x $&$ -\frac{1}{\sqrt{1 - x^2}} $\\\hline
					$\arctg x $&$ \frac{1}{1 + x^2} $\\\hline
					$\arcctg x $&$ -\frac{1}{1 + x^2} $\\\hline
				\end{tabular}
			\end{figure}

		}

	\proofitem{Понятие дифференциала (первого) функции в точке.}

		Функция $f(x)$ является дифференцируемой в точке $x_0$ своей области определения $D[f]$, если существует такая константа A, что:
		\[\begin{gathered}
			f(x) = f(x_0) + A(x - x_0) + \os(x - x_0) \\
			A = f'(x_0) = \lim_{\Delta \to 0} \dfrac{f(x_0 + \Delta) - f(x_0)}{\Delta} \\
		\end{gathered}\]

		Тогда выражение $f'(x_0) dx$ называют дифференциалом функции $f(x)$ в точке $x_0$. Обозначение: $df~=~df(x_0, dx)$. Обратите внимание, что $df$ зависит и от точки, и от $dx$.

	
	\addtocounter{subsection}{2}
	\addtocounter{colloqi}{2}

	\proofitem{[На коллоквиуме будет только производная высших порядков] Производные и дифференциалы высших порядков функции одной переменной в точке.}

		Рассмотрим функцию, дифференцируемую на множестве $E$. Т.е. $\exists f'(x)$, Если $f'(x)$ тоже дифференцируема на $E$, то $\exists (f'(x))' = f''(x)$. 

		Производной $n$-ого порядка будем считать $f^{(n)}(x) = (f^{(n - 1)}(x))'$, причем $f^{(0)}(x) = f(x)$. Разумеется, для существования производной $n$-ого порядка должны существовать производные всех меньших порядков.

		Множество функций, имеющих все производные до порядка $n$ включительно на множестве $E$, обозначается $C^{(n)}(E)$. Рассмотрми несколько примеров

		\begin{itemize}
			\item
			$f(x) = \sin x$

			\[\begin{gathered}
				f'(x) = \cos x = \sin \left(\dfrac{\pi}{2} + x\right) \\
				f''(x) = - \sin x = \sin \left(\dfrac{2 \pi}{2} + x\right) \\
				f'''(x) = - \cos x = \sin \left(\dfrac{3 \pi}{2} + x\right) \\
				f^{(4)}(x) = \sin x = \sin x
			\end{gathered}\]

			Докажем по индукции, что $f^{(n)}(x) = \sin \left(\dfrac{\pi n}{2} + x\right)$. При $n = 1$ уже было показано ранее. Пусть это верно при некотором $n$, покажем для $n = n + 1$.

			\[\begin{gathered}
				f^{(n + 1)}(x) = \sin \left(\dfrac{\pi(n + 1)}{2} + x\right) \\
				f^{(n + 1)}(x) = (f^{(n)}(x))' = \left(\sin \left(\dfrac{\pi n}{2} + x\right)\right)' = 
				\cos \left(\dfrac{\pi n}{2} + x\right) = \sin \left(\dfrac{\pi(n + 1)}{2} + x\right)
			\end{gathered}\]

			\item
			$f(x) = e^x$. $f^{(n)}(x) = e^x$

			\item
			$f(x) = x^m$. Беря $n$ раз производную, получаем, что $f^{(n)}(x) = m(m - 1) \dots (m - n + 1)x^{m - n}$

			\item
			$f(x) = \ln x$. $f'(x) = \dfrac{1}{x} = x^{-1}$. $f^{(n)}(x^{-1}) = (-1)(-2)\dots(-n)x^{-1-n} = (-1)^n \cdot n! \cdot x^{-1 - n}$, Тогда получаем, что

			\begin{equation*}
				f^{(n)}(x) = f^{(n - 1)}(x - 1) = (-1)^{n - 1} \cdot (n - 1)! \cdot x^{-n}
			\end{equation*}
		\end{itemize}

		\begin{namedtheorem}[(Формула Лейбница)]
			Пусть $u(x)$ и $v(x)$ имеют не менее $n$ производных на множестве $E$. Тогда
			\begin{equation*}
				(u \cdot v)^{(n)} = \sum_{k = 0}^n \binom{n}{k} \cdot u^{(n - k)} \cdot v^{(k)}
			\end{equation*}
		\end{namedtheorem}

		\begin{proof}
			Докажем по индукции. При $n = 1$
			\begin{equation*}
				(u \cdot v)' = u'v + uv' = \sum_{k = 0}^1 \binom{1}{k} \cdot u^{(1 - k)} \cdot v^{(k)}
			\end{equation*}

			Пусть равенство верно при некотором $n$, докажем его справедливость при $n = n + 1$. Беря по определению производную $(u \cdot v)^{(n + 1)}$

			\begin{align*}
				(u \cdot v)^{(n + 1)} 
				&= \left(\sum_{k = 0}^n \binom{n}{k} \cdot u^{(n - k)} \cdot v^{(k)}\right)' \\
				&= \sum_{k = 0}^n \binom{n}{k} \cdot u^{(n - k + 1)} \cdot v^{(k)} + \sum_{k = 0}^n \binom{n}{k} \cdot u^{(n - k)} \cdot v^{(k + 1)} \\
				&= \sum_{k = 0}^n \binom{n}{k} \cdot u^{(n - k + 1)} \cdot v^{(k)} + \sum_{k = 1}^{n + 1} \binom{n}{k - 1} \cdot u^{(n - k + 1)} \cdot v^{(k)} \\
				&= \binom{n}{0} \cdot u^{(n + 1)} \cdot v + \sum_{k = 1}^n \left(\binom{n}{k} + \binom{n}{k - 1}\right) \cdot u^{(n - k + 1)} \cdot v^{(k)} + \binom{n}{n} \cdot u \cdot v^{(n + 1)} \\
				&= \binom{n}{0} \cdot u^{(n + 1)} \cdot v + \sum_{k = 1}^n \binom{n + 1}{k} \cdot u^{(n - k + 1)} \cdot v^{(k)} + \binom{n}{n} \cdot u \cdot v^{(n + 1)} \\
				&= \sum_{k = 0}^{n + 1} \binom{n + 1}{k} \cdot u^{(n - k + 1)} \cdot v^{(k)}
			\end{align*}

			Последний переход сделан при помощи следующего рассуждения: 
			\begin{equation*}
				\binom{n + 1}{k} = \binom{n}{k - 1} + \binom{n}{k}
			\end{equation*}
		\end{proof}

	\proofitem{Понятие об экстремумах функции одной переменной.}

		Точка $x_0$ называется точкой локального максимума (минимума) функции $f$, если существует такая окрестность $U_{\delta} (x_0)$ точки $x_0$, что 
		\begin{equation*}
			\forall x \in U_{\delta}(x_0) \implies f(x) \leq f(x_0)
		\end{equation*}

		и точкой локального минимума, если
		\begin{equation*}
			\forall x \in U_{\delta}(x_0) \implies f(x) \geq f(x_0)
		\end{equation*}

		Точка $x_0$ будет называться точкой строгого локального экстремума, если заменить окрестность на проколотую окрестность и нестрогий знак заменить на строгий.

		\begin{namedtheorem}[Ферма]
			Если функция $y = f(x)$ имеет экстремум в точке $x_0$, то ее производная $f'(x_0)$ либо равна нулю, либо не существует.
		\end{namedtheorem}

	\proofitem{Локальный экстремум. Необходимое условие для внутреннего локального экстремума (теорема Ферма).}

		\begin{namedtheorem}[Ферма]
			Пусть функция $f$ определена на интервале $(a, b)$ и в некоторой точке $x_0 \in (a, b)$ принимает наибольшее (наименьшее) значение на этом интервале. Если существует $f'(x_0)$, то $f'(x_0) = 0$.
		\end{namedtheorem}

		\begin{proof}
			Пусть $x_0$ --- точка максимума функции $f$. Рассмотрим разностное отношение $\dfrac{f(x) - f(x_0)}{x - x_0}$. Так как $f(x) \leq f(x_0)$, то при $x > x_0$ имеем $\dfrac{f(x) - f(x_0)}{x - x_0} \leq 0$, и, следовательно, $f_+'(x_0) \leq 0$. Если же $x < x_0$, то $\dfrac{f(x) - f(x_0)}{x - x_0} \geq 0$, и поэтому $f_-'(x_0) \geq 0$. Но из дифференцируемости функции $f$ в точке $x_0$ следует, что $f_+'(x_0) = f_-'(x_0) = f'(x_0)$ (следует из равности предела справа и слева).
		\end{proof}

		С геометрической точки зрения теорема Ферма означает, что если в точке экстремума у графика функции существует касательная, то она параллельна оси $OX$.

	\proofitem{Основные теоремы о дифференцируемых функций на отрезке (теорема Ролля, формулы Лагранжа и Коши.}

		\begin{namedtheorem}[Ролля. О нуле производной функции, принимающей на концах отрезка равные значения]
			Пусть функция $y = f(x)$

			\begin{enumerate}
				\item
				непрерывна на отрезке $[a, b]$;

				\item
				дифференцируема на интервале $(a, b)$;

				\item
				$f(a) = f(b)$
			\end{enumerate}

			Тогда на интервале $(a, b)$ найдется, по крайней мере, одна точка $x_0$, в которой $f'(x_0) = 0$.
		\end{namedtheorem}

		\begin{proof}
			Если функция $f(x)$ постоянна на отрезке $[a, b]$ (а значит, её минимальное и максимальное значение совпадают), то производная равна нулю в любой точке интервала $(a, b)$, в этом случае утверждение справедливо.

			Иначе, минимальное и максимальное значение функции не совпадают. По второй теореме Вейерштрасса (о достижении функции значения точной верхней/нижней грани на отрезке), функция достигает своего наибольшего или наименьшего значения в точке $\xi$ интервала $(a, b)$, т.е. в точке $\xi$ существует локальный экстремум. Тогда по теореме Ферма производная в этой точке равна нулю
			\begin{equation*}
				f'(\xi) = 0
			\end{equation*}
		\end{proof}

		\begin{namedtheorem}[Лагранжа. Формула конечных приращений]
			Если функция $f(x)$ непрерывна на отрезке $[a, b]$ и дифференцируема на интервале $(a, b)$, то в этом интервале существует хотя бы одна точка $x_0$, что
			\begin{equation*}
				\dfrac{f(b) - f(a)}{b - a} = f'(x_0)
			\end{equation*}
		\end{namedtheorem}

		\begin{proof}
			Рассмотрим вспомогательную функцию $F(x) = f(x) + \lambda x$.

			Выберем число $\lambda$ таким, чтобы выполнялось условие $F(a) = F(b)$, тогда
			\begin{equation*}
				f(a) + \lambda a = f(b) + \lambda b 
				\implies f(b) - f(a) = \lambda(a - b) 
				\implies \lambda = -\dfrac{f(b) - f(a)}{b - a}
			\end{equation*}

			В результате получаем
			\begin{equation*}
				F(x) = f(x) - \dfrac{f(b) - f(a)}{b - a} \cdot x
			\end{equation*}

			Функция $F(x)$ непрерывна на отрезке $[a, b]$, дифференцируема на интервале $(a, b)$ и принимает одинаковые значения на концах отрезка. Следовательно, для неё выполнены все условия теоремы Ролля. Тогда в интервале $(a, b)$ существует такая точка $\xi$, что $F'(\xi) = 0$.

			Отсюда следует, что $0 = f'(\xi) - \dfrac{f(b) - f(a)}{b - a}$ или
			\begin{equation*}
				f'(\xi) = \dfrac{f(b) - f(a)}{b - a}
			\end{equation*}
		\end{proof}

		Теорема Лагранжа имеет простой геометрический смысл. Хорда, проходящая через точки графика, соответствующие концам отрезка $a$ и $b$ имеет угловой коэффициент, равный
		\begin{equation*}
			k = \tg \alpha = \dfrac{f(b) - f(a)}{b - a}
		\end{equation*}

		Тогда внутри отрезка существует точка $x = \xi$, в которой касательная к графику параллельна хорде.

		\begin{namedtheorem}[Коши. Обобщает формулу конечных приращений Лагранжа.]
			Пусть функции $f(x)$ и $g(x)$ непрерывны на отрезке $[a, b]$ и дифференцируемы на интервале $(a, b)$, причем $g'(x) \neq 0$ при всех $x \in (a, b)$. Тогда в этом интервале существует точка $x = \xi$ такая, что
			\begin{equation*}
				\dfrac{f(b) - f(a)}{g(b) - g(a)} = \dfrac{f'(\xi)}{g'(\xi)}
			\end{equation*}
		\end{namedtheorem}

		\begin{proof}
			\textbf{Доказательство совпадает с доказательством теормы Лагранжа}.


			Прежде всего заметим, что знаменатель в левой части формулы Коши не равен нулю: $g(b) - g(a) \neq 0$. Действительно, если $g(a) = g(b)$, то по теореме Ролля найдется точка $\mu \in (a, b)$, в которой $g'(\mu) = 0$. Это, однако, противоречит условию, где указано, что $\forall x \in (a, b): \ g'(x) \neq 0$.

			Введем вспомогательную функцию $F(x) = f(x) + \lambda g(x)$.

			Выберем число $\lambda$ таким, чтобы выполнялось условие $F(a) = F(b)$, тогда
			\begin{equation*}
				f(a) + \lambda g(a) = f(b) + \lambda g(b) 
				\implies f(b) - f(a) = \lambda(g(a) - g(b)) 
				\implies \lambda = -\dfrac{f(b) - f(a)}{g(b) - g(a)}
			\end{equation*}

			В результате получаем
			\begin{equation*}
				F(x) = f(x) - \dfrac{f(b) - f(a)}{g(b) - g(a)} \cdot g(x)
			\end{equation*}

			Функция $F(x)$ непрерывна на отрезке $[a, b]$, дифференцируема на интервале $(a, b)$ и при найденном значении $\lambda$ принимает одинаковые значения на концах отрезка. Следовательно, для неё выполнены все условия теоремы Ролля. Тогда в интервале $(a, b)$ существует такая точка $\xi$, что $F'(\xi) = 0$.

			Отсюда следует, что
			\begin{equation*}
				0 = f'(\xi) - \dfrac{f(b) - f(a)}{g(b) - g(a)} \cdot g'(\xi)
			\end{equation*}

			или
			\begin{equation*}
				\dfrac{f'(\xi)}{g'(\xi)} = \dfrac{f(b) - f(a)}{g(b) - g(a)}
			\end{equation*}
		\end{proof}

	\proofitem{Многочлен Тейлора и формула Тейлора для функций одной переменной с остаточным членом в форме Пеано и Лагранжа.}

		Предположим, что имеется некоторая функция $f(x)$ и надо исследовать ее поведение в некоторой точке $x_0$ или ее окрестности. Сама функция может быть при этом достаточно сложной, и поэтому непосредственное вычисление $\lim_{x \to x_0} f(x)$ (как пример того, что мы хотим узнать о функции в $x_0$) окажется крайне трудоемким. Идея в том, чтобы найти такой многочлен $P_n(x)$, что $f(x) \sim P_n(x - x_0)$ при $x \to x_0$, а затем исследовать его. Работать с многочленами практически всегда намного проще.

		Предположим пока, что $x_0 = 0$. Тогда $P_n(x) = c_0 + c_1x + c_2x^2 + \dots + c_nx^n$. $P_n(0) = c_0$, а $P_n'(x) = c_1 + 2c_2x + \dots + nc_nx^{n - 1}$, из чего следует, что $c_1 = P_n'(0)$. По аналогии можно получить, что $c_2 = \dfrac{P_n''(0)}{2!}, \dots, c_n = \dfrac{P_n^{(n)}(0)}{n!}$. Т.е. получаем, что $P_n(x) = P_n(0) + \dfrac{P_n'(0)}{1!}x + \dots + \dfrac{P_n^{(n)}(0)}{n!} x^n$.

		Пусть $\exists f^{(n)}(x_0)$, тогда справедлива формула:
		\begin{equation*}
			f(x) = f(x_0) + \dfrac{f'(x_0)}{1!}(x - x_0) + \dfrac{f''(x_0)}{2!}(x - x_0)^2 + \dots + \dfrac{f^{(n)}(x_0)}{n!}(x - x_0)^n + r_n(f, x)
		\end{equation*}

		Эта формула называется \textbf{формулой Тейлора} и обычно записывается в виде:
		\begin{equation*}
			f(x) = \underbrace{\sum_{k = 0}^n \dfrac{f^{(k)}(x_0)}{k!} \cdot (x - x_0)^k}_{\text{многочлен Тейлора}} + \underbrace{r_n(f, x)}_{\text{остаточный член}}
		\end{equation*}

		\begin{lemma}
			Пусть $\exists f^{(n)}(x_0)$ и $\exists f'(x)$ на некоторой $U(x_0)$. Тогда $(r_n(f, x))' = r_{n - 1}(f', x)$.
		\end{lemma}

		\begin{proof}
			\[\begin{gathered}
				r_n(f, x) = f(x) - \sum_{k = 0}^n \dfrac{f^{(k)}(x_0)}{k!} (x - x_0)^k \\
				(r_n(f, x))' = f'(x) - \sum_{k = 1}^n \dfrac{f^{(k)}(x_0)}{(k - 1)!} (x - x_0)^{k - 1}
				= f'(x) - \sum_{k = 0}^{n - 1} \dfrac{f^{(k + 1)}(x_0)}{k!}(x - x_0)^k = r_{n - 1}(f', x)
			\end{gathered}\]
			так как $f^{(k + 1)}(x_0) = (f')^{(k)}(x_0)$. Следует также обратить внимание на то, что дифференцирование $r_n(f, x)$ происходит по $x$, поэтому все члены суммы, кроме $(x - x_0)^k$, --- константы.
		\end{proof}

		\begin{namedtheorem}[о локальной форме остаточного члена (Форма Пеано)]
			Пусть $\exists f^{(n)}(x_0)$ и $\exists f^{(n - 1)}(x)$ на некоторой $U(x_0)$. Тогда справедлива формула Тейлора, причем $r_n(f, x) = \os((x - x_0)^n), x \to x_0$.
		\end{namedtheorem}

		\begin{proof}
			Докажем с помощью метода математической индукции. При $n = 1$, $f(x) = f(x_0) + f'(x_0)(x - x_0) + \os(x - x_0)$, что верно, т.к. $f(x)$ дифференцируема в точке $x_0$. Предположим теперь, что теорема верна для \textbf{произвольной функции} $f$ при $n = n - 1$, и докажем её при $n = n$.

			Заметим сначала, что $r_n(f, x_0) = 0$ (следует из обычной формулы Тейлора). Тогда $r_n(f, x) = r_n(f, x) - r_n(f, x_0) = (r_n(f, \xi))'(x - x_0)$, где $\xi$ принадлежит интервалу $(\min\{x, x_0\}, \max\{x, x_0\})$ по теореме Лагранжа.

			По лемме получаем, что $(r_n(f, \xi))'(x - x_0) = r_{n - 1}(f', \xi)(x - x_0)$. По предположению для произвольной функции $f$, у которой есть $n$-ая производная в $x_0$ и $(n - 1)$-ая в окрестности $x_0$, можно выполнить индукционный переход для $f'$, т.к. для $r_{n - 1}$ у $f'(x)$ существуют $(n - 1)$-ая производная в $x_0$ и $(n - 2)$-ая в окрестности $x_0$. Тогда $r_{n - 1}(f', \xi)(x - x_0) = \os((\xi - x_0)^{n - 1})(x - x_0) = [|\xi - x_0| < |x - x_0| \implies \os((\xi - x_0)^{n - 1}) = \os((x - x_0)^{n - 1})] = \os((x - x_0)^{n - 1})(x - x_0) = \os((x - x_0)^n)$
		\end{proof}

		\begin{namedtheorem}[о форме Лагранжа]
			Пусть $n \in \NN \cup \{0\}$ и $\exists f^{(n)}(x)$, причем $f^{(n)}(x)$ непрерывна на отрезке $[x_0, x]$. Кроме того, $\exists f^{(n + 1)}(x)$ на $(x_0, x)$. Тогда справедлива формула Тейлора, причем $r_n(f, x) = \dfrac{f^{(n + 1)}(\xi)}{(n + 1)!}(x - x_0)^{n + 1}$, где $\xi \in (x_0, x)$.
		\end{namedtheorem}

		\begin{proof}
			Снова воспользуемся методом математической индукции. При $n = 0$, $f(x) = f(x_0) + f'(\xi)(x - x_0)$ --- формула Лагранжа. Предположим теперь, что для произвольной функции $f$ справедливо, что $r_{n - 1}(f, x) = \dfrac{f^{(n)}(\xi)}{n!}(x - x_0)^n$, где $\xi \in (x_0, x)$. При $n = n$ имеем:

			\[\begin{gathered}
				\dfrac{r_n(f, x)}{(x - x_0)^{n + 1}} = \dfrac{r_n(f, x) - r_n(f, x_0)}{(x - x_0)^{n + 1} - (x_0 - x_0)^{n + 1}} \text{ [по формуле Коши] } = \\
				= \dfrac{(r_n(f, \mu))'}{(n + 1)(\mu - x_0)^n} \text{ [по лемме, доказанной выше] } = \\
				= \dfrac{r_{n - 1}(f', \mu)}{(n + 1)(\mu - x_0)^n} = \dfrac{(f'(\xi))^{(n)}}{(n + 1)n!} \implies r_n(f, x) = \dfrac{f^{(n + 1)}(\xi)}{(n + 1)!}(x - x_0)^{n + 1}
			\end{gathered}\]
		\end{proof}

	\proofitem{Формулы Маклорена для основных элементарных функций (без доказательства).}

		При $x_0 = 0$ формула Тейлора с остаточным членом в форме Пеано называется формулой Маклорена

		Приведем пример: $f(x) = \sin x$. Вспомним, что
		\begin{equation*}
			f^{(n)}(x) = \sin \left(x + \dfrac{\pi n}{2}\right) \implies f^{(n)}(0) = \sin \left(\dfrac{\pi n}{2}\right) 
			= \begin{cases}
				0, \quad &\text{если $n = 2k$}, \\
				(-1)^k, \quad &\text{если $n = 2k + 1$}
			\end{cases}
		\end{equation*}

		Тогда получаем следующее разложение:
		\begin{equation*}
			\sin x = x - \dfrac{x^3}{3!} + \dfrac{x^5}{5!} - \dots + (-1)^{n - 1} \cdot \dfrac{x^{2n - 1}}{(2n - 1)!} + \os(x^{2n})
		\end{equation*}

		\begin{enumerate}[leftmargin=*]
			\item
			$e^x = 1 + \dfrac{x}{1!} + \dfrac{x^2}{2!} + \dots + \dfrac{x^n}{n!} + \os(x^n), x \to 0$

			\item
			$\ln(1 + x) = x - \dfrac{x^2}{2} + \dfrac{x^3}{3} - \dots + (-1)^{n + 1} \cdot \dfrac{x^n}{n} + \os(x^n), x \to 0$

			\item
			$\underset{\alpha \in \RR}{(1 + x)^{\alpha}} = 1 + \sum_{k = 1}^{n} \dbinom{\alpha}{k} x^k + \os(x^n)$

			Например $(1 + x)^{\frac{1}{3}} - 1 = \dbinom{\frac{1}{3}}{1}x + \dbinom{\frac{1}{3}}{2} x^2 + \os(x^2) = \dfrac{1}{3} x + \dfrac{\frac{1}{3}(\frac{1}{3} - 1)}{2} x^2 + \os(x^2)$

			\item
			$\sin (x) = x - \dfrac{x^3}{3!} + \dfrac{x^5}{5!} - \dots + (-1)^{n - 1} \cdot \dfrac{x^{2n - 1}}{(2n - 1)!} + \os(x^{2n})$

			\item
			$\cos (x) = 1 - \dfrac{x^2}{2!} + \dfrac{x^4}{4!} + \dots + (-1)^{n} \dfrac{x^{2n}}{(2n)!} + \os(x^{2n + 1})$

			\item
			$\tg (x) = x + \dfrac{x^3}{3} + \dfrac{2}{15} x^5 + \dots + \dfrac{B_{2n}(-4)^n(1 - 4^n)}{(2n)!} \cdot x^{2n - 1} + \os(x^{2n - 1})$, где $B_{2n}$ --- числа Бернулли

			Но достаточно помнить, что
			$\tg (x) = x + \dfrac{x^3}{3} + \dfrac{2}{15}x^5 + \os(x^5)$, т.е. общая формула для семинаров \underline{не} нужна

			\item
			$\arcsin (x) = x + \dfrac{x^3}{6} + \dfrac{3}{40} x^5 + \dots + \dfrac{(2n)!}{4^n (n!)^2 (2n + 1)} \cdot x^{2n + 1} + \os(x^{2n + 1})$

			Достаточно знать $\arcsin (x) = x + \dfrac{x^3}{6} + \dfrac{3}{40} x^5 + \os(x^5)$

			\item
			$\arccos (x) = \dfrac{\pi}{2} - \arcsin (x)$

			\item
			$\arctg(x) = x - \dfrac{x^3}{3} + \dfrac{x^5}{5} + \dots + (-1)^{n + 1} \dfrac{x^{2n - 1}}{2n - 1} + \os(x^{2n - 1})$
		\end{enumerate}

	\proofitem{Правило Лопиталя.}
		\begin{namedtheorem}[Лопиталя (первое правило)]
			Если функции $f(x)$ и $g(x)$ таковы, что

			\begin{enumerate}
				\item
				$f(x)$ и $g(x)$ дифференцируемы в проколотой окрестности точки $a$

				\item
				$\lim_{x \to a} f(x) = \lim_{x \to a} g(x) = 0$

				\item
				$g'(x) \neq 0$ в окрестности $U(a)$

				\item
				Существует $\lim_{x \to a} \dfrac{f'(x)}{g'(x)}$
			\end{enumerate}

			Тогда существует $\lim_{x \to a} \dfrac{f(x)}{g(x)} = \lim_{x \to a} \dfrac{f'(x)}{g'(x)}$
		\end{namedtheorem}

		\begin{proof}
			Доопределим функции в точке $a$ нулём (непрерывности не нарушится, так как предел этих функций при $x \to a$ равен 0). Из первого условия следует, что $f(x)$ и $g(x)$ непрерывны на отрезке $[a, x]$, где $x$ принадлежит рассматриваемой окрестности точки $a$. 

			Применим обобщённую формулу конечных приращений (Коши) к $f(x)$ и $g(x)$ на отрезке $[a, x]$.
			\begin{equation*}
				\exists \xi \in [a, x]: \ \dfrac{f(x) - f(a)}{g(x) - g(a)} = \dfrac{f'(\xi)}{g'(\xi)}
			\end{equation*}

			Так как $g(a) = f(a) = 0$ получим, что $\forall x \ \exists \xi \in [a, x]: \dfrac{f(x)}{g(x)} = \dfrac{f'(\xi)}{g'(\xi)}$.

			По определению предела, $\lim_{x \to a + 0} \dfrac{f'(x)}{g'(x)} = A \iff \forall \eps > 0 \ \exists \delta > 0: \ \forall x: \ a < x < a + \delta \implies \left|\dfrac{f'(x)}{g'(x)} - A\right| < \eps$. Но для каждого $x$ из указанного интервала найдется своё $\xi_x$, такое что $\dfrac{f'(\xi_x)}{g'(\xi_x)} = \dfrac{f(x)}{g(x)}$. Но раз $\xi_x \in (a, x)$, то выполняется $\left|\dfrac{f'(\xi_x)}{g'(\xi_x)} - A\right| < \eps \implies \left|\dfrac{f(x)}{g(x)} - A\right|$, что и требовалось доказать.
		\end{proof}

		\begin{namedtheorem}[Лопиталя (второе правило)]
			Если для функций $f(x)$ и $g(x)$ справедливо следующее:

			\begin{enumerate}
				\item
				$f(x)$ и $g(x)$ дифференцируемы на интервале $(a, b)$

				\item
				$\lim_{x \to a + 0} f(x) = \lim_{x \to a + 0} g(x) = \infty$

				\item
				$g'(x) \neq 0$ при $x \in (a, b)$

				\item
				Существует $\lim_{x \to a + 0} \dfrac{f'(x)}{g'(x)} = A$
			\end{enumerate}

			то $\lim_{x \to a + 0} \dfrac{f(x)}{g(x)} = \lim_{x \to a + 0} \dfrac{f'(x)}{g'(x)} = A$
		\end{namedtheorem}

		\begin{proof}
			Для начала положим, что $A \leq 0$ (при $A > 0$ доказательство практически аналогично приведенному). Пусть $\eps \in \left(0, \dfrac{1}{4}\right)$. Тогда по определению предела 
			\begin{equation*}
				\lim_{x \to a + 0} \dfrac{f'(x)}{g'(x)} = A \iff \exists x_{\eps} \in (a, b): \ \forall x \in (a, x_{\eps}) \implies \left|\dfrac{f'(x)}{g'(x)} - A\right| < \eps
			\end{equation*}

			Здесь мы просто приняли, что $x_{\eps} = a + \delta$, в остальном же интерпретация определения предела не изменилась.

			Выберем произвольное $x$ из данного интервала $(a, x_{\eps})$. Заметим, что выполняется теорема Коши (доопределим функции $f$ и $g$ в точке $a$, а в точке $x_{\eps}$ они уже определены):

			\[\begin{gathered}
				\dfrac{f(x) - f(x_{\eps})}{g(x) - g(x_{\eps})} = \dfrac{f'(\xi)}{g'(\xi)} \text{, где } a < x < \xi < x_{\eps} < b \\
				\dfrac{f(x)}{g(x)} \cdot \dfrac{1 - \dfrac{f(x_{\eps})}{f(x)}}{1 - \dfrac{g(x_{\eps})}{g(x)}} = \dfrac{f'(\xi)}{g'(\xi)} \\
				\dfrac{f(x)}{g(x)} = \dfrac{f'(\xi)}{g'(\xi)} \cdot \dfrac{1 - \dfrac{g(x_{\eps})}{g(x)}}{1 - \dfrac{f(x_{\eps})}{f(x)}}
			\end{gathered}\]

			Заметим теперь, что $\lim_{x \to a + 0} \dfrac{f(x_{\eps})}{f(x)} = \lim_{x \to a + 0} \dfrac{g(x_{\eps})}{g(x)} = 0$, т.к. $f(x_{\eps})$ и $g(x_{\eps})$ --- константы (а знаменатели по условию стремятся к $\infty$). Тогда выберем для текущего закрепленного $\eps$ такое $\delta(\eps) > 0$:
			\begin{equation*}
				\forall x \in (a, a + \delta), \delta + a < b \implies \left|\dfrac{f(x_{\eps})}{f(x)}\right| < \eps, \left|\dfrac{g(x_{\eps})}{g(x)}\right| < \eps
			\end{equation*}

			Тогда получаем следующую оценку:
			\begin{equation*}
				\dfrac{1 - \dfrac{g(x_{\eps})}{g(x)}}{1 - \dfrac{f(x_{\eps})}{f(x)}} \in \left(\dfrac{1 - \eps}{1 + \eps}, \dfrac{1 + \eps}{1 - \eps}\right)
			\end{equation*}

			Поскольку $\eps \in \left(0, \dfrac{1}{4}\right)$, то $\dfrac{1 - \eps}{1 + \eps} = 1 - \dfrac{2\eps}{1 + \eps} > 1 - 2\eps$ и $\dfrac{1 + \eps}{1 - \eps} = 1 + \dfrac{2\eps}{1 - \eps} < 1 + \dfrac{8}{3} \eps$. Учитывая, что $\dfrac{f'(x)}{g'(x)} \in \overset{\circ}{U}_\eps(A)$:
			\[\begin{gathered}
				\dfrac{f(x)}{g(x)} = \dfrac{f'(\xi)}{g'(\xi)} \cdot \dfrac{1 - \dfrac{g(x_{\eps})}{g(x)}}{1 - \dfrac{f(x_{\eps})}{f(x)}} \in \left((A - \eps)(1 - 2\eps), (A + \eps) \left(1 + \dfrac{8}{3} \eps\right)\right) = \\
				= \left(A - (\eps + 2A\eps - 2\eps^2), A + \left(\eps + \dfrac{8}{3} \eps A + \dfrac{8}{3}\eps^2\right)\right) \implies \\
				\implies \dfrac{f(x)}{g(x)} \in U_{\mu}(A), \text{ где } \mu = \max \left\{\eps + 2A\eps - 2\eps^2, \eps + \dfrac{8}{3}\eps A + \dfrac{8}{3}\eps^2\right\}
			\end{gathered}\]

			Как видно, $\lim_{\eps \to 0} \mu = 0$, а для любого сколько угодно малого $\mu$ всегда можно найти соответствующее $\eps$, такое, что все значения отношения функций попадут в заданную $\mu$-трубку. Это и означает, что предел отношения функции равен $A$.
		\end{proof}
	
	\proofitem{Достаточное условие строгого возрастания (убывания) функции на промежутке.}

		\begin{theorem*}
			Для того чтобы дифференцируемая функция $f(x)$ на интервале $(a, b)$ строго возрастала, достаточно, чтобы $\forall x \in (a, b): \ f'(x) > 0$

			Для того чтобы дифференцируемая функция $f(x)$ на интервале $(a, b)$ строго убывала, достаточно, чтобы $\forall x \in (a, b): \ f'(x) < 0$
		\end{theorem*}

		\begin{proof}
			Докажем для строгого возрастания. Пусть $f'(x) > 0 \ \forall x \in (a, b)$. Выберем произвольные точки $x_1, x_2 \in (a, b)$, и, не ограничивая общности, скажем, что $x_1 < x_2$.

			Применим формулу конечных приращений Лагранжа. Так как $f'(\xi) > 0$ и $x_2 > x_1$, имеем
			\begin{equation*}
				f(x_2) - f(x_1) = f'(\xi) \cdot (x_2 - x_1) > 0 \implies f(x_2) > f(x_1)
			\end{equation*}
		\end{proof}

    \end{colloq}

    \section{Вопросы, которые были убраны из программы коллоквиума}

    \begin{colloq}
    
    \proofitem{Эквивалентность определений предела функции по Коши и Гейне} 

    	\begin{theorem*}
    		Определения предела функции в точке по Коши и Гейне эквивалентны
    	\end{theorem*}

    	\begin{proof}
    		Пусть $f$ определена на множестве $X$ и число $A$ является пределом функции $f$ в точке $x_0$ в смысле Коши. Выберем произвольную подходящую последовательность $x_n, n \in \NN$, т.е. такую, для которой $\forall n \in \NN: \ x_n \in X$ и $\lim_{n \to \infty} x_n = x_0$. Покажем, что $A$ является пределом в смысле Гейне.

    		Зададим произвольное число $\eps > 0$ и укажем для него такое $\delta > 0$, что $\forall x \in X$ из условия $0 < |x - x_0| < \delta$ следует неравенство $|f(x) - A| < \eps$. В силу того, что $\lim_{n \to \infty} x_n = x_0$, для $\delta > 0$ найдется такой номер $N \in \NN$, что для всех $n \geq N$ будет выполняться неравенство $|f(x_n) - A| < \eps$, т.е. $\lim_{n \to \infty} f(x_n) = A$.


    		Докажем теперь обратное утверждение: предположим, что $A = \lim_{x \to x_0} f(x)$ в смысле Гейне, и покажем, что число $A$ является пределом функции $f$ в точке $x_0$ в смысле Коши. Предположим, что это неверно, т.е.
    		\begin{equation*}
    			\exists \eps_0 > 0 \ \forall \delta > 0 \ \exists x_{\delta} \in X: \ 0 < |x_{\delta} - x_0| < \delta: \ |f(x_{\delta}) - A| \geq \eps
    		\end{equation*}

    		В качестве $\delta$ рассмотрим $\delta = \dfrac{1}{n}$, а соответствующие значения $x_{\delta}$ будем обозначать $x_n$. Тогда при любом $n \in \NN$ выполняются условия $x_n \neq x_0, |x_n - x_0| < \dfrac{1}{n}$ и $|f(x_n) - A| \geq \eps$. Отсюда следует, что последовательность $\{x_n\}$ является подходящей, но число $A$ не является пределом функции $f$ в точке $x_0$. Получили противоречие.
    	\end{proof}

    	\proofitem{Теорема о дифференцируемости обратной функции.}

		\begin{theorem*}
			Рассмотрим функцию $f(x)$, которая является строго монотонной на некотором интервале $(a, b)$. Если в этом интервале существует такая точка $x_0$, что $f'(x_0) \neq 0$, то функция $x = \phi(y)$, обратная к функции $y = f(x)$, также дифференцируема в точке $y_0 = f(x_0)$ и её производная равна
			\begin{equation*}
				\phi'(y_0) = \dfrac{1}{f'(x_0)}
			\end{equation*}
		\end{theorem*}

		\begin{proof}
			Пусть переменная $y$ в точке $y_0$ получает приращение $\Delta y \neq 0$. Соответствующее ему приращение переменной $x$ в точке $x_0$ обозначим как $\Delta x$, причем $\Delta x \neq 0$ в силу строгой монотонности функции $y = f(x)$. Запишем отношение приращений в виде
			\begin{equation*}
				\dfrac{\Delta x}{\Delta y} = \dfrac{1}{\frac{\Delta y}{\Delta x}}
			\end{equation*}

			Допустим, что $\Delta y \to 0$, тогда $\Delta x \to 0$, поскольку обратная функция $x = \phi(y)$ является непрерывной в точке $y_0$. В пределе, при $\Delta x \to 0$, правая часть записанного соотношения становится равной
			\begin{equation*}
				\lim_{\Delta x \to 0} \dfrac{1}{\frac{\Delta y}{\Delta x}} = \dfrac{1}{f'(x_0)}
			\end{equation*}

			В таком случае левая часть тоже стремится к пределу, который по определению равен производной обратной функции:
			\begin{equation*}
				\lim_{\Delta y \to 0} \dfrac{\Delta x}{\Delta y} = \phi'(y_0)
			\end{equation*}

			Таким образом,
			\begin{equation*}
				\phi'(y_0) = \dfrac{1}{f'(x_0)}
			\end{equation*}
		\end{proof}

	\proofitem{Производные функций, графики которых заданы параметрически.}

		\begin{theorem*}
			Зависимость между аргументом $x$ и функцией $y$ может быть задана в параметрическом виде с помощью двух уравнений:
			\begin{equation*}
				\begin{cases}
					x = \phi(t), \\
					y = \psi(t) \\
				\end{cases}
			\end{equation*}

			Пусть $x = \phi(t)$ и $y = \psi(t)$ определены и дифференцируемы при $t \in (a, b)$, причем $x'_t = \phi'(t) \neq 0$ и $x = \phi(t)$ имеет обратную функцию $t = \theta(x)$, то
			\begin{equation*}
				y'_x = \dfrac{\psi'(t)}{\phi'(t)}
			\end{equation*}
		\end{theorem*}

		\begin{proof}
			Перейдем от параметрического задания к явному. При этом получаем сложную функцию $y = \psi(t) = \psi(\theta(x))$, аргументов которой является $x$.

			По правилу нахождения производной сложной функции имеем
			\begin{equation*}
				y'_x = (\psi(\theta(x)))' = \psi'(\theta(x)) \cdot \theta'(x)
			\end{equation*}

			По теореме об обратной функции $\theta'(x) = \dfrac{1}{\phi'(t)}$. А значит
			\begin{equation*}
				y'_x = \psi'(\theta(x)) \cdot \theta'(x) = \dfrac{\psi'(t)}{\phi'(t)}
			\end{equation*}
		\end{proof}

	\proofitem{Геометрический смысл дифференциала.}

		Дифференциал функции численно равен приращению ординаты касательной, проведенной к графику функции $y = f(x)$ в данной точке, когда аргумент $x$ получает приращение $\Delta x$.

		\href{https://lms2.sseu.ru/courses/eresmat/metod/met1/razdmet1_4/parmet1_4_2.htm}{Подробнее тут}

    \end{colloq}

\end{document}