\ProvidesFile{quest-09.tex}[Билет 9]

\section{Билет 9}

\begin{center}
    \it Случайные векторы с абсолютно непрерывным распределением и плотность совместного распределения.
    Связь с функцией совместного распределения.
    Вычисление плотности компонент по совместной плотности.
    Плотность случайного вектора, являющегося функцией от другого случайного вектора.
    Равномерное распределение на многомерных областях.
\end{center}

\sectionbreak
\subsection{Случайные векторы с абсолютно непрерывным распределением и плотность совместного распределения}

\begin{definition}
    Если существует такая интегрируемая и неотрицательная функция $\rho_{X, Y}(x, y)$, что
    \[
        F_{X, Y}(x, y) = \iint\limits_{(-\infty, x] \times (-\infty, y]} \rho_{X, Y}(x, y) \dd x \dd y,
    \]
    то говорят, что совместное распределение случайных величин $X, Y$ {\it абсолютно непрерывно}.
    Функцию $\rho_{X, Y}$ называют {\it плотностью} совместного распределения случайных величин $X, Y$ (случайного вектора).
\end{definition}

\sectionbreak
\subsection{Связь с функцией совместного распределения}

Ясно, что
\[
    P(a < X \leqslant b,~c < Y \leqslant d) = \mu_{X, Y}((a, b] \times (c, d]) = \iint\limits_{(a, b] \times (c, d]} \rho_{X,Y}(x, y) \dd x \dd y.
\]
Можно доказать, что
\[
    P((X, Y) \in B) = \mu_{X, Y}(B) = \iint\limits_B \rho_{X, Y}(x, y) \dd x \dd y.
\]
для всякого множества $A$, для которого имеет смысл интеграл Римана в правой части.
В каждой точке непрерывности плотности $\rho_{X, Y}$ выполнено равенство
\[
    \frac{\partial^2}{\partial x \partial y} F_{X, Y}(x, y) = \rho_{X, Y}(x, y).
\]

\sectionbreak
\subsection{Вычисление плотности компонент по совместной плотности}

Если известна плотность $\rho_{X, Y}$ совместного распределения $X$ и $Y$, то можно найти плотности распределения каждой из случайных величин.
Например, для случайной величины $X$:
\[
    F_X(t) = P(X \leqslant t,~Y \in \mathbb{R}) = \int_{-\infty}^t \Bigl( \int_{-\infty}^{+\infty} \rho_{X, Y}(x, y) \dd y \Bigr) \dd x.
\]
и, следовательно,
\[
    \rho_{X}(x) = \int_{-\infty}^{+\infty} \rho_{X, Y}(x, y) \dd y.
\]
Если распределение каждой из случайных величин задается плотностью, то совместное распределение
может не иметь плотность.

\sectionbreak
\subsection{Плотность случайного вектора, являющегося функцией от другого случайного вектора}

\begin{theorem}
    Пусть распределение $X, Y$ задано плотностью $\rho_{X, Y}$.
    Рассмотрим две случайные величины $\xi = f(X, Y), \eta = g(X, Y)$ и предположим, что отображение $T \colon (x,y) \mapsto (f(x,y), g(x,y))$ удовлетворяет условиям теоремы о замене переменных в кратном интеграле Римана (например непрерывно дифференцируемо с невырожденным якобианом).
    Тогда
    \[
        \rho_{\xi, \eta}(u, v) = \rho_{X, Y} \bigl( T^{-1}(u, v) \bigr) \cdot \abs{J \bigl( T^{-1}(u, v) \bigr)}^{-1},
    \]
    где $J$ --- якобиан отображения $T$.
\end{theorem}

\begin{proof}
    Заметим, что
    \[
        P((\xi, \eta) \in A) = P((X, Y) \in T^{-1}(A)) = \iint\limits_{T^{-1}(A)} \rho_{X, Y}(x, y) \dd x \dd y.
    \]
    Сделаем замену в интеграле $u = f(x, y), v = g(x, y)$, т.е. $(x, y) = T^{-1}(u, v)$.
    Тогда последний интеграл равен
    \[
        \iint\limits_{A} \rho_{X, Y}\bigl(T^{-1}(u, v)\bigr) \cdot \abs{J \bigl( T^{-1}(u, v) \bigr)}^{-1} \dd u \dd v,
    \]
    что завершает доказательство.
\end{proof}

\sectionbreak
\subsection{Равномерное распределение на многомерных областях}

Говорят, что вектор $(\xi, \eta)$ {\it равномерно распределен} на множестве $B$, имеющем положительную площадь, если его распределение задано плотностью
\[
    \rho(x, y) = \begin{cases}
        \frac{1}{\abs{B}}, & (x, y) \in B, \\
        0, & (x, y) \notin B.
    \end{cases}
\]
Аналогичным образом определяется равномерно распределенный вектор с любым конечным числом координат.
