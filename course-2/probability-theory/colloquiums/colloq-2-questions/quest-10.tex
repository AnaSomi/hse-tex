\ProvidesFile{quest-10.tex}[Билет 10]

\section{Билет 10}

\begin{center}
    \it Независимые случайные величины: характеризация в терминах функций распределения, в терминах совместного распределения, в терминах плотностей.
    Независимость функций от независимых случайных величин.
    Формула свертки для плотности суммы независимых случайных величин, заданных плотностями.
\end{center}

\sectionbreak
\subsection{Независимые случайные величины: характеризация в терминах функций распределения, в терминах совместного распределения, в терминах плотностей}

\paragraph{Напоминание}
Пусть задано вероятностное пространство $(\Omega,\mathcal A, P)$.
Функция $X:\Omega\rightarrow\R$ называется случайной величиной, если для всякого числа $t\in\R$ выполнено
\[
    X^{-1}((-\infty, t]) = \{\omega\in\Omega: X(\omega)\le t\}\in{A}
\]

\begin{definition*}
    Случайные величины $X$ и $Y$ называются {\it независимыми}, если:
    \[
        F_{X, Y}(x, y) = F_X (x) F_Y(y).
    \]
\end{definition*}

\begin{proposal*}
    Случайные величины $X$ и $Y$ независимы тогда и только тогда, когда для произвольных $U, V\in \mathcal{B}(\mathbb{R})$ выполнено:
    \[
        P(\{\omega\colon X(\omega)\in U, \, Y(\omega)\in V\}) = P(\{\omega\colon X(\omega)\in U\})\cdot P(\{\omega\colon Y(\omega)\in V\}.
    \]
\end{proposal*}

\begin{proof}
    Если $V = (-\infty, y]$, то две меры $ U\to\frac{\mu_{X, Y}(U\times V)}{\mu_Y(V)}$ и $U\to\mu_X(U)$ совпадают на всех лучах $(-\infty, x]$, т.е. имеют одинаковые функции распределения, а значит совпадают на всех борелевских множествах $U$.
    Теперь для произвольного борелевского множества $U$ меры $V\to \frac{\mu_{X, Y}(U\times V)}{\mu_X(U)}$ и $V\to\mu_Y(V)$ совпадают на всех лучах $(-\infty, y]$, а значит и на всех борелевских множествах $V$.
\end{proof}

\begin{proposal*}
    Пусть распределения $X$ и $Y$ заданы плотностями. Тогда независимость $X$ и $Y$ равносильна тому, что совместное распределение задано плотностью и эта плотность имеет вид:
    $$
    \rho_{X, Y}(x, y)=\rho_{X}(x)\rho_{Y}(y).
    $$
\end{proposal*}

\begin{proof}
    Если $X$ и $Y$ независимы, то
    $$
        F_{X, Y}(x,y)=F_X(x)\cdot F_Y(y)
        =\int_{-\infty}^x\rho_X(t)\, dt\cdot
        \int_{-\infty}^y\rho_Y(s)\, ds
        =
        \iint\limits_{(-\infty, x]\times(-\infty, y]}\rho_X(t)\rho_Y(s)\, dtds.
    $$
    Обратно,
    $$
        F_{X, Y}(x,y)=
        \iint\limits_{(-\infty, x]\times(-\infty, y]}\rho_X(t)\rho_Y(s)\, dtds
        =
        \int_{-\infty}^x\rho_X(t)\, dt\cdot
        \int_{-\infty}^y\rho_Y(s)\, ds
        =
        F_X(x)\cdot F_Y(y).
    $$
\end{proof}

\sectionbreak
\subsection{Независимость функций от независимых случайных величин}

\begin{definition*}
    Функция $f\colon \mathbb{R} \to \mathbb{R}$ называется борелевской, если $f^{-1}((-\infty, t])\in \mathcal{B}(\mathbb{R})$ для каждого $t\in \mathbb{R}$.
\end{definition*}

Например, такими функциями будут все монотонные функции или все непрерывные.

\begin{corollary*}
    Пусть $X$ и $Y$ независимы, а $f, g$ --- борелевские функции.
    Тогда $f(X)$ и $g(Y)$ также независимы.
\end{corollary*}

\sectionbreak
\subsection{Формула свертки для плотности суммы независимых случайных величин, заданных плотностями}

\begin{theorem*}[Формула свертки]
    Предположим, что $X$ и $Y$ независимы и их распределения заданы плотностями $\rho_X$ и $\rho_Y$. Тогда распределение суммы $Z = X + Y$ задано плотностью
    \[
        \rho_{Z}(z) = \int_{-\infty}^{+\infty} \rho_X(t) \rho_Y(z - t) \dd t.
    \]
\end{theorem*}

\begin{proof}
    По определению $F_{Z}(t) = P(\{\omega \mid X(\omega) + Y(\omega) \leqslant t\})$.
    С другой стороны, эта вероятность выражается через интеграл:
    \[
        F_{Z}(t) = P(\{\omega \mid X(\omega) + Y(\omega) \leqslant t\}) = \iint\limits_{x + y \leqslant t} \rho_{X}(x) \rho_{Y}(y) \dd x \dd y.
    \]
    Переходя к новым переменным $u = x + y$, $v = x$, и, применяя теорему Фубини\footnote{Также известная, как сведение двойного интеграла к повторному.}, преобразуем этот интеграл:
    \[
        F_{Z}(t) = \int_{-\infty}^t \Bigl(\int_{-\infty}^{+\infty} \rho_{X}(v) \rho_{Y}(u - v) \dd v \Bigr) \dd u.
    \]
    Следовательно, распределение $Z$ имеет плотность требуемого вида.
\end{proof}