\subsection{Неравенство Чебышева и закон больших чисел в слабой форме для общих случайных величин. Усиленный закон больших чисел Колмогорова (б/д). Сходимости случайных величин: почти наверное и по вероятности. Взаимосвязь сходимостей по вероятности и почти наверное.}

\subsubsection{Неравенство Чебышева и закон больших чисел в слабой форме для общих случайных величин.}

\begin{theorem*}[Неравенство Маркова]
    Пусть $X$ это случайная величина и $X \geq 0$ почти наверное. Тогда для любого $t > 0$ выполняется
    \begin{equation*}
        \P[X \geq t] \leq \dfrac{\E[X]}{t}.
    \end{equation*}
\end{theorem*}

\begin{proof}
    Заметим, что для любого $t > 0$ выполняется $t \cdot \I[x \geq t] \leq X$ почти наверное (здесь $\I$ это индикатор), так как в левой части будут учтены $t \leq X$, с суммарным коэффициентом не больше $1$.

    Возьмем математическое ожидание от обеих сторон и получим то, что нас просили:
    \begin{equation*}
        t \cdot \I[x \geq t] \leq X
        \iff t \cdot \P[x \geq t] \leq \E[X] 
        \iff \P[x \geq t] \leq \dfrac{\E[X]}{t}.
    \end{equation*}
\end{proof}

\begin{theorem*}[Неравенство Чебышева]
    Пусть у случайной величины $X$ конечный второй момент, то есть $\E[X^2] \leq \infty$. Тогда
    \begin{equation*}
        \P[|X - \E[X]| \geq \eps] \leq \dfrac{\D[X]}{\eps^2}.
    \end{equation*}
\end{theorem*}

\begin{proof}
    Для доказательства рассмотрим случайную величину $Y = |X - \E[X]|^2$ и применим неравенство Маркова.

    Для любого $\eps$ выполняется
    \begin{equation*}
        \P[Y \geq \eps^2] \leq \dfrac{\E[Y]}{\eps^2}
        \iff \P[|X - \E[X]|^2 \geq \eps^2] \leq \dfrac{\D[X]}{\eps^2}
        \iff \P[|X - \E[X]| \geq \eps] \leq \dfrac{\D[X]}{\eps^2}.
    \end{equation*}
\end{proof}

\begin{theorem*}[Закон Больших Чисел в слабой форме]
    Рассмотрим последовательность $\{X_n\}_n$ случайных независимых величин, что $\E[X_n^2] < \infty$ для любого $n$.

    Обозначим $\E[X_n] = a_n$ и $\D[X_n] = \sigma_n^2$. Если
    \begin{equation*}
        \lim_{n \to \infty} \dfrac{\sigma_1^2 + \dots + \sigma_n^2}{n^2} = 0,
    \end{equation*}
    то для всякого $\eps > 0$ выполняется
    \begin{equation*}
        \P\left[\left|\dfrac{X_1 + \dots + X_n}{n} - \dfrac{a_1 + \dots + a_n}{n}\right| \geq \eps\right] \leq \dfrac{\sigma_1^2 + \dots + \sigma_n^2}{n^2 \eps^2}.
    \end{equation*}
\end{theorem*}

\begin{proof}
    Рассмотрим случайную величину $X = \dfrac{X_1 + \dots + X_n}{n}$.

    По линейности математического ожидания получаем
    \begin{equation*}
        \E[X] = \dfrac{\E[X_1] + \dots + \E[X_n]}{n} = \dfrac{a_1 + \dots + a_n}{n}.
    \end{equation*}

    Теперь необходимо найди дисперсию случайной величины $X$:
    \begin{itemize}
    \item 
        Константа из дисперсии выносится с возведением в квадрат, поэтому
        \begin{equation*}
            \D[X] = \D\left[\dfrac{X_1 + \dots + X_n}{n}\right] = \dfrac{\D[X_1 + \dots + X_n]}{n^2}.
        \end{equation*}

    \item 
        Так как $\{X_n\}_n$ это последовательность \textbf{независимых} случайных величин, дисперсия суммы может быть раскрыта как сумма дисперсий:
        \begin{equation*}
            \D[X] = \dfrac{\D[X_1 + \dots + X_n]}{n^2} = \dfrac{\D[X_1] + \dots + \D[X_n]}{n^2} = \dfrac{\sigma_1^2 + \dots + \sigma_n^2}{n^2}.
        \end{equation*}
    \end{itemize}

    Воспользуемся неравенством Чебышева для случайной величины $X$ и подставим найденное математическое ожидания и дисперсию:
    \begin{equation*}
        \P[|X - \E[X]| \geq \eps] \leq \dfrac{\D[X]}{\eps^2}
        \iff \P\left[\left|\dfrac{X_1 + \dots + X_n}{n} - \dfrac{a_1 + \dots + a_n}{n}\right| \geq \eps\right] \leq \dfrac{\sigma_1^2 + \dots + \sigma_n^2}{n^2 \eps^2}
    \end{equation*}
\end{proof}

Закон больших чисел удобно применять, когда $X_n$ это независимые одинаково распределенные случайные величины (с конечным вторым моментом). В частности это означает, что у всех величин одно и то же математическое ожидание и одна и та же математическая дисперсия: $\E[X_n] = a$ и $\D[X_n] = \sigma^2$. 

Тогда дисперсия среднего арифметического $\dfrac{\D[X_1] + \dots + \D[X_n]}{n^2} = \dfrac{\sigma^2}{n}$ стремится к нулю и получаем
\begin{equation*}
    \P\left[\left|\dfrac{X_1 + \dots + X_n}{n} - a\right| \geq \eps\right] \to 0.
\end{equation*}

То есть в каком-то смысле среднее арифметическое приближается к математическому ожиданию.

\subsubsection{Усиленный закон больших чисел Колмогорова (б/д).}

\begin{theorem*}[Усиленный закон больших чисел Колмогорова]
    Пусть $\{X_n\}_n$ --- это последовательность независимых одинаково распределенных случайных величин, у которых есть математическое ожидание и пусть $\E[X_n] = a$.

    Тогда
    \begin{equation*}
        \P\left[\lim_{n \to \infty} \dfrac{X_1 + \dots + X_n}{n} = a\right] = 1.
    \end{equation*}
\end{theorem*}

Заметьте, что мы не требуем наличия второго момента, в отличие ЗБЧ в слабой форме. Также, эта сходимость более сильная, так как предел находится внутри условия вероятности, это будет объяснено позже.

\subsubsection{Сходимости случайных величин: почти наверное и по вероятности.}

\begin{definition*}
    Последовательность случайных величин $X_n$ сходится к случайной величине $X$ \textbf{по вероятности}, если для любого $\eps > 0$
    \begin{equation*}
        \lim_{n \to \infty} \P[|X_n - X| \geq \eps] = 0.
    \end{equation*}

    Записывают в следующем виде: $X_n \xrightarrow{P} X$.
\end{definition*}

\begin{definition*}
    Последовательность случайных величин $X_n$ сходится к случайной величине $X$ \textbf{почти наверное}, если
    \begin{equation*}
        \P[\lim_{n \to \infty} X_n = X] = 1.
    \end{equation*}

    Записывают в следующем виде: $X_n \xrightarrow{\text{п. н.}} X$.
\end{definition*}

То есть в законе больших чисел в слабой форме речь идет о сходимости по вероятности, а в усиленном законе больших чисел Колмогорова --- о сходимости почти наверное.

Из сходимости почти наверное следует сходимость по вероятности, поэтому усиленный закон больших чисел называется усиленным.

\subsubsection{Взаимосвязь сходимостей по вероятности и почти наверное.}

\begin{theorem*}
    Если последовательность случайных величин $X_n$ сходится к $X$ почти наверное, то $X_n$ сходится к $X$ и по вероятности.
\end{theorem*}

\begin{proof}
    Хотим доказать. что $\P[|X_n - X| > \eps] \to 0 $, что равносильно $\P[|X_n - X| < \eps] \to 1$, что мы и будем заказывать.

    Переформулируем выражение <<множество исходов, что для любого $\eps > 0$ существует $N$, что для любого $n > N$ выполняется $|X_n - X| < \eps$>> с помощью множеств:
    \begin{equation*}
        \bigcup_N \bigcap_{n = N + 1}^{\infty} \{w: |X_n - X| < \eps\}.
    \end{equation*}

    Но это множество включает в себя множество исходов, для которых $\lim X_n = X$:
    \begin{equation*}
        \bigcup_N \bigcap_{n = N + 1}^{\infty} \{w: |X_n - X| < \eps\}
        \supseteq \{w: \lim X_n = X\}.
    \end{equation*}

    Но по условию $\P[\lim X_n = X] = 1$, поэтому
    \begin{equation*}
        \P\left[ \bigcup_N \bigcap_{n = N + 1}^{\infty} \{w: |X_n - X| < \eps\}\right] = 1.
    \end{equation*}

    Обозначим $B_N = \bigcap_{n = N + 1}^{\infty} \{w: |X_n - X| < \eps\}$. Тогда 
    \begin{equation*}
        B_{N + 2} \supseteq B_{N + 1} \supseteq B_{N} \supseteq \dots \supseteq B_1,
    \end{equation*}
    так как чем больше номер множества, тем из меньшего числа пересечения оно состоит.

    Из второго модуля про вероятность вложенных событий мы знаем (теорема о непрерывности вероятностных мер), что
    \begin{equation*}
        \P\left[\bigcup_{N = 1}^{\infty} B_N\right] = \lim_{N \to \infty} \P[B_N].
    \end{equation*}

    Но мы уже доказали, что $\P\left[\bigcup_{N = 1}^{\infty} B_N\right] = 1$, тогда
    \begin{equation*}
        \P\left[\bigcap_{n = N + 1}^{\infty} \{w: |X_n - X| < \eps\}\right] \xrightarrow[N \to \infty]{} 1,
    \end{equation*}

    Заметим, что вероятность одного множества событий не меньше вероятности пересечения, поэтому о лемме о двух миллиционерах:
    \begin{equation*}
        \P[\{w: |X_n - X| < \eps\}] \to 1.
    \end{equation*}
\end{proof}