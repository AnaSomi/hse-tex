% Здесь НЕ НУЖНО делать begin document, включать какие-то пакеты..
% Все уже подрубается в головном файле
% Хедер обыкновенный хсе-теха, все его команды будут здесь работать
% Пожалуйста, проверяйте корректность теха перед пушем

% Здесь формулировка билета
\subsection{Сформулируйте и докажите критерий Дарбу интегрируемости ограниченной функции}

Разность точных граней ограниченной функции $f$ на множестве $D_i$ называется колебанием функции и обозначается:
\[\omega_i = M_i - m_i = \sup_{x,y\in D_i}{|f(x) - f(y)|} \geq 0\]
используя это обозначение сформулируем теорему\\
\textbf{\underline{Теор.:} } \textit{Критерий Дарбу интегрируемости} функции по Риману. \\
Пусть $f$ - ограниченная функция, тогда $f$ - интегрируема на жордановом множестве $D$ тогда и только тогда когда выполнено следующее
\[\forall \varepsilon > 0 \ \exists \delta > 0: \ \Delta(\tau) < \delta \Rightarrow \ S_D(f, \tau) - s_D(f, \tau) = \sum\limits_i\omega_i\mu(D_i) < \varepsilon\]
\textbf{\underline{Док-во:} } \\
\textit{Необходимость: }Пусть $f \in \mathcal{R}(D)$, тогда выполнено следующее
\[|I_D(f, \tau, p') - I_D(f, \tau, p'')| < \frac{\varepsilon}{3}, \ \ \text{при } \Delta(\tau) < \delta\]
(доказывается элементарно) \\
Выбором $p$ интегральная сумма ограниченной функции может быть сделана сколь угодно близкой к нижней (верхней) сумме Дарбу
\[I_D(f, \tau, p') - s_D(f, \tau) < \frac{\varepsilon}{3}, \ \ \ S_D(f, \tau) - I_D(f, \tau, p'') < \frac{\varepsilon}{3}\]
(также доказывается элементарно) \\
из этих 3 неравенств следует
\begin{multline*}
    \varepsilon > |S_D(f, \tau) - I_D(f, \tau, p'')| + |I_D(f, \tau, p'') - I_D(f, \tau, p')| + |I_D(f, \tau, p') - s_D(f, \tau)| \geq \\ \geq |S_D(f, \tau) - I_D(f, \tau, p'') + I_D(f, \tau, p'') - I_D(f, \tau, p')| + |I_D(f, \tau, p') - s_D(f, \tau)| = \\ = |S_D(f, \tau) - I_D(f, \tau, p')| + |I_D(f, \tau, p') - s_D(f, \tau)| \geq \\ \geq |S_D(f, \tau) - I_D(f, \tau, p') + I_D(f, \tau, p') - s_D(f, \tau)| = \\ = |S_D(f, \tau) - s_D(f, \tau)| < \varepsilon
\end{multline*}
\textit{Достаточность: } Пусть критерий Дарбу выполнен. Сперва докажем, что $\overline{s}_D(f) = \underline{S}_D(f)$. Пусть это не так, тогда $\overline{s}_D(f) < \underline{S}_D(f)$, в таком случае для какого либо $\tau$ 
\[s_D(f, \tau) \leq \overline{s}_D(f) < \underline{S}_D(f) \leq S_D(f, \tau)\] \\
В таком случае можно подобрать такой $\varepsilon$, что критерий выполнятся не будет $\Rightarrow$ противоречие. \\
Теперь пусть $I = \overline{s}_D(f) = \underline{S}_D(f) $ \\
Очевидно, что для любого разбиения $\tau$ и cистемы точек $p$ выполняется
\[s_D(f, \tau) \leq I, I(f, \tau, p) \leq S_D(f, \tau)\] 
Принимая во внимание данное неравенство, а также критерий Дарбу можно утверждать что 
\[|I_D(f, \tau, p) - I| \varepsilon, \ \ \ \text{причем } \Delta(\tau) < \delta\]
что как раз значит, что функция Интегриурема по Риману на $D$
\begin{flushright}
$\blacksquare$
\end{flushright}

