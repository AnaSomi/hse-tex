\newcommand{\darrowwtextover}[3]{%
\raisebox{0pt}[0pt][0pt]{%
$\overset{\mathclap{\substack{#1\\\left\downarrow\vcenter{\hrule height #2}\right.}}}{#3}%
$}%
}

\subsection{Тригонометрический ряд Фурье.
Теорема о сходимости тригонометрического ряда Фурье в средне-квадратичном (без доказательства полноты тригонометрической системы).
Представление частичной суммы ряда Фурье через ядро Дирихле.
Лемма Римана.
Условие Дини и теорема о поточечной сходимости ряда Фурье.
Разложение $\sin$ в бесконечное произведение.}

Лекции 3.8 (примерно с часа) и 3.9

\subsubsection{Тригонометрический ряд Фурье}

Тригонометрическая система: $\left\{ 1, \cos x, \sin x, \cos 2x, \dotsc \right\}$.
Ряд Фурье, который пользуется именно этой системой, и является тригонометрическим.

\subsubsection{Теорема о сходимости тригонометрического ряда Фурье в средне-квадратичном
(без доказательства полноты тригонометрической системы).}

Если тригонометрическая система $\left\{ 1, \cos x, \sin x, \cos 2x, \dotsc \right\}$ полна
в $\mathcal{R}_2 \left( [-\pi, \pi], \RR \right)$
(а это так, но без доказательства), то:
\begin{enumerate}
\item
$\forall f \in \mathcal{R}_2$ ряд Фурье сходится в $\mathcal{R}_2$ к $f$.
(Это не поточечная сходимость, а среднеквадратичная!)
\[
    f(x) \underset{\mathcal{R}_2}{=} \frac{a_0}{2} +
    \sum_{n=1}^{\infty} \left( a_n \cos nx + b_n \sin nx \right)
\]
\item
Выполняется равенство Парсеваля
\[
    \frac{1}{\pi} \int_{-\pi}^{\pi} f^2(x) dx = \frac{a_0^2}{2} +
    \sum_{n=1}^{\infty} \left( a_n^2 + b_n^2 \right)
\]
\end{enumerate}

Если функция не в $\mathcal{R}_2 \left( [-\pi, \pi], \RR \right)$, а в
$\mathcal{R}_2 \left( [-\pi, \pi], \CC \right)$, то
\[
    f(x) \underset{\mathcal{R}_2}{=} \sum_{n=-\infty}^{\infty} c_n e^{inx},
    \text{ где } c_n = \frac{1}{2\pi} \int_{-\pi}^{\pi} f(x) e^{-inx} dx
\]
, а равенство Парсеваля выглядит как
\[
    \frac{1}{2\pi} \int_{-\pi}^{\pi} |f(x)|^2 dx =
    \sum_{n=-\infty}^{\infty} |c_n|^2
\]

\subsubsection{Представление частичной суммы ряда Фурье через ядро Дирихле.}
$f: [-\pi, \pi] \rightarrow \RR$ — продолжаем периодически на всю вещественную прямую,
разрывы нам здесь не страшны
\begin{flalign*}
    & \text{Частичная сумма Фурье} \quad S_n(x) =
    \frac{a_0}{2} + \sum_{k=1}^n \left( a_k \cos kx + b_k \sin kx \right) =
    \sum_{k=-n}^n c_k e^{ikx} = \\
    & = \sum_{k=-n}^n \left( \frac{1}{2\pi} \int_{-\pi}^{\pi} f(t) e^{-ikt} dx \right) e^{ikx} =
    \text{\em суммирование конечное, поэтому заносим под интеграл} = \\
    & = \frac{1}{2\pi} \int_{-\pi}^{\pi} f(t) D_n(x - t) dt = \textcolor{blue}{(*)}\\
    & \left. \begin{array} {lr}
    D_n(u) = \sum_{k=-n}^n e^{ikn} = \frac{e^{i(n+1)u} - e^{-inu}}{e^{iu} - 1}
    \left( \frac{\phantom{w}\frac{1}{e^{in/2}}\phantom{w}}{\frac{1}{e^{in/2}}} \right) =
    \frac{\sin \left( n + \frac{1}{2} \right)u}{\sin \frac{1}{2} u}, & u \neq 2 \pi k \\
    D_n(u) = 2n + 1, & u = 2 \pi k
    \end{array} \right\}
    \parbox{10em}{\raggedright — Ядро Дирихле.\\
              Доопределили там, где потребовалось} \\
    & \text{Свойства:} \qquad
    D_n(u)\; 2 \pi \text{ периодическая}; \quad
    D_n(u) = _n(-u); \quad
    \frac{1}{2\pi} \int_{-\pi}^{\pi} D_n(u) du = 1 \\
    & \text{Заметим, что если } g(t) \text{ — } T \text{ периодическая, то }
    \int_0^T g(t) dt = \int_x^{T+x} g(t) dt \\
    & \textcolor{blue}{(*)} = \frac{1}{2\pi} \int_{-\pi}^{\pi} f(t) D_n(x - t) dt = \left\{ t' = x - t \right\}
    \frac{1}{2\pi} \int_{x-\pi}^{x + \pi} f(x - t') D_n(t') dt' =
    \frac{1}{2\pi} \int_{-\pi}^{\pi} f(x - t') D_n(t') dt' = \\
    & = \frac{1}{2\pi} \int_{-\pi}^{0} f(x - t') D_n(t') dt' +
    \frac{1}{2\pi} \int_{0}^{\pi} f(x - t') D_n(t') dt' = \left\{ \begin{array} {c}
        t^* = -t' \\
        \text{слева}
    \end{array}  \right\} = \\
    & = \frac{1}{2\pi} \int_{0}^{\pi} f(x + t^*) D_n(t^*) dt^* +
    \frac{1}{2\pi} \int_{0}^{\pi} f(x - t') D_n(t') dt' = \text{переобозначим } t =
    \frac{1}{2\pi} \int_0^{\pi} D_n(t) \left( f(x + t) + f(x - t) \right) dt
\end{flalign*}

\subsubsection{Лемма Римана.}
\begin{lemma*}
Если $f(\omega_1, \omega_2) \rightarrow \RR$ (бесконечности тоже можно) локально интегрируема
(интегрируема на любом подотрезке) и абсолютно интегрируема хотя бы в несобственном смысле на всем
$(\omega_1, \omega_2)$, то
\[
    \int_{\omega_1}^{\omega_2} f(x) e^{i \lambda x} dx \xrightarrow[\lambda \to \infty]{} 0
\]
\end{lemma*}
\begin{proof}
Так как $f$ абсолютно интегрируема, $\exists [a, b] \subset (\omega_1, \omega_2):
\left| \int_{\omega_1}^{\omega_2} f(x) e^{i \lambda x} dx - \int_a^b f(x) e^{i\lambda x}dx \right| \leq \eps$.
Утверждать так можем потому что $|f(x) e^{i\lambda x} | = |f(x)| \Rightarrow f(x)e^{i \lambda x}$ тоже абс. инт.

Рассмотрим нижнюю сумму Дарбу:
$0 \leq \int_a^b f(x) dx - \sum_{j=1}^n m_j \Delta x_j < \eps$. Здесь возникает разбиение
$a = x_0 < x_1 < x_2 < \dotsb < x_n = b$. Построим по нему кусочную функцию
\[
    h(x) = \left\{ \begin{array} {lr}
        m_1, & x \in [x_0, x_1) \\
        m_2, & x \in [x_1, x_2) \\
        \hdotsfor{2} \\
        m_n, & x \in [x_{n-1}, x_n]
    \end{array} \right.
\]
Заметим тогда, что
\begin{flalign*}
    & 0 \leq
    \left| \int_a^b f(x) e^{i \lambda x} dx - \int_a^b h(x) e^{i\lambda x} dx \right| \leq
    \int_a^b |f(x) - h(x)| \underbrace{|e^{i \lambda x}|}_{= 1} dx =
    \underbrace{\int_a^b |f(x) - h(x)| dx}_{\text{Дарбу!}} < \eps
\end{flalign*}

Разложим $h(x)$ по частям
\begin{flalign*}
    & \int_a^b h(x) e^{i \lambda x} dx = \sum_{j=1}^n \int_{x_{j-1}}^{x_j} m_j e^{i \lambda x} dx
    \darrowwtextover{\text{руками взять интеграл}}{1em}{=}
    \frac{1}{i\lambda} \sum_{j=1}^n m_j \left( e^{i\lambda x_j} - e^{i\lambda x_{j-1}} \right)
    \xrightarrow[\lambda \to \infty]{} 0
\end{flalign*}
Так как все множители под суммой ограничены, $n$ от лямбды не зависит, а мы еще делим на бесконечно большую лямбду

А теперь вспоминаем те интегралы, которые отличаются друг от друга на сколь угодно малую величину:
\begin{flalign*}
    & \int_a^b h(x) e^{i\lambda x} dx \to 0 \;\; \implies \;\; \int_a^b f(x) e^{i \lambda x} dx \to 0
    \;\; \implies \;\; \int_{\omega_1}^{\omega_2} f(x) e^{i \lambda x} dx \to 0
\end{flalign*}

\end{proof}

На самом деле функция может быть и комплекснозначной, для этого придется раскрыть $e^{i\lambda x}$
на синус и косинус и силой рук посчитать. Получается и вещественная часть $f(x) e^{i\lambda x}$
будет интегрироваться в ноль, и комплексная.
Доказательства этого не требуется, но стоит запомнить для следующего пункта.

\subsubsection{Условие Дини и теорема о поточечной сходимости ряда Фурье.}
\textbf{Запомните ПМИшники, }
$\sum_{n = -N}^N c_n e^{inx} = \frac{a_0}{2} + \sum_{n=1}^{N} \left( a_n \cos nx + b_n \sin nx \right)$

\begin{theorem*}
Пусть $f: \RR \to \CC$ $2 \pi$ периодическая (хотя можно и другой период, но мы упрощаем себе жизнь),
абсолютно интегрируемая на $[-\pi, \pi]$ и удовлетворяет в точке $x_0$ условию Дини:
\begin{enumerate}
\item $\exists f(x_0 \pm 0) $ — конечные
\item $\exists \eps > 0: \quad
    \int_0^{\eps} \left( \frac{f(x_0 - t) - f(x_0 - 0)}{t} + \frac{f(x_0 + t) - f(x_0 + 0)}{t} \right) dt$
    — сходится абсолютно
\end{enumerate}

Тогда разложение Фурье функции $f$ сходится в точке $x_0$:
$\sum_{n=-N}^{N} c_n e^{inx_0} \to \frac{f(x_0 - 0) + f(x_0 + 0)}{2} $

\end{theorem*}

\begin{proof}
Тут несложно, но много писать

\begin{flalign*}
    & S_N(x_0) - \frac{f(x_0 - 0) + f(x_0 + 0)}{2} = \\
    & = \frac{1}{2 \pi} \int_0^{\pi} \left( f(x_0 - t) + f(x_0 + t) \right)
    \frac{\sin \left( N + \frac{1}{2} \right) t}{\sin \frac{t}{2}} dt -
    \frac{f(x_0 - 0) + f(x_0 + 0)}{2} \cdot
    \darrowwtextover{\scalebox{0.75}{$\displaystyle
        \frac{2}{2\pi} \int_{0}^{\pi} \frac{\sin \left( N + \frac{1}{2} \right) t}{\sin \frac{t}{2}} dt
    $}}{1em}{
        \underbrace{\frac{1}{2\pi} \int_{-\pi}^{\pi} \frac{\sin \left( N + \frac{1}{2} \right) t}{\sin \frac{t}{2}} dt}_{=1}
    } = \\
    & = \frac{1}{2 \pi} \int_0^\pi
    \Big( \left( f(x_0 - t) + f(x_0 + t) \right) - \left( f(x_0 - 0) + f(x_0 + 0) \right) \Big)
    \frac{\sin \left( N + \frac{1}{2} \right) t}{\sin \frac{t}{2} } dt = \\
    & = \frac{1}{\pi} \int_0^\pi
    \underbracket{
        \left( \frac{f(x_0 - t) - f(x_0 - 0)}{t} + \frac{f(x_0 + t) - f(x_0 + 0)}{t} \right)
    }_{\text{абс. инт. по условию Дини}}
    \underbrace{\frac{t}{2 \sin\frac{t}{2} }}_{\mathclap{1 \text{ при } t \to 0}}
    \underbracket{\sin \left( N + \frac{1}{2} \right) t}_{\mathclap{\sim \Im e^{i \lambda t}}} dt
    \xrightarrow[N \to \infty]{} 0 \text{ по Риману}
\end{flalign*}
Здесь в последнем переходе как раз было махание руками про комплексные значения,
объясняется что ``если очень захотеть, то можно и доказать, но долго и неинтересно''
\end{proof}


\begin{definition*} 
Функция $f: [a, b] \to \CC$ называется кусочно-дифференцируемой, если 
\begin{enumerate} 
\item множество точек разрыва $M$ конечно, и каждая точка разрыва первого рода 
    (слева справа разные конечные значения)
\item Функция дифференцируема во всех точках $x \in [a, b] \setminus M$
\item $\forall x_0 \in M$ существуют конечные левая и правая производные.
\end{enumerate} 
\end{definition*} 

Кусочно-дифференцируемая функция удовлетворяет условию Дини, чем сейчас и воспользуемся


\subsubsection{Разложение $\sin$ в бесконечное произведение.}

Рассмотрим функцию $f(x) = \cos ax, [-\pi, \pi], |a| < 1$. Периодически продолжим на всю вещественную прямую, 
получится кусочно-дифференцируемая функция. А это значит, что с полученным рядом Фурье будет равенство.

Функция четная, поэтому $b_n = 0$.
\begin{flalign*}
    & a_n = \frac{1}{\pi} \int_{-\pi}^{\pi} \cos ax \cos nx dx = 
    \frac{(-1)^n \sin \pi a}{\pi} \frac{2a}{a^2 - n^2} \\
    & \cos ax = \frac{2a \sin \pi a}{\pi} \left( \frac{1}{2a^2} + 
    \sum_{n=1}^{\infty} \frac{(-1)^n}{a^2 - n^2} \cos nx  \right) \text{ верно } \forall x \in [-\pi, \pi] \\
    & \pmb{x = \pi\colon} \ctg \pi a - \frac{1}{\pi a} = \frac{2a}{\pi} \sum_{n=1}^{\infty} \frac{1}{a^2 - n^2}
    \text{ сходится равномерно при } a \in [-a_0, a_0]; a_0 < 1
    \text{ — проинтегрируем! (в $a = 0$ доопр.)} \\
    & \int_0^x \left( ctg \pi a - \frac{1}{\pi a} \right) da = 
    \frac{1}{\pi} \sum_{n=1}^{\infty} \int_0^x \frac{2a}{a^2 - n^2} da \quad \text{ интегрируем обе стороны} \\
    & \ln \frac{\sin \pi x}{\pi x} = \sum_{n=1}^{\infty} \ln(1 - \frac{x^2}{n^2}), \quad |x| < 1 \\
    & \frac{\sin \pi x}{\pi x} = \prod_{n=1}^{\infty} \left( 1 - \frac{x^2}{n^2} \right) \text{ ну вот и все}
\end{flalign*}
