\subsection{Свойства равномерно сходящегося несобственного интеграла. Теорема о предельном переходе под знаком несобственного интеграла. Монотонный предельный переход и теорема Дини и равномерной сходимости семейства функций. Следствие из теоремы Дини о монотонном предельном переходе под знаком несобственного интеграла. Теорема о непрерывности несобственного интеграла по параметру.}

\subsubsection{Свойства равномерно сходящегося несобственного интеграла.}
    \begin{properties}
    \begin{enumerate}
        \item Предельный переход под знаком интеграла.
        \begin{example}
            \[ \int\limits_{0}^{+\infty} f(x, n)\,dx \xrightarrow[n \to \infty]{} \int\limits_{0}^{+\infty} \varphi(x)\,dx \]
        \end{example}
        
        Покажем, что $f(x, n) \rightrightarrows \varphi(x)$ недостаточно:
        \[ f(x, n) = \left\{\begin{array}{ll} \frac{n}{x^3}\,e^{-\frac{n}{2x^2}}, & x > 0; \\ 0, & x = 0. \end{array}\right. 
        \ \ \ \forall n \in \NN \text{ непрерывна на } \; [0; +\infty) \]
        
        Проверяем равномерную сходимость $f(x, n) \rightrightarrows \varphi(x) = 0$:
        \[ \underset{x \ge 0}{\sup}\:\left| f(x, n) - \varphi(x) \right| = 
        \underset{x > 0}{\sup}\:\frac{n}{x^3}\,e^{-\frac{n}{2x^2}} \underset{\underset{x = \sqrt{\frac{n}{3}}}{\uparrow}}= 
        3 \sqrt{\frac3n}\,e^{-3/2} \xrightarrow[n \to \infty]{} 0 \;\Rightarrow\; \text{ сходимость равномерная} \]
        
        Проверяем значение интеграла:
        \[ \int\limits_{0}^{+\infty} \varphi(x)\,dx = 0 \]
        \[ \int\limits_{0}^{+\infty} f(n, x)\,dx = \int\limits_{0}^{+\infty} \frac{n}{x^3}\,e^{-\frac{n}{2x^2}}\,dx = 
        \int\limits_{-\infty}^{0} e^{z}\,dz = e^z \Bigm|_{-\infty}^0 = 1 \]
        \[ z = -\frac{n}{2x^2}, \; dz = \frac{n}{x^3}\,dx \]
        \[ \lim_{n \to \infty} \int\limits_{0}^{+\infty} f(x, n)\,dx \ne \int\limits_{0}^{+\infty} \lim_{n \to \infty} f(x, n)\,dx \]
        
        Требуется равномерная сходимость несобственного интеграла.
        
        \begin{theorem*} О предельном переходе под знаком несобственного интеграла.
        
            Рассмотрим $f(x, y)$, определенную на $[a; \omega) \times H$ ($\omega$ --- особая точка).
            
            Пусть \\
            \phantom{Пусть} $\forall y \in H \ \ f(x, y)$ несобственно интегрируема на $[a; \omega)$, \\
            \phantom{Пусть} причем $\int\limits_{a}^{\omega} f(x, y)\,dx$ сходится равномерно по $y \in H$, \\
            \phantom{Пусть} $\forall t \in [a; \omega) \ \ 
            f(x, y) \overset{x \in [a; t]}{\underset{y \to y_0}{\rightrightarrows}} \varphi(x), \; y_0 \in H$.
            
            Тогда $\varphi$ несобственно интегрируема на $[a; \omega)$, причем 
            \[ \lim_{y \to y_0} \int\limits_{a}^{\infty} f(x, y)\,dx = \int\limits_{a}^{\omega} \varphi(x)\,dx \]
        \end{theorem*}
        \begin{proof}
        \begin{enumerate}
            \item Покажем, что $\int\limits_{a}^{\omega} \varphi(x)\,dx$ сходится: \\[5 pt]
            $a \le t_1 < t_2 < \omega$ \\[3 pt]
            По критерию Коши для равномерно сходящегося несобственного интеграла $\int\limits_{a}^{\omega} f(x, y)\,dx$
            \[ \left| \int\limits_{t_1}^{t_2} f(x, y)\,dx \right| < \varepsilon \ \ \ \forall t_1, t_2 \in U(\omega), \; \forall y \in H \]
            \[ y \to y_0: \ \ \ \lim_{y \to y_0} \int\limits_{t_1}^{t_2} f(x, y)\,dx = 
            \int\limits_{t_1}^{t_2} \left( \lim_{y \to y_0} f(x, y) \right) dx = \int\limits_{t_1}^{t_2} \varphi(x)\,dx \ \ \ 
            \text{(т.к. $f(x, y) \overset{x \in [a; t]}{\underset{y \to y_0}{\rightrightarrows}} \varphi(x)$)} \]
            \[ \Rightarrow\; \left| \int\limits_{t_1}^{t_2} \varphi(x)\,dx \right| \le \varepsilon \ \ \ \forall t_1, t_2 \in U(\omega) \]
            \[ \Rightarrow\; \varphi \text{ несобственно интегрируема} \]
            
            \item Покажем, что $\left| \int\limits_{a}^{\omega} f(x, y)\,dx - \int\limits_{a}^{\omega} \varphi(x)\,dx \right|
            \xrightarrow[y \to y_0]{} 0$:
            \[ \left| \int\limits_{a}^{\omega} f(x, y)\,dx - \int\limits_{a}^{\omega} \varphi(x)\,dx \right| \le 
            \underset{< \text{\large$\frac{\varepsilon}{3}$} \ \begin{array}{l} \forall t \in U(\omega) \\[-7 pt] 
            \forall y \in H \end{array}}{\underbrace{\left| \int\limits_{a}^{\omega} f(x, y)\,dx - \int\limits_{a}^{t} f(x, y)\,dx \right|}} + \]
            \[ + \underset{< \text{\large$\frac{\varepsilon}{3}$} \ \begin{array}{c} 
            \text{\footnotesize При фикс. } t \in U(\omega) \\[-7 pt] |y - y_0| < \delta \end{array}}
            {\underbrace{\left| \int\limits_{a}^{t} f(x, y)\,dx - \int\limits_{a}^{t} \varphi(x)\,dx \right|}} + 
            \underset{< \text{\large$\frac{\varepsilon}{3}$} \ \displaystyle\forall t \in U(\omega)}
            {\underbrace{\left| \int\limits_{a}^{t} \varphi(x)\,dx - \int\limits_{a}^{\omega} \varphi(x)\,dx \right|}} < \varepsilon \ \ 
            \forall \varepsilon > 0, \ \ \text{что и требовалось} \]
        \end{enumerate}
        \end{proof}
        
        \item Монотонный предельный переход
        
        \begin{theorem*} Теорема Дини
        
            Пусть \\
            \phantom{Пусть} $f(x, y) \ge 0$ и непрерывна $\forall y \in H$ по $x \in [a; \omega)$, \\
            \phantom{Пусть} при $\forall x \in [a; \omega) \ \ f(x, y) \uparrow$ по $y$ и $f(x, y) \xrightarrow[y \to y_0]{} \varphi(x)$, \\
            \phantom{Пусть} $\varphi$ --- непрерывна на $[a; \omega)$.
            \[ \text{Тогда} \ \ f(x, y) \overset{[a; t]}{\underset{y \to y_0}{\rightrightarrows}} \varphi(x) \ \ \forall t \in (a; \omega) \]
        \end{theorem*}
        \begin{proof}
            В силу монотонности по $y$:
            \[ (\varphi(x) - f(x, y)) \downarrow \text{ по } y, \ \varphi(x) - f(x, y) \ge 0 \]
            \[ \forall y_1 < y_2 \ \ \ \varphi(x) - f(x, y_1) \ge \varphi(x) - f(x, y_2) \;\Rightarrow \]
            \[ \Rightarrow\; \underset{x}{\sup}\:\left| \varphi(x) - f(x, y_1) \right| \ge 
            \underset{x}{\sup}\:\left| \varphi(x) - f(x, y_2) \right| \;\Rightarrow \]
            \[ \Rightarrow\; \psi(y) = \underset{x}{\sup}\:\left| \varphi(x) - f(x, y) \right| \ \text{--- убывает и } \ge 0 \]
            
            Докажем, что $\psi(y) \xrightarrow[y \to y_0]{} 0$. От противного: пусть $\psi(y) \ge \varepsilon > 0$.
    
            Тогда $\exists \{ (x_n, y_n) \} : \: y_n \nearrow y_0, \ \varphi(x_n) - f(x_n, y_n) \ge \varepsilon / 2$
            
            $\{ x_n \} \subset [a; t] \;\Rightarrow\;$ из нее можно выделить сходящуюся подпоследовательность $\{ x_{n_k} \}$ такую, что $x_{n_k} \xrightarrow[k \to \infty]{} c \in [a; t]$.
            \[ \varphi(x_{n_k}) - f(x_{n_k}, y) \ge \varepsilon / 2 \]
            \[ k \to \infty: \ \varphi(c) - f(c, y) \ge \varepsilon / 2 \ \ \ \forall y \in U(y_0) \]
            --- противоречит тому, что $f(x, y) \to \varphi(x)$.
        \end{proof}
        \begin{corollary}
            Пусть $f(x, y) \ge 0$ и непрерывна по $x \in [a; \omega)$ $\forall y \in H$, \\
            \phantom{Пусть} $\forall x \in [a; \omega) \ \ f(x, y) \uparrow$ по $y$ и $f(x, y) \underset{y \to y_0}{\nearrow} 
            \varphi(x)$, \\
            \phantom{Пусть} $\int\limits_a^{\omega} \varphi(x)\,dx$ --- сходится.
            \[ \text{Тогда} \ \ \lim_{y \to y_0} \int\limits_a^{\omega} f(x, y)\,dx = \int\limits_a^{\omega} \varphi(x)\,dx \]
        \end{corollary}
        \begin{proof}
            По теореме Дини: $f(x, y) \overset{[a; t]}{\underset{y \to y_0}{\rightrightarrows}} \varphi(x) \varphi(x) \ \ \ 
            \forall t \in (a; \omega)$
            
            $0 \le f(x, y) \le \varphi(x)$
            
            Т.к. $\int\limits_a^{\omega} \varphi(x)\,dx$ сходится, то $\int\limits_a^{\omega} f(x, y)\,dx$ сходится равномерно (по признаку Вейерштрасса) 
        \end{proof}
        
        \begin{example}
            \[ \int\limits_0^{+\infty} e^{-x^2}\,dx = ? \]
            \[ \left( 1 + \frac{x^2}{n} \right)^n \underset{n \to \infty}{\nearrow} e^{x^2} \ \ \forall x \ge 0 \]
            Т.к. функции непрерывны, сходимость равномерная (по теореме Дини) \\
            $\Rightarrow$ равномерно сходится $f(x, n) = \left( 1 + \frac{x^2}{n} \right)^{-n} 
            \underset{n \to \infty}{\rightrightarrows} e^{-x^2} = \varphi(x)$
            
            Но последовательность $f(x, n)$ убывает --- рассмотрим разность
            \[ g(x, n) = f(x, 1) - f(x, n) = \left( 1 + x^2 \right)^{-1} - \left( 1 + \frac{x^2}{n} \right)^{n} \ge 0 \ \ 
            \uparrow \text{ по } n \]
            \[ g(x, n) \rightrightarrows \psi(x) = (1  +x^2)^{-1} - e^{-x^2} \ge 0 \ \ \ \int\limits_0^{+\infty} \psi(x)\,dx \ \ 
            \text{ сходится} \]
            
            По следствию из теоремы Дини:
            \[ \Rightarrow\; \int\limits_0^{+\infty} g(x, n)\,dx \to \int\limits_0^{+\infty} \psi(x)\,dx \]
            \[ \int\limits_0^{+\infty} \left( \frac1{1 + x^2} - \frac1{\left( 1 + \frac{x^2}n \right)^n} \right) dx \to
            \int\limits_0^{+\infty} \left( \frac1{1 + x^2} - e^{-x^2} \right) dx \]
            \[ \int\limits_0^{+\infty} e^{-x^2}\,dx = 
            \lim_{n \to \infty} \int\limits_0^{+\infty} \frac{dx}{\left( 1 + \frac{x^2}n \right)^n} \]
            \[ \int\limits_0^{+\infty} \frac{dx}{\left( 1 + \frac{x^2}n \right)^n} = 
            \frac{(2n - 3)!!}{(2n - 2)!!} \cdot \sqrt{n} \cdot \frac{\pi}2 \]
            Формула Валлиса: $\ \prod\limits_{k = 1}^n \frac{4k^2}{4k^2 - 1} \to \frac{\pi}2$
        \end{example}
        
        \item Непрерывность интеграла
        \[ f(x, y) \ \ \ [a; \omega) \times [c; d] \]
        Пусть $\ f(x, y)$ непрерывна на $[a; \omega) \times [c; d]$ \\
        \phantom{Пусть} $F(y) = \int\limits_a^{\omega} f(x, y)\,dx$ сходится равномерно на $[c; d]$
        
        Тогда $F(y)$ непрерывна на $[c; d]$
        \begin{proof}
            \[ g(t, y) = \int_a^t f(x, y)\,dx \text{ --- непрерывна по } y \in [c; d] \ \ \forall t \in (a; \omega) \]
            \[ \int_a^t f(x, y)\,dx \text{ --- сходится равномерно} \;\Leftrightarrow\; g(t, y) 
            \overset{y \in [c; d]}{\underset{t \to \omega}{\rightrightarrows}} F(y) \]
            Равномерно сходящееся семейство непрерывных функций сходится к непрерывной функции:
            \[ |F(y) - F(y_0)| \le \underset{< \text{\large$\frac{\varepsilon}{3}$} \ 
            \begin{array}{l} \forall t \in U(\omega) \\[-7 pt] 
            \forall y \in H \end{array}}{\underbrace{\left| F(y) - g(t, y) \right|}} + 
            \underset{< \text{\large$\frac{\varepsilon}{3}$} \ \begin{array}{c} 
            \text{\footnotesize При фикс. } t \in U(\omega) \\[-7 pt] |y - y_0| < \delta \end{array}}
            {\underbrace{\left| g(t, y) - g(t, y_0) \right|}} + 
            \underset{< \text{\large$\frac{\varepsilon}{3}$} \ \begin{array}{l} \forall t \in U(\omega) \\[-7 pt] 
            \forall y \in H \end{array}}
            {\underbrace{\left| g(t, y_0) - F(y_0) \right|}} < \varepsilon \ \ 
            \forall \varepsilon > 0 \]
        \end{proof}
        
        \item Дифференцирование по параметру
        \[ F(y) = \int_a^{\omega} f(x, y)\,dx, \ \ \ y \in [c; d] \]
        Пусть: $\ f(x, y), \; f_y'(x, y)$ --- непрерывны на $[a; \omega) \times [c; d]$ \\
        \phantom{Пусть} $\Phi(y) = \int_a^{\omega} f_y'(x, y)\,dx$ --- сходится равномерно на $[c; d]$ \\
        \phantom{Пусть} $\int_a^{\omega} f(x, y)\,dx$ сходится хотя бы в 1 точке $y_0 \in [c; d]$
        
        Тогда:
        \[ \int_a^{\omega} f(x, y)\,dx \text{ сходится равномерно на } [c; d], \]
        причем $F(y)$ дифференцируема на $[c; d]$ и $F'(y) = \Phi(y)$
        \begin{proof}
            \[ g(t, y) = \int_a^t f(x, y)\,dx \]
            По теореме о дифференцировании собственного интеграла по параметру, $g(t, y)$ дифференцируема по $y$ на $[c; d]$ и 
            \[ g_y'(t, y) = \int_a^t f_y'(x, y)\,dx \overset{y \in [c; d]}{\underset{t \to \omega}{\rightrightarrows}} \Phi(y) \]
            По условию $\ g(t, y_0) \xrightarrow[t \to \omega]{} F(y_0)$
            
            Рассмотрим семейство $g(t, y)$. По теореме о дифференцировании семейства функций по параметру:
            \[ g(t, y) \overset{y \in [c; d]}{\underset{t \to \omega}{\rightrightarrows}} F(y), \ \ F'(y) = \Phi(y) \]
        \end{proof}
        
        \begin{example}
            Вычислим интеграл Дирихле: $\int\limits_0^{+\infty} \frac{\sin x}x\,dx$
            
            Рассмотрим вспомогательный интеграл: $F(y) = \int\limits_0^{+\infty} \frac{\sin x}x \cdot e^{-xy}\,dx, \ y > 0$
            \[ f(x, y) = \frac{\sin x}x \cdot e^{-xy}, \ \ f_y'(x, y) = -\sin x \cdot e^{-xy} 
            \text{ --- непрерывна на } [0; +\infty) \times [c; d], \]
            где $[c; d] \subset (0; +\infty)$
            \[ \Phi(y) = -\int\limits_0^{+\infty} \sin x \cdot e^{-xy}\,dx \text{ --- вспомогательный интеграл} \]
            \[ \left| \sin x \cdot e^{-xy} \right| \le e^{-xy}, \ xy \ge cx, \ \int\limits_0^{+\infty} e^{-cx}\,dx \text{ --- сходится} \]
            \[ \Rightarrow\; \int\limits_0^{+\infty} \sin x \cdot e^{-xy}\,dx \text{ --- равномерно сходится по признаку Вейерштрасса} \]
            При $y_0 > 0 \ \ \int\limits_0^{+\infty} \frac{\sin x}x \cdot e^{-xy}\,dx$ сходится по признаку Абеля \\
            (т.к. $\int\limits_0^{+\infty} \frac{\sin x}x$ --- сходится, а $e^{-xy}$ --- монотонная и ограниченная)
            
            $\Rightarrow$ по теореме можно внести $\frac{d}{dy}$ под знак интеграла:
            \[ F'(y) = -\int\limits_0^{+\infty} \sin x \cdot e^{-xy}\,dx = \star \]
            
            \[ \int \sin x \cdot e^{-xy}\,dx = -\int e^{-xy}\,d(\cos x) = -e^{-xy} \cos x - y \int \cos x \cdot e^{-xy}\,dx = \]
            \[ = -e^{-xy} \cos x - y \int e^{-xy}\,d(\sin x) = -e^{-xy} \cos x - y e^{-xy} \sin x - y^2 \int \sin x \cdot e^{-xy}\,dx \]
            \[ \Rightarrow;\ \int \sin x \cdot e^{-xy}\,dx = -\frac{e^{-xy} (\cos x + y \sin x)}{1 + y^2} + C \ \ (y > 0) \]
            \[ \star = -\left( 0 + \frac1{1 + y^2} \right) \]
            Т.е. $F'(y) = -\frac1{1 + y^2} \;\Rightarrow\; F(y) = -\arctg y + C$
            
            Вычислим $\lim_{y \to +\infty} F(y) = \lim_{y \to +\infty} \int\limits_0^{+\infty} \frac{\sin x}x \cdot e^{-xy}\,dx$
            \[ \left\{\begin{array}{l} 
            \int\limits_0^{+\infty} \frac{\sin x}x \cdot e^{-xy}\,dx \text{ --- сходится равномерно}, \\
            \lim_{y \to +\infty} \frac{\sin x}x \cdot e^{-xy} = 0 \ \ \text{ при } x \in [a; b] \subset (0; +\infty), \\
            \text{причем } \ \frac{\sin x}x \cdot e^{-xy} \overset{x \in [a; b]}{\underset{y \to +\infty}{\rightrightarrows}} 0
            \end{array}\right. \]
            \[ \Rightarrow\; \text{можно внести } \lim_{y \to +\infty} \text{ под знак интеграла} \]
            
            \[ \lim_{y \to +\infty} F(y) = \int\limits_0^{+\infty} 0\,dx = 0 \;\Rightarrow\; C = \frac{\pi}2 \]
            Итак: $F(y) = -\arctg y + \frac{\pi}2$
            Осталось внести $\lim_{y \to +\infty}$ под интеграл $F(y)$ и получить интеграл Дирихле
            \[ \left\{\begin{array}{l} 
            \int\limits_0^{+\infty} \frac{\sin x}x\,e^{-xy}\,dx \text{ --- сходится равномерно по признаку Абеля}, \\
            \lim_{y \to +0} \frac{\sin x}x\,e^{-xy} = \frac{\sin x}x, \\
            \left| \frac{\sin x}x\,e^{-xy} - \frac{\sin x}x \right| \le \frac{|\sin x|}x \cdot \left( 1 - e^{-xy} \right) \le 
            \frac1a \cdot \left( 1 - e^{-by} \right) \\
            \Rightarrow\; \frac{\sin x}x\,e^{-xy} \rightrightarrows \frac{\sin x}x
            \end{array}\right. \ \ \ x \in [a; b] \subset (0; +\infty) \]
            \[ \Rightarrow\; \lim_{y \to +0} F(y) = \int\limits_0^{+\infty} \frac{\sin x}x\,dx \]
            \[ \Rightarrow\; \int\limits_0^{+\infty} \frac{\sin x}x\,dx = 
            \lim_{y \to +0} \left( -\arctg y + \frac{\pi}2 \right) = \frac{\pi}2 \]
        \end{example}
        
        \item Собственный интеграл по параметру
        \[ \int\limits_c^d F(y)\,dy = \int\limits_c^d dy \int\limits_a^{\omega} f(x, y)\,dx \]
        Пусть \\
        \phantom{Пусть} $f(x, y)$ непрерывна на $[a, \omega) \times [c, d]$ \\
        \phantom{Пусть} $F(y) = \int\limits_a^{\omega} f(x, y)\,dx$ --- сходится равномерно на $[c, d]$
        
        Тогда $F$ --- непрерывна на $[c, d]$ (следовательно, интегрируема) и 
        \[ \int\limits_c^d dy \int\limits_a^{\omega} f(x, y)\,dx = \int\limits_a^{\omega} dx \int\limits_c^d f(x, y)\,dy \]
        \begin{proof}
            Пусть $a < t < \omega$. Для собственного интеграла 
            \[ g(t, y) = \int\limits_a^t f(x, y)\,dx \]
            возможность внесения $\int\limits_c^d dy$ следует из непрерывности $f$:
            \[ \int\limits_c^d g(t, y)\,dy = \int\limits_c^d dy \int\limits_a^t f(x, y)\,dx = 
            \underset{\text{непр. ф-ция}}{\underbrace{\int\limits_a^t dx \int\limits_c^d f(x, y)\,dy}} \]
            \phantom{$\int\limits_c^d g(t, y)\,dy =\ \ \ \ \ $} $\overset{\downarrow}{\int\limits_c^d F(y)\,dy} \ \ \ \ \ = \ \ \ \ \ 
            \overset{\downarrow}{\int\limits_a^{\omega} dx \int\limits_c^d f(x, y)\,dy}$
            \[ \lim_{t \to \omega}\int\limits_c^d  dy \left( \int\limits_a^t f(x, y)\,dx \right) = 
            \int\limits_c^d \left( \lim_{t \to \omega} \int\limits_a^t f(x, y)\,dx \right) dy = \int\limits_c^d F(y)\,dy \]
        \end{proof}
        
        \item Несобственный интеграл по параметру
        \[ \int\limits_c^{\tilde \omega} F(y)\,dy = \int\limits_c^{\tilde \omega} dy \int\limits_ac^{\omega} f(x, y)\,dx \]
        Пусть \\
        \phantom{Пусть} $f(x, y)$ непрерывна на $[a, \omega) \times [c, \tilde \omega]$ \\
        \phantom{Пусть} $F(y) = \int\limits_a^{\omega} f(x, y)\,dx$ --- сходится равномерно на $[c, \tau]$, 
        где $c < \tau < \omega$, \\
        \phantom{Пусть} хотя бы 1 из интегралов:
        \[ \int\limits_c^{\tilde \omega} dy \int\limits_a^{\omega} |f(x, y)|\,dx, \ \ 
        \int\limits_a^{\omega} dx \int\limits_c^{\tilde \omega} |f(x, y)|\,dy \]
        \phantom{Пусть} сходится.
        
        Тогда $\int\limits_c^{\tilde \omega} dy \int\limits_a^{\omega} f(x, y)\,dx = 
        \int\limits_a^{\omega} dx \int\limits_c^{\tilde \omega} f(x, y)\,dy$
    \end{enumerate}
        \begin{proof}
            \[ \forall \tau \in (c; \tilde \omega) \ \ \ \int\limits_c^{\tau} dy \int\limits_a^{\omega} f(x, y)\,dx = 
            \int\limits_a^{\omega} dx \int\limits_c^{\tau} f(x, y)\,dy \]
            Рассмотрим предельный переход $\tau \to \tilde \omega$:
            \[ \lim_{\tau \to \tilde \omega} \int\limits_a^{\omega} dx \int\limits_c^{\tau} f(x, y)\,dy \]
            \[ \varphi(\tau, x) = \int\limits_c^{\tau} f(x, y)\,dy \overset{x \in [a; t]}{\underset{\tau \to \tilde \omega}
            {\rightrightarrows}} \Phi(x) = \int\limits_c^{\tilde \omega} f(x, y)\,dy \]
            --- по условию
            \[ |\varphi(\tau, x)| \le \int\limits_c^{\tilde \omega} |f(x, y)|\,dy, \ \ \int\limits_a^{\omega} |\varphi(\tau, x)|\,dx
            \le \int\limits_a^{\omega} dx \int\limits_c^{\tilde \omega} |f(x, y)|\,dy \]
            --- сходятся по условию \\
            $\Rightarrow$ по признаку Вейерштрасса $\int\limits_a^{\omega} \varphi(\tau, x)\,dx$ сходится равномерно
        \end{proof}
    \end{properties}
