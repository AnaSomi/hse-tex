\subsection{Свойства равномерно сходящегося семейства функций. Теорема о предельном переходе. Теорема о непрерывности по параметру. Теорема об интегрировании по параметру. Теорема о дифференцировании по параметру.}

\subsubsection{Теорема о пределльном переходе}

\begin{theorem*}[Теорем о предельном переходе] \ \\
    Пусть: 
    \begin{itemize}
        \item $f(x, y) \overset{y \in H}{\underset{x \to a}{\rightrightarrows}} g(y)$
        \item $\forall x \in D: f(x, y) \underset{y \to b}{\to} h(x)$
        \item $h(x) \underset{x \to a}{\to} c$
    \end{itemize}  
    Тогда $\displaystyle \lim_{y \to b} g(y) = \lim_{x \to a} h(x) = c$
\end{theorem*}
\begin{proof}[Доказательство теоремы о предельном переходе]
    Необходимо доказать, что $|g(y) - c|$ мала. \\
    $|g(y) - c| \leqslant |g(y) - f(x, y)| + |f(x + y) - h(x)| + |h(x) - c|$ \\
    \begin{itemize}
        \item $|g(y) - f(x, y)| < \dfrac{\epsilon}{3}$, при $ 0 < |x - a| < \delta_1, \ \forall y \in H$
        \item $|h(x) - c| < \dfrac{\epsilon}{3}$, при $0 < |x - a| < \delta_2$
        \item $|f(x, y) - h(x)| < \dfrac{\epsilon}{3}$ при фиксированном
         $x$ и  $0 < |y - b| < \delta_3$
    \end{itemize}  
    Для $\delta = \delta_3: |g(y) - c| < \dfrac{\epsilon}{3} + 
    \dfrac{\epsilon}{3} +  \dfrac{\epsilon}{3} = \epsilon$
\end{proof}
\subsubsection{Теорема о непрерывности по параметру}
\begin{theorem*}[Теорема о непрерывности по параметру]
    Пусть
    \begin{itemize}
        \item $f(x, y) \overset{y \in H}{\underset{x \to a}{\rightrightarrows}} g(y)$
        \item $f(x, y)$, $\forall x \in D$ -- непрерывна как функция от $y$ в точке $y = b$
    \end{itemize}
    Тогда $g(y)$ -- непрерывна в точке $y = b$
\end{theorem*}
\begin{proof}[Доказательство теоремы о непрерывности по парметру]
    Необходимо доказать, что предел разности $|g(y) - g(b)|$ равен нулю. \\ 
    $|g(y) - g(b)| \leqslant |g(y) - f(x, y)| + |f(x, y) - f(x, b)| + |f(x, b) - g(b)|$
    \begin{itemize}
        \item $|g(y) - f(x, y)| < \dfrac{\epsilon}{3}$, при $0 < |x - a| < \delta_1, \forall y \in H$ 
        (в силу условий равномерной сходимости)
        \item $|f(x, b) - g(b)| < \dfrac{\epsilon}{3}$ -- это частный случай предыдущего 
        пункта (так как $b \in H$ по условию теоремы)
        \item $|f(x, y) - f(x, b)| < \dfrac{\epsilon}{3}$ при фиксированном $x$ и, в виду непрерывности
        $f(x, y)$ по $y$ в точки $y = b$ (условие теоремы), $|y - b| < \delta_2$
    \end{itemize}
    Для $\delta = \delta_2: |g(y) - g(b)| < \dfrac{\epsilon}{3} + 
    \dfrac{\epsilon}{3} +  \dfrac{\epsilon}{3} = \epsilon$
\end{proof}

\subsubsection{Теорема об интегрировании по параметру}
\begin{theorem*}[Теорема об интегрировании по параметру]
    Пусть
    \begin{itemize}
        \item Пусть $H$ -- жорданово множество
        \item $f(x, y)$ -- ограничена на $D \times H$
        \item $\forall x \in D: f(x, y)$ -- интегрируема по $y$
        \item $f(x, y) \overset{y \in H}{\underset{x \to a}{\rightrightarrows}} g(y)$
    \end{itemize}
    Тогда функция $g(y)$ интегрируема и $\displaystyle \int_{H} g(y)dy = 
    \lim_{x \to a} \int_{H}f(x,y)dy$
\end{theorem*}
\begin{proof}[Доказательство теоремы об интегрировании по параметру]
    Сначала докажем, что функция $g(y)$ -- интегрируема. \\
    Случай, когда $\mu(H) = 0$ -- тривиален: любая функция интегрируема на этом множестве 
    и интеграл равен нулю. Поэтому далее рассматриваем случай $\mu(H) \neq 0$. \\  
    Для этого воспользуемся критерием Дарбу.\\
    Пусть $\{H_i\}$ -- разбиение множества $H$. Тогда необходимо доказать: \\
    $\displaystyle \sum_{i} \sup_{y_1, y_2 \in H_i} |g(y_1) - g(y_2)| \mu(H_i) =
    \sum_{i} \omega_{g}(H_i) \mu(H_i) < \eps$, где $\mu(H_i)$ -- мера множества $H_i$. \\ 
    $|g(y_1) - g(y_2) \leqslant |g(y_1) - f(x, y_1)| + |f(x, y_1) - f(x, y_2)| 
    + |f(x, y_2) - g(y_2)|$
    \begin{itemize}
        \item $|g(y_1) - f(x, y_1)| < \dfrac{\epsilon}{3 \mu(H)}$,
        при $ 0 < |x - a| < \delta_1, \ \forall y_1 \in H$
        \item $|f(x, y_2) - g(y_2)| < \dfrac{\epsilon}{3 \mu(H)}$,
        при $ 0 < |x - a| < \delta_2, \ \forall y_2 \in H$
    \end{itemize} 
    Теперь перепишем сумму с учётом оценок выше для $\delta = \min(\delta_1, \delta_2)$: \\
     $\displaystyle \sum_{i} \sup_{y_1, y_2 \in H_i} |g(y_1) - g(y_2)| \mu(H_i) \leqslant
    \sum_i \dfrac{\epsilon}{3 \mu(H)} \mu(H_i) + 
    \sum_i \sup_{y_1, y_2 \in H} |f(x, y_1) - f(x, y_2)| \mu(H_i) +
    \sum_i \dfrac{\epsilon}{3 \mu(H)} \mu(H_i)$ \\
    \begin{itemize}
        \item $\displaystyle \sum_i \sup_{y_1, y_2 \in H} |f(x, y_1) - f(x, y_2)| \mu(H_i) 
        < \dfrac{\epsilon}{3}$ по критерию Дарбу, так как $f(x, y)$ интегрируема по $y$ при
        фиксированном $x$ (условие теоремы)
    \end{itemize} 
    Таким образом, $\displaystyle \sum_{i} \sup_{y_1, y_2 \in H_i} |g(y_1) - g(y_2)| \mu(H_i)
    \leqslant \dfrac{\epsilon}{3} + \dfrac{\epsilon}{3} + \dfrac{\epsilon}{3} = \epsilon
    \Rightarrow g$ -- интегрируема по критерию Дарбу \\
    
    Теперь докажем вторую часть утверждения -- научимся брать интеграл от функции $g$.\\
    $\displaystyle \left|\int_{H} (g(y) - f(x, y))dy\right| \leqslant
    \int_{H} |(g(y) - f(x, y)|dy$ 
    \begin{itemize}
        \item $ |(g(y) - f(x, y)| < \epsilon$ при $ 0 < |x - a| < \delta$ и $\forall y \in H$
    \end{itemize}
    Следовательно, $\displaystyle \left|\int_{H} (g(y) - f(x, y))dy\right| < \epsilon \mu(H)$ \\
    $\displaystyle \Rightarrow \int_{H} ((g(y) - f(x, y))dy \underset{x \to a}{\to 0}$ \\
    $\displaystyle \int_{H} f(x,y)dy = \int_{H} g(y)dy + \int_{H} ((g(y) - f(x, y))dy 
    \to \int_{H} g(y)dy$ 
\end{proof}

\subsubsection{Теорема о дифференцировании по параметру}
\begin{theorem*}[Теорема о дифференцировании по параметру]
    Пусть
    \begin{itemize}
        \item $H$ -- выпуклое ограниченное множество (например: отрезок $[c, d]$)
        \item $\forall x \in D: f(x, y)$ -- дифференцируема по $y \in H$
        \item $f(x, y) \overset{y \in H}{\underset{x \to a}{\to}} g(y), a \in \overline{D}$
        \item $f'_y(x, y) \overset{y \in H}{\underset{x \to a}{\rightrightarrows}} h(y)$
    \end{itemize}
    Тогда $g(y)$ -- дифференцируема на множестве $H$ и $g'(y) = h(y)$
\end{theorem*}
\begin{proof}[Доказательство теоремы о дифференцировании по параметру]
   Сначала докажем, что $f(x, y) \overset{y \in H}{\underset{x \to a}{\rightrightarrows}} g(y)$. Для
   этого воспользуемся критерием Коши: хотим доказать, что $|f(x_1, y) - f(x_2, y)| < \epsilon$
   равномерно по всем $y$, если только $x_1$ и $x_2$ достаточно близко к точке $a$ лежат. Тогда будет
   выполнено условие Коши, а значит, что семейство $f(x, y)$ равномерно сходится к своей предельной
   функции $g$. \\
   Возьмём какое-нибудь $y_0 \in H$, тогда: \\
   $|f(x_1, y) - f(x_2, y)| \leqslant |(f(x_1, y) - f(x_2, y)) - (f(x_1, y_0) - f(x_2, y_0))| +
   |(f(x_1, y_0) - f(x_2, y_0))|$ \\
   Теперь зафиксируем $x_1$ и $x_2$, тогда можем рассматривать функцию $q(y) = (f(x_1, y) - f(x_2, y))$.
   Так как мы из условия теоремы знаем, что $f(x,y)$ дифференцируема по $y \in H$, то и 
   функции $q(y)$ дифференцируема по $y \in H$. Теперь необходимо применить теорему Лагранжа
   для функции $q(y)$. Модифицируем равенство дальше: \\
   $|f(x_1, y) - f(x_2, y)| \leqslant |q(y) - q(y_0)| +
   |q(y_0)| =$ [Теорема Лагранжа] $= |q'(y*)| \cdot |y - y_0| + |q(y_0)|$, где
   $y* \in [\min(y_0, y), \max(y_0, y)]$ \\
   Вернёмся к записи через функцию $f$: \\
   $|f(x_1, y) - f(x_2, y)|  \leqslant |f'_y(x_1, y*) - f'_y(x_2, y*)|\cdot |y - y_0| +|(f(x_1, y_0) - f(x_2, y_0))| $ \\
   \begin{itemize}
       \item По условию теоремы $f'_y(x, y) \overset{y \in H}{\underset{x \to a}{\rightrightarrows}} h(y) \Rightarrow$,
   применив критерий Коши для производной можем сказать, что $|f'_y(x_1, y*) - f'_y(x_2, y*)|
   < \dfrac{\epsilon}{2 \cdot diam(H)}$
        \item $|y - y_0| < diam(H)$
        \item $|(f(x_1, y_0) - f(x_2, y_0))| < \dfrac{\epsilon}{2}$, так как
        $f(x, y_0) \to g(y_0)$  
    \end{itemize}
    Итого: $|f(x_1, y) - f(x_2, y)| < \epsilon$

    Теперь хотим доказать $\dfrac{f(x, y) - f(x, b)}{y - b} \overset{y \in H}{\underset{x \to a}{\rightrightarrows}}
    \dfrac{g(y) - g(b)}{y - b}$, $b \in H, y \neq b$ \\
    Для этого давайте снова воспользуемся критерием Коши: \\
    $\left|\dfrac{f(x_1, y) - f(x_1, b)}{y - b} - \dfrac{f(x_2, y) - f(x_2, b)}{y - b}\right|$ \\
    Снова введём функцию $F(y) = f(x_1, y) - f(x_2, y)$ как в первой части доказательства и снова воспользуемся для 
    неё формулой Лагранжа. Перепишем равенство: \\
    $\left|\dfrac{f(x_1, y) - f(x_1, b)}{y - b} - \dfrac{f(x_2, y) - f(x_2, b)}{y - b}\right| =
    \left|\dfrac{F'(y*) \cdot (y - b)}{y - b}\right| = |F'(y*)| $, где $y* \in [b, y]$ \\
    Тогда, вернувшись к записи с $f$ получаем: \\
    $\left|\dfrac{f(x_1, y) - f(x_1, b)}{y - b} - \dfrac{f(x_2, y) - f(x_2, b)}{y - b}\right| =
    |f'_y(x_1, y*) - f'_y(x_2, y*)| < \epsilon$, так как по условию теоремы
    $f'_y(x, y) \overset{y \in H}{\underset{x \to a}{\rightrightarrows}} h(y)$ \\
    
    Теперь осталось воспользоваться теоремой о внесении предела под знак равномерной сходимости: \\
    $\lim_{y \to b} \dfrac{g(y) - g(b)}{y - b} =$ [см. пункт 3.1 лекций] = 
    $\lim_{x \to a} \lim_{y \to b} \dfrac{f(x, y) - f(x, b)}{y - b} = 
    \lim_{x \to a} f'_y(x, b) = h(b) \Rightarrow g'(y) = h(y)$, что и требовалось доказать.
\end{proof}
