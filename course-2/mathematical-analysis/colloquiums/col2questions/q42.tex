% Здесь НЕ НУЖНО делать begin document, включать какие-то пакеты..
% Все уже подрубается в головном файле
% Хедер обыкновенный хсе-теха, все его команды будут здесь работать
% Пожалуйста, проверяйте корректность теха перед пушем

% Здесь формулировка билета
\subsection{Как определяется диффеоморфизм в $\mathbb{R}^m$? Покажите, что отображение обратное к диффеоморфизму, само является диффеоморфизмом}
\begin{definition*}
    Пусть U, X $\subseteq\mathbb{R}^m$ и $\phi:U \rightarrow X$ -- биекция
    \[
        \begin{cases}
            x_1 = \phi_1(u_1, \ldots, u_m) \\
            \ldots \\
            x_m = \phi_m(u_1, \ldots, u_m) \\
        \end{cases}
    \]
    Вводим следующие обозначения
    \[u = (u_1, \ldots, u_m),\text{ }x = (x_1, \ldots, x_m)\]
    Пусть $\phi$ непрерывный диффиренцируема на U и якобиан(определитель матрицы Якоби)
    \[J_\phi(u) = det\frac{\partial x}{\partial u}\]
    сохраняет знак в U и $\neq$ 0
    
    Такое отображение $\phi$ называется \textbf{диффеоморфизмом}.
\end{definition*}
\begin{theorem}
    Отображение обратное к диффеоморфизму само является диффеоморфизмом.
\end{theorem}
\begin{proof}
    Перемножением матриц Якоби этих функций, должна быть единичная матрица. Значит $j_{\phi^1}\neq 0$ тоже сохраняет знак
    $\phi^{-1}$ непрерывно диффиренцируема т.к. $\phi$ непрерывно диффиренцируема.
\end{proof}

