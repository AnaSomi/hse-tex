% Здесь НЕ НУЖНО делать begin document, включать какие-то пакеты..
% Все уже подрубается в головном файле
% Хедер обыкновенный хсе-теха, все его команды будут здесь работать
% Пожалуйста, проверяйте корректность теха перед пушем

% Здесь формулировка билета
\subsection{При каких значениях параметра p > 0 сходятся несобственные интегралы\\ $\int\int_{x^2 + y^2 \leqslant 1}\frac{dxdy}{(x^2 + y^2)^p}$, $\int\int_{x^2 + y^2 \geqslant 1}\frac{dxdy}{(x^2 + y^2)^p}$}
\[\int\int_{x^2 + y^2 \leqslant 1}\frac{dxdy}{(x^2 + y^2)^p}\]
Перейдем к полярным координатам (см. вопрос 44)

\[\int_{0}^{2\pi}d\phi\int_{0}^{1}\frac{dr}{r^{2p}} = 2\pi\cdot\int_{0}^{1}\frac{dr}{r^{2p - 1}}\]
сходится при p < $1$

\[\int\int_{x^2 + y^2 \geqslant 1}\frac{dxdy}{(x^2 + y^2)^p}\]

\[\int_{0}^{2\pi}d\phi\int_{1}^{+\infty}\frac{dr}{r^{2p}} = 2\pi\cdot\int_{1}^{+\infty}\frac{dr}{r^{2p-1}}\]
сходится при p $>1$

