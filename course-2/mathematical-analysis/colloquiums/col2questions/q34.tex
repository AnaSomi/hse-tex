\subsection{Каким образом интеграл определяет некоторый заряд? В каком случае это будет мера?}

Пусть $f$ это ограниченная интегрируемая функция на множестве $D$. В силу свойства аддитивности интеграла имеем
\begin{equation*}
    \nu(A) = \underset{A}{\int} f(x) dx.
\end{equation*}

Неотрицательный заряд является мерой. Тогда если интеграл, описанный выше, является неотрицательной функцией, то заряд является мерой.