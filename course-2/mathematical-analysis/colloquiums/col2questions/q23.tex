% Здесь НЕ НУЖНО делать begin document, включать какие-то пакеты..
% Все уже подрубается в головном файле
% Хедер обыкновенный хсе-теха, все его команды будут здесь работать
% Пожалуйста, проверяйте корректность теха перед пушем

% Здесь формулировка билета
\subsection{Докажите, что равномерно непрерывная на жордановом множестве функция - интегрируема}

\begin{theorem*}
    Равномерно непрерывная на жордановом множестве функция - интегрируема.
\end{theorem*}

\begin{proof}
    \textbf{\underline{Опр.:} } $f$ \textit{равномерно непрерывна на $D$} $\Leftrightarrow\forall \varepsilon>0\ \exists\delta(\varepsilon) >0:\ |x-y|<\delta\Rightarrow |f(x)-f(y)|<\varepsilon$
    Если $\Delta(\tau) < \delta \Rightarrow |x-y|<\delta$ для $x,y\in D_i\Rightarrow |f(x)-f(y)|<\varepsilon\Rightarrow \omega_i\leq \varepsilon$, тогда
    $0\leq \sum\limits_{i=1}^n \omega_i\cdot\mu(D_i)\leq \varepsilon\cdot\sum\limits_{i=1}^n \cdot\mu(D_i)=\varepsilon\cdot\mu(D)$
    так как $D$ измеримо по Жордану, следовательно $\mu(D)$ конечна, значит $\sum\limits_{i=1}^n \omega_i\cdot\mu(D_i)<\varepsilon$, а значит по критерию Дарбу $f$ интегрируема.
\end{proof}
\begin{flushright}
$\blacksquare$
\end{flushright}
