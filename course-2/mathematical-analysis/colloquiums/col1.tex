\documentclass[a4paper]{article}
\usepackage{header}


\title{\Huge Математический Анализ - 2 - Коллоквиум 1}
\author{
    Серёжа Рахманов | \href{https://t.me/virg1n}{telegram}, \href{http://shoraii.github.io}{website}
    \\
    Денис Болонин | \href{https://t.me/ultrakekul}{telegram}

}
\date{Версия от {\ddmmyyyydate\today} \currenttime}

\begin{document}
    \maketitle

    \begin{enumerate}
        \item Дайте определения: числовой ряд, частичная сумма ряда, сумма ряда, сходящийся ряд, расходящийся ряд. Рассмотрим ряд с общим членом $a_n$. Докажите, что $a_n \to 0$.
        \begin{definition}
        Пусть $a_{n}$ -- последовательность, т.е. $\NN \rightarrow \RR$. Формальная бесконечная сумма: $a_1 + a_2 + a_3 + \dots = \sum_{n=1}^{\infty} a_n$ называется рядом.
        $S_N = \sum_{n = 1}^{N} a_n$ -- частичная сумма, сумма ряда: $S = \lim_{N \rightarrow \infty} S_N$
        \end{definition}
        
        Возможны 3 случая:
        \begin{enumerate}
            \item $\exists S \in \RR$
            \item $\exists S = \infty$
            \item $\nexists S$
        \end{enumerate}
        
        В первом случае говорят, что ряд сходится, иначе -- что ряд расходится.

        \begin{comment}~
            Если ряд сходится, то $a_n \rightarrow 0$
        \end{comment}
        \begin{proof}
            $a_n = S_n - S_{n - 1} \rightarrow 0$, т.к. $S_n \rightarrow S$ и $S_{n - 1} \rightarrow S$
        \end{proof}
        \item Сформулируйте критерий Коши сходимости числовой последовательности. Сформулируйте и докажите критерий Коши сходимости числового ряда.
        \begin{definition}
            ${a_n}$ называется фундаментальной, если $\forall \epsilon > 0$  $\exists N: \forall n > m > N, |S_n - S_m| < \epsilon$
        \end{definition}
        \begin{theorem}
            ${S_n}$ -- сходится $\Leftrightarrow {S_n}$ -- фундаментальная
        \end{theorem}
        \begin{proof}
        $S_n - S_m = \sum_{k = m + 1}^{n} a_{k}$
        Тогда $\sum a_n$ -- сходится $\Leftrightarrow$ $\forall \epsilon > 0$  $\exists N: \forall n > m > N$
        $|a_{m + 1} + a_{m + 2} + \dots + a_{n}| < \varepsilon$
        \end{proof}
        \item Сформулируйте и докажите признак сравнения положительных числовых рядов, основанный на неравенстве $a_n \leq b_n$.
        
        $a_n \leqslant b_n$ при всех $n \geqslant n_0$
	
	    Ряд $\sum b_n$ cходится $\implies$ ряд $\sum a_n$ сходится
	
        Ряд $\sum a_n$ расходится $\implies$ ряд $\sum b_n$ расходится
        
        \begin{proof}
        	На основании того, что отбрасывание конечного числа элементов ряда не отражается на его поведении, мы можем считать, что $a_n \leqslant b_n$ при всех $n = 1, 2, 3, \dots$ Обозначив частные суммы через $A$ и $B$ соответственно, имеем $A_n \leqslant B_n$. Пусть ряд $\sum b_n$ сходится, тогда $B_n$ ограничена, $B_n \leqslant S (S = const, n = 1, 2, 3, \cdots)$. В таком случае $A_n$ также меньше либо равна некоторому $S$, что даёт нам ограниченность $\sum a_n$.
        \end{proof}
    
        \item Сформулируйте и докажите признак сравнения положительных числовых рядов, основанный на неравенстве $\frac{a_{n+1}}{a_n} \leqslant \frac{b_{n+1}}{b_n}$.
        
        Ряд $\sum b_n$ cходится $\implies$ ряд $\sum a_n$ сходится
	
        Ряд $\sum a_n$ расходится $\implies$ ряд $\sum b_n$ расходится
        
        \begin{proof}~
        
        $a_{n_0+1} \leqslant \frac{a_{n_0}}{b_{n_0}}\cdot b_{n_0 + 1}$
        
        $a_{n_0+2} \leqslant \frac{a_{n_0 + 1}}{b_{n_0 + 1}}\cdot b_{n_0 + 2} \leqslant \frac{a_{n_0}}{b_{n_0}}\cdot b_{n_0 + 2}$
        
        $\vdots$
        
        $a_{n_0+k} \leqslant \frac{a_{n_0}}{b_{n_0}}\cdot b_{n_0 + k} \implies \sum_{n=n_0}^{N} a_n \leqslant \frac{a_{n_0}}{b_{n_0}}\cdot \sum_{n=n_0}^{N} b_n$
        \end{proof}

        \item Сформулируйте и докажите признак сравнения положительных числовых рядов, основанный на пределе $\lim \frac{a_n}{b_n}$.
        
        $\lim_{n \to \infty} \frac{a_n}{b_n} \in (0; +\infty) \implies$ сходимость $\sum a_n \iff$ сходимость $\sum b_n$
        
        \begin{proof}~
            
        $c = \lim_{n \to \infty} \frac{a_n}{b_n} > 0$
        
        $\forall \epsilon\ \exists n_0:\ c - \epsilon \leqslant \frac{a_n}{b_n} \leqslant c + \epsilon$, при $n \geqslant n_0$
        
        Возьмём $c - \epsilon > 0 \implies (c - \epsilon)\cdot b_n \leqslant a_n \leqslant (c + \epsilon)\cdot b_n$
        
        Сходимость следует из правой части неравенства, а расходимость из левой. 
        \end{proof}
        \item Пусть последовательности $\{a_n\}$, $\{A_n\}$ таковы, что $a_n - (A_n - A_{n - 1}) = c_n$ и ряд $\sum c_n$ сходится.
        Докажите, что существует $C$ такое, что $a_1 + a_2 + \dots + a_n = A_n + C + o(1)$. 
        \item Сформулируйте и докажите признак Лобачевского-Коши.
        \begin{proposal}
            Пусть $a_n > 0$ и $a_n \downarrow$
        
            Тогда ряды $\sum a_n$ и $\sum 2^n \cdot a_{2^n}$ ведут себя одинаково
        \end{proposal}
        \begin{proof}
            $a_1 + (a_2) + (a_3 + a_4) + (a_5 + \dots + a_8) + \dots$
        
            $a_2 \leq a_1$
            
            $a_2 \leq a_2$
        
            
            $a_3 + a_4 \leq 2a_2$
            
            $a_3 + a_4 \geq 2a_4$
        
            $a_5 + \dots + a_8 \leq 4a_4$
        
            $a_5 + \dots + a_8 \geq 4a_8$
        
        
            $\dots$
        
            $a_1 + \sum_{n=0}^{m - 1} 2^n a_{2n} \leq \sum_{n = 1}^{2^m} a_n \leq a_1 + \dfrac{1}{2} \sum_{n=0}^{m} 2^n a_{2n}$
        
        \end{proof}
        \item Примените признак Лобачевского-Коши к ряду $\sum_{n=2}^{\infty}\dfrac{1}{n \ln n \ln^{p}(\ln n)}$, $p > 0$
		
		Рассмотрим данный нам ряд. Заметим, что $\dfrac{1}{n \ln n \ln^{p}(\ln n)}$ убывает, поскольку $n \ln n \ln^{p}(\ln n)$ является возрастающей функцией ($n, \ln n$ и $\ln^{p}(\ln n)$ сами по себе возрастают). Кроме того, $\forall n, n \geqslant 2, a_n > 0$, поскольку $1 > 0$ и $n \ln n \ln^{p}(\ln n) > 0$. В таком случае, аналогично данному ряду будет вести себя ряд $\sum_{n=2}^{\infty}\dfrac{2^n}{2^n \ln 2^n \ln^{p}(\ln 2^n)} = \sum_{n=2}^{\infty}\dfrac{1}{\ln 2^n \ln^{p}(\ln 2^n)}, p > 0$.

        \item Сформулируйте теорему Штольца о пределе последовательности $\frac{p_n}{q_n}$, $p_n$, $q_n \to 0$.
        \begin{theorem}
            (Штольца.) Если $p_n, q_n \to 0, q_n \downarrow$ и $\exists lim \dfrac{p_{n + 1} - p_n}{q_{n + 1} - q_n}$, то
            $\lim \dfrac{p_n}{q_n} = \lim \dfrac{p_{n + 1} - p_n}{q_{n + 1} - q_n}$
        \end{theorem}
        \item Покажите на примере, как с помощью теоремы Штольца можно уточнить асимптотическую оценку для частичной суммы ряда.
        \item Пусть $\sum a_n$, $\sum a_n'$ - сходящиеся положительные ряды. Говорят, что ряд $\sum a_n'$ сходится быстрее ряда $\sum a_n$, если $a_n' = o(a_n)$. Докажите, что в этом случае также $r_n' = o(r_n)$, где $r_n$, $r_n'$ - остатки соответствующих рядов.
        
        Рассмотрим остатки каждого из рядов. $r_n = S - S_N$, где $S_N$ - частичная сумма ряда $\sum a_n$ и $S_N \rightarrow S$ при $N \rightarrow \infty$. Для $\sum a_n'$ аналогично $r_n' = S' - S_N'$, где $S_N'$ - частичная сумма ряда $\sum a_n'$ и $S_N' \rightarrow S'$ при $N \rightarrow \infty$. Идёт речь о том, что ряд $a_n'$ сходится быстрее ряда $a_n$, т.е. оба ряда сходятся и $S = S'$. Но, поскольку члены рядов находятся в отношении $a_n' = o(a_n)$, то мы можем сделать выводы о частичных суммах $S_N$ и $S_N'$. $\forall N, S_N' = o(S_N)$, что указывает нам в результате на отношение между остатками $r_n' = o(r_n)$.
        \item Пусть $\sum a_n$, $\sum a_n'$ - расходящиеся положительные ряды. Говорят, что ряд $\sum a_n'$ расходится медленнее ряда $\sum a_n$, если $a_n' = o(a_n)$. Докажите, что в этом случае также $S_n' = o(S_n)$, где $S_n$, $S_n'$ - частичные суммы соответствующих рядов.
        
        Оба ряда расходятся, тогда $S_n \rightarrow \infty$ и $S_n' \rightarrow \infty$ при $n \rightarrow \infty$. Мы понимаем, что $S_n = \sum_{n = 1}^{N} a_n$, $S_n' = \sum_{n = 1}^{N} a_n'$. Это значит, что для некоторого $n_1$ мы имеем следующее: $S_{n_1} = \sum_{n = 1}^{n_1} a_n$, $S_{n_1}' = \sum_{n = 1}^{n_1} a_n'$, где для любого $n = 1, 2, 3, \dots, n_1$ выполняется отношение $a_n' = o(a_n)$. В таком случае для частичных сумм справедливо отношение $S_{n_1}' = o(S_{n_1})$. А так как и для всех последующих $a_n$ и $a_n'$ также справедливо отношение $a_n' = o(a_n)$, то мы можем сказать, что $S_n' = o(S_n)$.
        \item -
        \item -
        \item Сформулируйте признак Даламбера для положительного ряда
        \begin{theorem}
            Признак Даламбера. Пусть $a_n > 0$.
            
            $\overline{\lim} \dfrac{a_{n+1}}{a_n} < 1 \implies $ ряд $\sum a_n$ сходится.
            
            $\underline{\lim} \dfrac{a_{n+1}}{a_n} > 1 \implies $ ряд $\sum a_n$ расходится.
        \end{theorem}
        \item Сформулируйте радикальный признак Коши для положительного ряда.
        \begin{theorem}
            Радикальный признак Коши. Пусть $a_n \geq 0$.
            
            $\overline{\lim} \sqrt[n]{a_n} < 1 \implies$ ряд $\sum a_n$ сходится.
             
            $\underline{\lim} \sqrt[n]{a_n} > 1 \implies$ ряд $\sum a_n$ расходится.
        \end{theorem}
        \item Докажите, что всякий раз, когда признак Даламбера даёт ответ на вопрос о сходимости ряда, то радикальный признак Коши даёт тот же ответ на этот вопрос.
        
        Пусть $a_n > 0$. Тогда:

        $$ \underline{\lim} \dfrac{a_{n+1}}{a_n} \leq \underline{\lim}{\sqrt[n]{a_n}} \leq \overline{\lim}{\sqrt[n]{a_n}} \leq \overline{\lim}\dfrac{a_{n+1}}{a_n}$$

        Если $\overline{\lim}\dfrac{a_{n+1}}{a_n} < 1 \implies \overline{\lim}{\sqrt[n]{a_n}} < 1$

        Если $\underline{\lim}\dfrac{a_{n+1}}{a_n} > 1 \implies \underline{\lim} \sqrt[n]{a_n} > 1$

        Если $\exists \lim \frac{a_{n+1}}{a_n}$, то $\overline{\lim} \frac{a_{n+1}}{a_n} = \underline{\lim} \frac{a_{n+1}}{a_n} \Rightarrow \exists \lim \sqrt[n]{a_n} = \lim \frac{a_{n+1}}{a_n}$
        \item -
        \item -
        \item Приведите пример ряда, который сходится медленнее любого ряда геометрической прогрессии, но быстрее любого обобщённого гармонического яда (с обоснованием).
        
        Рассмотрим ряд $\sum \dfrac{1}{n (\ln n) (\ln \ln n)^2}$. Применим радикальный признак Коши для того, чтобы определить, быстрее или медленнее геометрической прогрессии сходится данный ряд. Проанализируем $\sqrt[n]{\dfrac{1}{n (\ln n) (\ln \ln n)^2}}$. Пусть $n \rightarrow \infty$, заметим, что в знаменателе фигурируют натуральные логарифмы, и самое главное - $n$, делаем вывод о том, что дробь ведёт себя схоже с гиперболой и при бесконечно больших $n$ она сходится к 0 и корень $n$-ой степени будет давать нам число, близкое к 1, и чем больше будет $n$, тем меньше будет дробь, и тем больше будет корень. Таким образом, получаем, что $\sqrt[n]{\dfrac{1}{n (\ln n) (\ln \ln n)^2}}$ сходится к 1, а значит наш ряд сходится медленнее геометрической прогрессии, сравнение с которой происходит в радикальном признаке Коши.
        	
        Теперь сравним наш ряд с гармоническим. Мы понимаем, что гармонический ряд имеет вид $\sum \dfrac{1}{n}$, так что для нашего ряда для любого $n$ начиная с некоторого $n_0$ будет справедливо $a_n = o(a_n')$, где $a_n$ - элемент нашего ряда, $a_n'$ - элемент гармонического ряда. В таком случае, так как оба ряда сходятся (можно проверить наш ряд на сходимость интегральным признаком), то мы получим, что наш ряд сходится быстрее гармонического ряда (пункт 11 говорит, почему).
        
        \item Сформулируйте признак Гаусса для положительного ряда. Приведите пример применения признака Гаусса.

        Если $\exists \delta > 0,\; p$:$ \dfrac{a_{n+1}}{a_n} = 1 - \dfrac{p}{n} + O\left(\dfrac{1}{n^{1 + \delta}}\right) $
        то:

        $p > 1 \implies$ ряд $\sum a_n$ сходится.

        $p \leq 1 \implies$ ряд $\sum a_n$ расходится.
        \item -
        \item -
        \item Что такое улучшение сходимости положительного ряда? Покажите на примере как можно улучшить сходимость ряда.

        Пусть у нас есть некоторый ряд $\sum a_n$ и он сходится медленно. В таких случаях для расчёта суммы ряда с необходимой точностью потребуется взять больше членов, что неудобно. Мы можем преобразовать наш ряд для улучшения сходимости, т.е. получить некоторый ряд $\sum a_n'$, который будет сходиться быстрее, чем исходный $\sum a_n$.
       	\begin{example}
            Пусть у нас есть ряд $S = \sum_{n = 1}^{\infty} \dfrac{1}{n^2 + 2} \approx \sum_{n = 1}^{\infty} \dfrac{1}{n^2}$. Воспользуемся  методом Куммера. Для улучшения сходимости будем брать ряды вида $\sum_{n = 1}^{\infty} \dfrac{1}{n(n+1)} = 1, \sum_{n = 1}^{\infty} \dfrac{1}{n(n+1)(n+2)} = \dfrac{1}{4}, \dots$. 
            
            В данном случае нам подойдёт первый ряд в этом списке, поскольку $\dfrac{1}{n^2} \sim \dfrac{1}{n(n+1)}$.
            
            $\sum_{n = 1}^{\infty} \left( \dfrac{1}{n^2 + 2} - \dfrac{1}{n(n + 1)} \right) = S - 1 \implies S = 1 + \sum_{n = 1}^{\infty} \left( \dfrac{1}{n^2 + 2} - \dfrac{1}{n(n + 1)} \right)$.
             
            $\dfrac{1}{n^2 + 2} - \dfrac{1}{n(n + 1)} = \dfrac{1}{n^2} \cdot \left( \dfrac{1}{\frac{2}{n^2}} - \dfrac{1}{1 + \frac{1}{n}} \right) = \dfrac{1}{n^2} \cdot \left( 1 - \dfrac{2}{n^2} + o \left( \dfrac{1}{n^2} \right) - 1 + \dfrac{1}{n} - \dfrac{1}{n^2} - o \left( \dfrac{1}{n^2} \right) \right) = \dfrac{1}{n^3} + o \left( \dfrac{1}{n^3} \right)$.
         
            Получили ряд $\sum_{n = 1}^{\infty} \dfrac{1}{n^3}$, который сходится быстрее, $1 + \sum_{n = 1}^{\infty} \dfrac{1}{n^3} \approx \sum_{n = 1}^{\infty} \dfrac{1}{n^2 + 2}$.
        \end{example}
    
   		\item Дайте определения: знакопеременный ряд, знакочередующийся ряд, абсолютно сходящийся ряд, условно сходящийся ряд, положительная часть ряда, отрицательная часть ряда.
    
    	\begin{definition}
    		Пусть существует ряд $\sum a_n$. такой, что $\forall i$, $a_i$ может быть, как больше 0, так и меньше 0. В таком случае ряд $\sum a_n$ называется знакопеременным.
    	\end{definition}
    		
    	\begin{definition}
    		Пусть существует ряд $\sum a_n$. такой, что $\forall i$, $a_i \cdot a_{i+1} < 0$. В таком случае ряд $\sum a_n$ называется знакочередующимся.
    	\end{definition}
    
    	\begin{definition}
    		Рассмотрим дополнительный ряд $\sum |a_n|$. В случае, когда он расходится, мы называем ряд $\sum a_n$ абсолютно сходящимся. Если $\sum |a_n|$ расходится, то $\sum a_n$ называется сходящимся условно.
   		\end{definition}
   	
   		\begin{definition}
   			Введем два ряда: $a_n^+ = \begin{cases}
   			a_n, a_n > 0 \\
   			0
   			\end{cases}$ 
   			и $a_n^- = \begin{cases}
   			|a_n|, a_n < 0 \\
   			0
   			\end{cases}$.
   			Тогда ряды $\sum a_n^+$ и $a_n^-$ соответственно называются положительной и отрицательной частью ряда $\sum a_n$.
   		\end{definition}
   	
   		\item Докажите, что ряд сходится абсолютно ровно в том случае, когда сходятся его положительная и отрицательная части.
   			
   		\begin{proof}
   		
   			Рассмотрим ряд $\sum a_n$, дополнительный ряд $\sum |a_n|$, а также положительную и отрицательную части $\sum a_n^+$ и $\sum a_n^-$.
   			
   			1) Пусть ряд $\sum a_n$ сходится абсолютно. В таком случае ряд $\sum |a_n|$ сходится, а так как члены рядов $\sum a_n^+$ и $\sum a_n^-$ все содержатся в ряде $\sum |a_n|$, то для всех их частичных сумм справедливо следующее: $P_k \leqslant A_n'$ и $Q_m \leqslant A_n'$, где $P_k$ и $Q_m$ - частичные суммы положительной и отрицательной части соответственно, а $A_n'$ - частичная сумма дополнительного ряда и $A_n' = P_k + Q_m, n = m + k$. Это значит, что оба ряда $\sum a_n^+$ и $\sum a_n^-$ сходятся.
   			
   			2) Исходя из того, что $S_n = P_k - Q_m, n = m + k$ и положительных и отрицательных элементов в $\sum a_n$ бесконечное множество, мы получаем, что при $n \rightarrow \infty$ одновременно $m \rightarrow \infty$ и $k \rightarrow \infty$. Переходя к пределу получаем, что исходный ряд сходится абсолютно и его сумма будет равна $P - Q$.
   			
   			
   		\end{proof}
   	
   		\item Докажите, что если ряд сходится условно, то его положительная и отрицательная части расходятся (имеют бесконечные суммы).
   		
   		\begin{proof}
   			Рассмотрим ряд $\sum a_n$, дополнительный ряд $\sum |a_n|$, а также положительную и отрицательную части $\sum a_n^+$ и $\sum a_n^-$. Поскольку ряд $\sum a_n$ сходится условно, то $\sum |a_n|$ расходится. Рассмотри частичные суммы $\sum |a_n|$, $\sum a_n^+$ и $\sum a_n^-$ - $A_n', P_k, Q_m$ соответственно. Для любого $n = m + k$, $A_n' = P_k + Q_m$. При $n \rightarrow \infty$, $m \rightarrow \infty$ и $k \rightarrow \infty$. Так как ряд $\sum |a_n|$ расходится, то сумма $A_n' \rightarrow \infty$. Поскольку число положительных и отрицательных элементов бесконечно, то получаем $P_k \rightarrow \infty$ и $Q_m \rightarrow \infty$, а значит ряды $\sum a_n^+$ и $\sum a_n^-$ расходятся.
   		\end{proof}
           
        \item Сформулируйте мажорантный признак Вейерштрасса. Приведите пример применения признака
        
        \begin{theorem}
            Если $|a_n| \leq b_n$ при $n > n_0$ и положительный ряд $\sum b_n$ сходится,
            то $\sum a_n$ сходится, причём абсолютно.
        \end{theorem}
        
        \begin{example}
        $\sum_{n=1}^{\infty} \dfrac{\sin(nx)}{n^p}$, $p > 0$
        
        $|sin(nx)| \leq 1 \implies \left|\dfrac{sin(nx)}{n^P}\right| \leq \dfrac{1}{n^p}$
        
        $\sum \dfrac{1}{n^p} $ сходится $(p > 1) \implies \sum_{n=1}^{\infty} \dfrac{\sin(nx)}{n^p}$ сходится абсолютно.
        \end{example}

        \item Что такое группировка членов ряда? Докажите, что любой ряд, полученный из сходящегося ряда группировкой его членов, сходится и имеет ту же сумму
        
        Говорят, что ряд $\sum b_k$ получен из $\sum a_n$ группировкой членов, если $\exists n_1 < n_2 < \dots$:

        $b_1 = a_1 + a_2 + \dots + a_{n_1}$

        $b_2 = a_{n_1 + 1} + a_{n_1 + 2} + \dots + a_{n_2}$

        $\dots$

        \begin{comment}
            Если $\sum a_n$ сходится, то ряд $\sum b_k$ сходится к той же сумме.
        \end{comment}

        \begin{proof}
        $\sum_{k=1}^{m} b_k = \sum_{n=1}^{n_m} a_n$
        \end{proof}

        \textit{Обратное утверждение неверно:} $(1 - 1) + (1 - 1) + \dots$

        \item Как с помощью группировки преобразовать знакопеременный ряд в знакочередующийся? Что можно утверждать о сходимости полученного знакочередующегося ряда?
        
        Знакопеременный ряд при помощи группировки сводится к знакочередующемуся:

        $a_1 \leq 0$, $\dots$, $a_{n_1} \leq 0$; $b_1 = \sum_{i=1}^{n_1} a_i \leq 0$

        $a_{n_1+1} \geq 0$, $\dots$, $a_{n_2} \geq 0$; $b_1 = \sum_{i={n_1 + 1}}^{n_2} a_i \leq 0$

        При такой группировке сходимость исходного ряда $\iff$ сходимость $\sum b_n$

        \item Приведите пример преобразования знакопеременного ряда к знакочередующемуся.

        \begin{example}
            $\sum_{n=1}^{\infty} \dfrac{(-1)^{[\ln n]}}{n}$

            $\sum_{k=0}^{\infty} b_k$, где $b_k = (-1)^k$

            $|b_k| = \sum_{n=[e^k] + 1}^{[e^{k+1}]} \dfrac{1}{n} \leq \dfrac{1}{[e^k] + 1} \cdot ([e^{k+1}]-[e^k]) \approx \dfrac{e^{k+1} - e^k}{e^k} \to e - 1 > 0$
        \end{example}

        \item -
        \item Сформулируйте признак Лейбница для знакочередующегося ряда. Приведите пример применения признака Лейбница.
        
        \begin{theorem}
            Признак Лейбница. Если $u_n \downarrow 0$, то ряд сходится, причём $|r_n| \leq u_{n+1}$
            \end{theorem}
            
            \begin{example}
                $\sum_{n=1}^{\infty} \dfrac{(-1)^{n}}{n^p}$, $p > 0$
            
                $\dfrac{1}{n^p} \downarrow 0 \implies $ ряд сходится (при $\forall p > 0$)
            \end{example}

        \item -
        \item -
        
        \item Сформулируйте признак Дирихле. Приведите пример его применения.
        
        $\sum_{n=1}^{\infty}a_n \cdot b_n$

        \begin{theorem}
            Признак Дирихле. Если $a_n \downarrow 0$, а частичные суммы $\left| \sum_{n=1}^N b_n \right| \leq C$ ограничены,
            то $\sum_{n=1}^{\infty}a_n \cdot b_n$ сходится.
        \end{theorem}

        \begin{example}
            $\sum_{n=1}^{\infty} \dfrac{\sin(nx)}{n^p}$, $p > 0$
        
            $a_n = \dfrac{1}{n^p} \downarrow 0$, $b_n = \sin nx$
        
            $b_1 + b_2 + b_3 + \dots + b_N = \sin x+ \sin 2x + \dots + \sin Nx = \dfrac{\cos \dfrac{x}{2} - \cos\left((N + 1/2)x\right)}{2 \sin \dfrac{x}{2}}$; $\left|\sum_{n=1}^{N}b_n\right| \leq \dfrac{2}{2\sin{\dfrac{x}{2}}} = \dfrac{1}{\sin{\dfrac{x}{2}}}$
        
            Ряд сходится по признаку Дирихле
        \end{example}

        \item Сформулируйте признак Абеля. Выведите утверждение признака Абеля из признака Дирихле.
        
        \begin{theorem}
            Признак Абеля. Если $a_n$ монотонна и ограничена, а ряд $\sum_{n=1}^{\infty}b_n$ сходится,
            то $\sum_{n=1}^{\infty}a_n \cdot b_n$ сходится.
        \end{theorem}
        
        % $a_n \to a$, $a_n = a +- \alpha_n$, $\alpha_n \downarrow 0$; $\sum_{n=1}^{\infty}a_n \cdot b_n = a \sum_{n=1}^{\infty}b_n +- \sum_{n=1}^{\infty}\alpha_n \cdot b_n$

        \item Что такое перестановка членов ряда? Приведите пример.
        
        Пусть $f: \NN \to \NN$ -- биекция

        Говорят, что ряд $\sum b_n$ получен из $\sum a_n$ перестановкой членов, если $b_n = a_{f(n)}$

        \item Сформулируйте свойство абсолютно сходящегося ряда, связанное с перестановкой членов. (теорема Римана)
        
        \begin{theorem}
            (Римана) Если ряд $\sum a_n$ сходится условно, то для $\forall S \in [-\infty; +\infty]$ то $\exists$ перестановка $f$ такая, что $\sum a_{f(n)} = S$
        \end{theorem}
    \end{enumerate}

\end{document}