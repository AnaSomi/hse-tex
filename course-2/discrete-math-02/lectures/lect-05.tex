\ProvidesFile{lect-05.tex}[Лекция 5]

\section{Лекция 5}

\subsection{Неформальное описание машины Тьюринга}

Что такое машины Тьюринга и зачем они нам нужны?
В предшествующих лекциях мы с вами работали с понятием алгоритма и достаточно далеко развили соответствующую теорию при том, что алгоритм мы никак не определяли.
Дело в том, что определений алгоритма может быть много, и нельзя сказать, что одно из них хуже или лучше другого, ровно как и нельзя сказать, что один язык программирования лучше или хуже другого.
В общем-то, эти определения могут выглядеть по-разному, однако они эвкивалентны в том смысле, что дают одно и то же множество вычислимых функций.
Собственно говоря, именно по тому, какие функции считаются вычислимыми, мы и оцениваем данное конкретное понятие алгоритма.

Машина Тьюринга --- это такая модель алгоритмов, которая немного похожа на примитивный электромеханический (или даже электронный) компьютер.
Она не самая удобная для практических рассуждений или математических доказательств, но, с другой стороны, она сочитает в себе относительную простоту, наглядность и относительную близость к практике.
Так что машины Тьюринга прочно укрепились в преподавании и, по-сути, являются простейшими моделями, которые нам надо знать.
Дадим неформальное описание машины Тьюринга.

\ProvidesFile{lect-05-drawing-01.tex}[Машина Тьюринга 1]

\begin{figure}[h!]
    \centering
    \begin{tikzpicture}[every node/.style={block},
                        block/.style={
                            minimum height=1.5em,
                            outer sep=0pt,
                            draw,
                            rectangle,
                            node distance=0pt
                        },
                        roundnode/.style={round},
                        round/.style={
                            minimum height=1.5em,
                            outer sep=0pt,
                            draw,
                            circle,
                            node distance=0pt
                        }]
        \node (A) {$\sigma$};
        \node (B) [left=of A] {$\ldots$};
        \node (C) [left=of B] {$a$};
        \node (D) [right=of A] {$\ldots$};
        \node (E) [right=of D] {$\$ $};
        \node[roundnode] (F) [above = 0.75cm of A, draw=red, thick] {$q$};
        
        \draw[-latex] (F) -- (A);
        \draw[-latex] (A) -- (F);
        
        \draw[-latex, blue] ($(F.east)!0.5!(A.east)$) -- ++(7mm, 0);
        \draw[-latex, blue] ($(F.west)!0.5!(A.west)$) -- ++(-7mm, 0);
        
        \draw (C.north west) -- ++(-1cm, 0) 
              (C.south west) -- ++(-1cm, 0) 
              (E.north east) -- ++(1cm, 0) 
              (E.south east) -- ++(1cm, 0);
    \end{tikzpicture}
\end{figure}

Суть в том, что у нас имеется некоторая такая лента, разбитая на ячейки.
Такую ленту можно рассматривать как оперативную память.
Каждая ячейка имеет конечную емкость, то есть число состояний ячейки конечно.
Можно считать, что в ячейку пишется какой-то символ $a$ из {\it конечного} алфавита $\Gamma$.
По ленте ездит некоторое абстрактное устройство, которое называется {\it головкой}.
Что же эта головка делает?
Головка может читать символ из ячейки, и может писать символ в эту ячейку.
Также она может ленту двигать на один шаг влево или на один шаг вправо.
Соответственно, чтобы прочитать с ленты несколько последовательных символов, машина должна несколько раз ленту переместить (или проехаться по ней).
Кроме того, у головки самой имеется некоторая внутренняя память, которая содержательно нужна для того, чтобы устанавливать связь между содержимым ячеек.
Мы моделируем эту память следующим образом: мы считаем, что у головки есть какое-то состояние $q \in Q$, в котором она находится, и эти состояния вместе образуют некоторый {\it конечный алфавит состояний} $Q$.
Почему память можно моделировать конечным алфавитом состояний?
Если вы рассмотрите модель памяти в вашем компьютере в предположении, что вы не можете ее докупать, то эту память можно заполнить лишь конечным числом способов.
Например, если она имеет \enquote{длину} в $n$ бит, то мы можем придать ей лишь $2^{n}$ различных состояний.
С практической точки зрения удобно думать, что память головки --- это что-то типа состояния регистра процессора, а лента --- оперативная память (которая, при этом, считается неограниченной).
Мы считаем, что чистые (нетронутые) ячейки памяти заполнены специальным символом \# $\in \Gamma$, который называется {\it пробельным символом}.

Как же наша машина Тьюринга работает?
Работает она в соответствии с некоторой \enquote{программой}.
Тут важно понимать аналогию: не смотря на то, что мы говорим, что машина Тьюринга --- это модель компьютера, она является таким компьютером, который выполняет только одну программу.
Как же выглядит программа?
Программа, по сути дела, указывает машине что делать, в каком состоянии, в зависимости от символа, написанного в данной ячейке, в которой находится головка.
Программа состоит из инструкций вида $q a \mapsto q^{\prime} a^{\prime} S$, где
\begin{itemize}
    \item $q, q^{\prime} \in Q$ --- состояния;
    \item $a, a^{\prime} \in \Gamma$ --- символы алфавита;
    \item $S \in \left\{L, N, R\right\}$ --- элемент некоторого трехэлементного множества, в котором $L = $ \enquote{left}, $N = $ \enquote{neutral}, $R = $ \enquote{right}.
    Это символы, которые говорят нам, куда нужно сдвинуть головку.
    Например, $R$ означает, что машина должна сдвинуться в ячейку справа от текущей.
\end{itemize}
Интерпретация инструкций такая: если в состоянии $q$ увидишь $a$ в текущей ячейке, то запиши $a^{\prime}$ в текущую ячейку, перейди в состояние $q^{\prime}$ и сместись на $S$.
То есть, по сути дела, чем определяется поведение машины Тьюринга?
Поведение ее определяется тем, что написано на ленте, изначальным положением головки и ее состоянием.
На самом деле, в каждый момент времени, будущее состояние нашей машины определяется такой информацией.
Такая информация называется {\it конфигурацией машины Тьюринга}.

\begin{definition}
    Конечное множество $\Gamma$ называется ленточным алфавитом.
\end{definition}

\begin{definition}
    Будем обозначать через $\Gamma^{*}$ множество всех конечных последовательностей элементов из алфавита $\Gamma$.
    $$
        \Gamma^{*} = \bigcup\limits_{n \in \N} \Gamma^{n}.
    $$
\end{definition}

\begin{definition}
    Зафиксируем $A, B \in \Gamma^{*}$ --- какие-то слова ленточного алфавита, $q$ --- состояние головки МТ\footnote{машины Тьюринга --- прим.}, $c \in \Gamma$ --- символ в текущей ячейке, на которую машина смотрит.
    Тогда набор $\left(A, B, q, c\right)$ называется конфигурацией машины Тьюринга.
\end{definition}

Конфигурация трактуется следующим образом:
\begin{itemize}
    \item $A$ --- то, что стоит слева от головки;
    \item $B$ --- то, что стоит справа от головки;
    \item $c$ --- символ, который находится под головкой.
\end{itemize}
Проиллюстрировать это можно, например, вот так:

\ProvidesFile{lect-05-drawing-02.tex}[Машина Тьюринга 2]

\begin{figure}[h!]
    \centering
    \begin{tikzpicture}[every node/.style={block},
                        block/.style={
                            minimum height=1.5em,
                            outer sep=0pt,
                            draw,
                            rectangle,
                            node distance=0pt
                        },
                        roundnode/.style={round},
                        round/.style={
                            minimum height=1.5em,
                            outer sep=0pt,
                            draw,
                            circle,
                            node distance=0pt
                        }]
        \node (A) {$c$};
        \node (LB) [left=of A] {$a_{n}$};
        \node (LC) [left=of LB] {$\ldots$};
        \node (LD) [left=of LC] {$a_{0}$};
        \node (LE) [left=of LD] {\#};
        \node (LF) [left=of LE] {$\ldots$};
        \node (RB) [right=of A] {$b_{0}$};
        \node (RC) [right=of RB] {$\ldots$};
        \node (RD) [right=of RC] {$b_{k}$};
        \node (RE) [right=of RD] {\#};
        \node (RF) [right=of RE] {$\ldots$};
        \node[roundnode] (UA) [above = 0.75cm of A, draw=red, thick] {$q$};
        
        \draw[-latex] (UA) -- (A);
        \draw[-latex] (A) -- (UA);
        
        \draw[-latex, blue] ($(UA.east)!0.5!(A.east)$) -- ++(7mm, 0);
        \draw[-latex, blue] ($(UA.west)!0.5!(A.west)$) -- ++(-7mm, 0);
        
        \draw (LF.north west) -- ++(-1cm, 0) 
              (LF.south west) -- ++(-1cm, 0) 
              (RF.north east) -- ++(1cm, 0) 
              (RF.south east) -- ++(1cm, 0);
    \end{tikzpicture}
\end{figure}

Здесь $\left(a_{0}, \ldots, a_{n}\right) = A$, $\left(b_{0}, \ldots, b_{k}\right) = B$.
Обратим внимание, что нельзя сказать, что машина находится в какой-то одной конфигурации, потому что мы можем к словам $A$ и $B$ дописывать пробельные символы.
Поэтому, если быть супер формальными, речь идет о классах эквивалентности.
Поэтому мы не говорим, что машина Тьюринга находится в такой конфигурации, а говорим, что происходит на ленте, если конфигурация такая.
Здесь очень много всяких мелких деталей, поэтому мы их скрыли при построении теории алгоритмов.

Итак, в соответствии с программой и своей конфигурацией, МТ делает шаги.
Один шаг состоит в исполнении одной инструкции вида $q a \mapsto q^{\prime} a^{\prime} S$.

\begin{definition}
    Процесс (протокол) вычисления --- конечная последовательность конфигураций, где каждая следующая конфигурация получается из предыдущей за один шаг машины.
\end{definition}

Это определение можно себе представить следующим образом:
$$
    c_{1} \mto c_{2} \mto c_{3} \mto \ldots \mto c_{n},
$$
где $\MM$ --- обозначение машины Тьюринга, а $c_{i}$ --- $i$-ая конфигурация $\MM$.
Отношение \enquote{$\mto$} означает, что одна операция получается из другой за один шаг $\MM$.
Сам такой процесс мы можем понимать как {\it отношение достижимости конфигурации} за $n$ шагов.

Мы написали много всего абстрактного, посмотрим, как вся эта теория может работать.
Заведем МТ $\MM:$ $\Gamma = \{0, 1, \#\}$, $Q = \{q_{0}, q_{1}\}$.
Зададим ее программу:
\begin{align}
    q_{0} 0 &\mapsto q_{0} 1 R, \\
    q_{0} 1 &\mapsto q_{0} 0 R, \\
    q_{0} \# &\mapsto q_{1} \# N.
\end{align}
Пояснение текстом:
\begin{itemize}
    \item если мы видим $0$ в состоянии $q_{0}$, то давайте напишем $1$ в состоянии $q_{0}$ и сдвинемся направо;
    \item если мы видим $1$ в состоянии $q_{0}$, то давайте напишем $0$ в состоянии $q_{0}$ и сдвинемся направо;
    \item если мы видим \# в состоянии $q_{0}$, то перейдем в состояние $q_{1}$, напишем в этом состоянии \#, и никуда двигаться не будем.
\end{itemize}

\ProvidesFile{lect-05-drawing-03.tex}[Машина Тьюринга 3]

\begin{figure}[h!]
    \centering
    \begin{subfigure}[b]{0.3\textwidth}
        \centering
        \begin{tikzpicture}[every node/.style={block},
                            block/.style={
                                minimum height=1.5em,
                                outer sep=0pt,
                                draw,
                                rectangle,
                                node distance=0pt
                            },
                            roundnode/.style={round},
                            round/.style={
                                minimum height=1.5em,
                                outer sep=0pt,
                                draw,
                                circle,
                                node distance=0pt
                            }]
            \node (A) {$1$};
            \node (LB) [left=of A] {$1$};
            \node (LC) [left=of LB] {$1$};
            \node (LD) [left=of LC] {\#};
            \node (LE) [left=of LD] {$\ldots$};
            \node (RB) [right=of A] {$0$};
            \node (RC) [right=of RB] {$1$};
            \node (RD) [right=of RC] {\#};
            \node (RE) [right=of RD] {$\ldots$};
            \node[roundnode] (UA) [above = 0.75cm of A, draw=red, thick] {$q_{0}$};
            
            \draw[-latex] (UA) -- (A);
            \draw[-latex] (A) -- (UA);
            
            \draw[-latex, blue] ($(UA.east)!0.5!(A.east)$) -- ++(7mm, 0);
            \draw[-latex, blue] ($(UA.west)!0.5!(A.west)$) -- ++(-7mm, 0);
            
            \draw (LE.north west) -- ++(-0.4cm, 0) 
                  (LE.south west) -- ++(-0.4cm, 0) 
                  (RE.north east) -- ++(0.4cm, 0) 
                  (RE.south east) -- ++(0.4cm, 0);
        \end{tikzpicture}
        \caption{Состояние 1.}
    \end{subfigure}
    \hfill
    \begin{subfigure}[b]{0.3\textwidth}
        \centering
        \begin{tikzpicture}[every node/.style={block},
                            block/.style={
                                minimum height=1.5em,
                                outer sep=0pt,
                                draw,
                                rectangle,
                                node distance=0pt
                            },
                            roundnode/.style={round},
                            round/.style={
                                minimum height=1.5em,
                                outer sep=0pt,
                                draw,
                                circle,
                                node distance=0pt
                            }]
            \node (A) {$0$};
            \node (LB) [left=of A] {$1$};
            \node (LC) [left=of LB] {$1$};
            \node (LD) [left=of LC] {\#};
            \node (LE) [left=of LD] {$\ldots$};
            \node (RB) [right=of A] {$0$};
            \node (RC) [right=of RB] {$1$};
            \node (RD) [right=of RC] {\#};
            \node (RE) [right=of RD] {$\ldots$};
            \node[roundnode] (UB) [above = 0.75cm of RB, draw=red, thick] {$q_{0}$};
            
            \draw[-latex] (UB) -- (RB);
            \draw[-latex] (RB) -- (UB);
            
            \draw[-latex, blue] ($(UB.east)!0.5!(RB.east)$) -- ++(7mm, 0);
            \draw[-latex, blue] ($(UB.west)!0.5!(RB.west)$) -- ++(-7mm, 0);
            
            \draw (LE.north west) -- ++(-0.4cm, 0) 
                  (LE.south west) -- ++(-0.4cm, 0) 
                  (RE.north east) -- ++(0.4cm, 0) 
                  (RE.south east) -- ++(0.4cm, 0);
        \end{tikzpicture}
        \caption{Состояние 2.}
    \end{subfigure}
    \hfill
    \begin{subfigure}[b]{0.3\textwidth}
        \centering
        \begin{tikzpicture}[every node/.style={block},
                            block/.style={
                                minimum height=1.5em,
                                outer sep=0pt,
                                draw,
                                rectangle,
                                node distance=0pt
                            },
                            roundnode/.style={round},
                            round/.style={
                                minimum height=1.5em,
                                outer sep=0pt,
                                draw,
                                circle,
                                node distance=0pt
                            }]
            \node (A) {$0$};
            \node (LB) [left=of A] {$1$};
            \node (LC) [left=of LB] {$1$};
            \node (LD) [left=of LC] {\#};
            \node (LE) [left=of LD] {$\ldots$};
            \node (RB) [right=of A] {$1$};
            \node (RC) [right=of RB] {$1$};
            \node (RD) [right=of RC] {\#};
            \node (RE) [right=of RD] {$\ldots$};
            \node[roundnode] (UC) [above = 0.75cm of RC, draw=red, thick] {$q_{0}$};
            
            \draw[-latex] (UC) -- (RC);
            \draw[-latex] (RC) -- (UC);
            
            \draw[-latex, blue] ($(UC.east)!0.5!(RC.east)$) -- ++(7mm, 0);
            \draw[-latex, blue] ($(UC.west)!0.5!(RC.west)$) -- ++(-7mm, 0);
            
            \draw (LE.north west) -- ++(-0.4cm, 0) 
                  (LE.south west) -- ++(-0.4cm, 0) 
                  (RE.north east) -- ++(0.4cm, 0) 
                  (RE.south east) -- ++(0.4cm, 0);
        \end{tikzpicture}
        \caption{Состояние 3.}
    \end{subfigure}
\end{figure}

Это можно представить в виде такой последовательности состояний:
$$
    11q_{0}101 \mto 110q_{0}01 \mto 1101q_{0}1 \mto 11010q_{0}\#.
$$
В последнем состоянии окажется, что головка остановилась на пробельном символе.
Что же будет происходить в таком случае?
Будет ли машина дальше двигаться?
Если машина видит в состоянии $q_{0}$ решетку, то она переходит в конфигурацию $q_{1}$:
$$
    11010q_{0}\# \mto 11010q_{1}\#.
$$
А дальше мы что-нибудь делать будем?
Получается, что наша программа определена не полностью, хотя, формально, надо для любой ситуации описать, что нужно в ней делать.
Практически, указывают только те пары, которые чем-либо нам интересны, а остальные пары, по умолчанию, считаются \enquote{нейтральными}:
$$
    q_{1} \# \mapsto q_{1} \# N.
$$
Тогда получается, что начальная конфигурация
$$
    11q_{0}101 \mreach\footnote{через $c_{1} \mreach c_{2}$ я обозначаю достижимость конфигурации $c_{1}$ из конфигурации $c_{2}$ за конечное число шагов.} 11010q_{1}\#.
$$
Получается, что процесс вычисления на $\MM$ --- это переписывание последовательности конфигураций.
Пусть $\Sigma$ --- некоторое конечное множество, которое мы будем называть {\it входным алфавитом}.
\begin{definition*}
    Машина $\MM$ вычисляет функцию $f: \Sigma^{*} \pto \Sigma^{*}$, если $\Sigma \subseteq \Gamma$ и для любого слова $\vec{x} \in  \Sigma^{*}$
    $$
        \vec{x} \in \dom f \implies q\vec{x} \mto f\left(\vec{x}\right).
    $$
\end{definition*}

Возникает множество вопросов, которые надо уточнить:
\begin{description}
    \item[Q:] где должна находиться головка относительно входа?
    \item[A:] давайте считать, что головка должна стоять на самой последней решетке перед входом.
    \item[Q:] где должна быть головка после вычислений?
    \item[A:] нам очень удобно считать, что после вычислений машина поставит головку перед результатом.
    \item[Q:] а какие же у нас тут должны быть состояния?
    \item[A:] на самом деле, с помощью состояний мы можем передавать некоторую дополнительную информацию.
    Договоримся, что хотя бы одно состояние (выделенное) считается {\it начальным} (обозн. $q_{1}$), и хотя бы одно состояние считается {\it завершающим} (обозн. $q_{0}$).
\end{description}
Тогда определение можно записать следующим образом:
\begin{definition}[вычислимости функции]
    Машина $\MM$ {\it вычисляет функцию} $f: \Sigma^{*} \pto \Sigma^{*}$, если $\Sigma \subseteq \Gamma$ и для любого слова $\vec{x} \in  \Sigma^{*}$
    \begin{description}
        \item[$\vec{x} \in \dom f \implies$] $q_{1}\#\vec{x} \mreach q_{0}\#f\left(\vec{x}\right)$.
        \item[$\vec{x} \notin \dom f \implies$] для любой конфигурации $c$ $q_{1} \# \vec{x} \cancel{\mreach} c$, если $c$ содержит $q_{0}$.
    \end{description}
\end{definition}

\subsection{Формальное описание машины Тьюринга}

\begin{definition}[формальное определение МТ]
    Машина Тьюринга $\MM$ --- это набор конечных множеств
    $$
        \MM = \left(\Gamma, \Sigma, Q, q_{1}, q_{0}, \delta\right),
    $$
    где $\Sigma \subseteq \Gamma$, $\# \in \Gamma \setminus \Sigma$, $q_{1}, q_{0} \in Q$, $\Gamma \cap Q = \emptyset$, $\delta: \left(Q \setminus \{q_{0}\}\right) \times \Gamma \to Q \times \Gamma \times \left\{L, N, R\right\}$ --- функция переходов\footnote{ну или программа, которую выполняет $\MM$.}.
\end{definition}

Конечно, формальное определение машины ничего нам не говорит об определении того, как же ведутся вычисления (то есть, мы пока ничего не знаем про отношение \enquote{$\mto$}).
Давайте и его формально определим.
По сути дела, определив формально МТ, мы заодно определили понятие конфигурации МТ.
Пусть $\conf_{\MM}$ --- это множество всех конфигураций $\MM$.
Определим \enquote{$\mto$} индуктивно как наименьшее по включению отношение такое, что $\forall q, q^{\prime} \in Q$; $\forall a, a^{\prime} \in A$; $\forall b \in B$
\begin{enumerate}
    \item Если $\delta(q, a) = (q^{\prime}, a^{\prime}, R)$, то $AqabB \mto A a^{\prime} q^{\prime} b B$ и $A q a \mapsto A a^{\prime} q^{\prime} \#$.
    \item Аналогично (1) для сдвига $N$.
    \item Аналогично (1) для сдвига $L$.
\end{enumerate}
Обратим внимание, что такое формальное определение отношения на конфигурациях не зависит от слова \enquote{лента}.
Соответственно, $c \mreach c^{\prime} \iff \exists c_{1}, c_{n}: c \mto c_{1} \mto \ldots \mto c_{n} \mto c^{\prime}$.
В частности, может быть, что $c = c^{\prime}$.
Тогда \enquote{$\mreach$} --- рефлексивное транзитивное замыкание отношения \enquote{$\mto$}.

Давайте повторим, что означает вычислимость функции $f$, которая определена на словах конечного алфавита $\Sigma$.
Это значит, что на всех словах, где функция определена, мы из исходной конфигурации попадаем в завершающую, а на всех словах, где она не определена, мы никогда не попадем в завершающую конфигурацию.
Но ведь в теории алгоритмов у нас были функции на натуральных числах, то есть $f: \N \pto \N$.
Как же обработать эту ситуацию, учитывая тот факт, что $f$ определена на словах конечного алфавита, а $\N$ счетно.
Идея: закодируем натуральные числа через символы конечного алфавита.
У нас есть несколько видов кодирования:
\begin{description}
    \item[Унарное кодирование:] Определим функцию унарного кодирования $u: \N \to \{1\}^{*}$.
    Положим код нуля $u(0) = \varepsilon \in \{1\}^{*} \leftarrow$ нулевое слово.
    Тогда $u(n + 1) = 1u(n)$, то есть унарным кодом числа $n$ является $n$ последовательных единиц ($u(3) = 111$), код нуля --- отсутствие единиц.
    \item[Бинарное кодирование:] Определим функцию бинарного кодирования $b: \N \to \left\{0, 1\right\}^{*}$ --- обычная двоичная запись числа $n$: $b(0) = 0$, $b(2n) = b(n)0$, $b(2n + 1) = b(n)1$.
\end{description}

\begin{statement}
    Функция $f: \N \to \N$ такая, что $f(n) = n + 1$ вычислима на МТ в унарных кодах.
\end{statement}

\begin{proof}
    Существует МТ, которая по бинарному коду строит унарный и наоборот, поэтому от вида кодирования числа $n$ результат зависеть не будет.
    Нужно показать, что существует вычислимая функция $g: \left\{1\right\}^{*} \to \left\{1\right\}^{*}$ такая, что $g(u(n)) = u(f(n)) = u(n + 1)$.
    То есть, мы хотим построить такую МТ, что
    $$
        \underbrace{1\ldots1}_{n} \mreach \underbrace{1\ldots1}_{n + 1}.
    $$
    Самое главное --- задать функцию перехода $\delta$.
    Положим $Q = \left\{q_{0}, q_{1}\right\}$, $\Sigma = \left\{1\right\}$, $\Gamma = \left\{\#, 1\right\}$.
    Воспользуемся тем, что в унарной записи нет разницы, приписывать единицу в начало или в конец числа, поэтому определим $\delta$ следующим образом:
    $$
        q_{1}\# \mapsto q_{0}1L.
    $$
    Тогда у нас все получится.
\end{proof}

Важно понимать, что алфавит состояний не может быть бесконечным, как и не может зависеть от входа.
Какую-то дополнительную информацию можно хранить на ленте с помощью дополнительных символов, которые мы можем ввести.

% \begin{statement}
%     Функция $f: \N \to \N$ такая, что $f(n) = 2n$ вычислима на МТ.
% \end{statement}

% \begin{proof}
%     Нужно показать, что существует вычислимая функция $g: \left\{1\right\}^{*} \to \left\{1\right\}^{*}$ такая, что $g(u(n)) = u(f(n)) = u(2n) = u(n)u(n)$.
%     То есть, для любого $\omega \in \left\{1\right\}^{*}$ $g(\omega) = \omega \omega$.
%     Положим $\Sigma = \left\{1\right\}$, $\Gamma = \{\#, 1, 1^{\prime}\}$, $Q = \left\{q_{0}, q_{1}, q_{2}\right\}$

%     % TODO: дописать всратый пример
% \end{proof}

% Важно понимать, что наличие алгоритма не является доказательством его корректности.
% То есть, решая задачу на МТ, вы должны доказать, что ваша программа правильная.
% Но задачи такого рода --- трудные, поэтому не предполагается формальное доказательство, а только какие-то объяснения.
% Например, очень полезно при программировании на МТ изучить вырожденные случаи.
% Например, если в задаче выше $n = 0$, то $q_{1}$ не обнаружит единицу $\implies$ $q_{2}$ увидит решетку, а мы как раз этот случай и не предусмотрели, поэтому нужно дописать следующую инструкцию:
% $$
%     q_{2}\# \mapsto q_{0}\# N,
% $$
% ведь, если $n = 0$, то мы можем сразу остановиться, поскольку удваивать нам нечего.

На практике, в интернете имеется множество ресурсов, которые позволяют вам написать код МТ и его отладить.
Конечно, писать программу вслепую без отладки трудно, ведь даже такая простая программа как удвоение числа потребовала от нас каких-то усилий.

Итак, с помощью унарного кодирования мы поняли, как работать с функциями вида $f: \N \pto \N$.
А если, например, функция зависит от двух аргументов?
Тогда можно ввести дополнительный символ-разделитель \$ $\in \Gamma$, и записать наши аргументы вот так
$$
    \# \code(n) \$ \code(m) \mreach \# \code(f(n, m)).
$$
Заметим, что без \$ можно обойтись в бинарном коде:
$$
    \code\left(\left(n, m\right)\right) = \db(b(n))01\db(b(m)),
$$
где $abc \overset{\db}{\to} aabbcc$, то есть $\db$ удваивает каждый символ в записи числа $n$.
Понятно, что при таком подходе у нас никогда не встретится подстрока вида $01$, поэтому кодирование выше является корректным.
Например,
$$
    \code\left((3, 5)\right) = \underbrace{1111}_{n = 3}01\underbrace{110011}_{m = 5}.
$$

\subsection{Тезис Тьюринга}

Тезис Тьюринга говорит нам, что любая функция, вычислимая в интуитивном смысле\footnote{что бы это не значило.}, вычислима некоторой машиной Тьюринга (при выборе подходящего кодирования аргументов и значений).
По сути дела, этот принцип говорит нам, что машины Тьюринга соответствуют интуитивному понятию вычислений, то есть любое вычисление можно реализовать на машине Тьюринга.
Есть\footnote{см. в книге Верещагина-Шеня \enquote{Лекции по математической логике. Часть 3. Вычислимость.}.} философская аргументация в пользу этого тезиса.
Можно сравнить работу МТ с вычислениями, которые человек мог бы делать на листах бумаги, что, мол, на каждом листе бумаги вы можете написать лишь конечное число различимых текстов, значит каждый лист бумаги можно рассматривать как ячейку, содержащую один из символов конечного алфавита.
Состояний вашего мозга конечно много, то есть мозг --- это такая головка, у нее конечное число состояний.
Вы берете листы и обозреваете в каждый момент только один, переходите от одного к другому перелистывая их, что очень похоже на вычисления, производимые МТ.

\subsection{Свойства алгоритмов в терминах машин Тьюринга}

Попробуем проследить связь всей этой теории про МТ со свойствами алгоритмов.
\begin{description}
    \item[\enquote{Алгоритмов лишь счетно много}:] Можно ли сказать, что машин Тьюринга счетно много?
    Как мы помним, $\MM = \left(\Gamma, \Sigma, Q, q_{0}, q_{1}, \delta\right)$, где $\Gamma, \Sigma, Q$ конечны; $q_{0}, q_{1} \in Q$; $\delta: \left(Q \setminus q_{0}\right)  \times \Gamma \to Q \times \Gamma \times \left\{L, N, R\right\}$.
    То есть вопрос сводится к тому, сколько существует конечных множеств?
    Ответ: очень много.
    То есть настолько много, что нельзя говорить о том, что существует множество всех конечных множеств.
    С точки зрения формальной теории множеств, все конечные множества образуют т. н. {\it собственный класс} (как и все возможные множества), то есть множество они не образуют.
    Следовательно, машин Тьюринга тоже очень много.
    Как же получить счетное число таких машин?
    На самом деле, многие МТ отличаются друг от  друга только названиями состояний и символов, то есть они изоморфны.
    Две МТ $\left(\Gamma, \Sigma, Q, q_{0}, q_{1}, \delta\right)$ и $\left(\Gamma^{\prime}, \Sigma^{\prime}, Q^{\prime}, q_{0}^{\prime}, q_{1}^{\prime}, \delta^{\prime}\right)$ называются {\it изоморфными}, если существует отображение\footnote{по сути, такие МТ отличаются переименованием элементов множеств.} $\phi$ такое, что $\Gamma \overset{\phi}{\sim} \Gamma^{\prime}$, $Q \overset{\phi}{\sim} Q^{\prime}$, $\phi\left(q_{0}\right) = q_{0}^{\prime}$, $\phi\left(q_{1}\right) = q_{1}^{\prime}$, $\delta(q, a) = (t, b,  S) \iff \delta(\phi(q), \phi(a)) = (\phi(t), \phi(b), S)$.
    С точностью до изоморфизма существует лишь счетно много машин Тьюринга, как и алгоритмов.
    \item[\enquote{Композиция вычислимых функций вычислима}:] Пусть у нас машина $\MM$ вычисляет функцию $f: \Sigma^{*} \pto \Sigma^{*}$, а $\NN$ вычисляет $g: \Sigma^{*} \pto \Sigma^{*}$.
    У нас $\MM = \left(\Gamma, \Sigma, Q, q_{0}, q_{1}, \delta\right), \NN = \left(\Gamma^{\prime}, \Sigma, Q^{\prime}, q_{0}^{\prime}, q_{1}^{\prime}, \delta^{\prime}\right)$.
    С точностью до изоморфизма, считаем, что $Q \cap Q^{\prime} = \emptyset$.
    Построим машину $\LL = \left(\Gamma \cup \Gamma^{\prime}, \Sigma, Q \cup Q^{\prime}, q_{0}, q_{1}^{\prime}, \hat{\delta}\right)$, где $\hat{\delta} = \delta \cup \delta^{\prime} \cup \left\{q_{1} \# \to q_{0}^{\prime} \# N\right\}$.
    Поскольку
    $$
        q_{0}\#\vec{x} \mreach q_{1}\# f(\vec{x}) \underset{\LL}{\twoheadrightarrow}  q_{0}^{\prime} \# f(\vec{x}) \underset{\NN}{\twoheadrightarrow} q_{1}^{\prime} \# g\left(f\left(\vec{x}\right)\right),
    $$
    то машина $\LL$ позволяет вычислить композицию $f \circ g$, ведь она содержит в себе машины $\MM$ и $\NN$.
    Формально, состояние $q_{1}^{\prime}$ не является завершающим для машины $\NN$, поэтому мы должны доопределить $\hat{\delta}$, чтобы все было корректно.
    \item[\enquote{Существование у. в. ф.}:] Пусть алфавит $\Sigma \supseteq \left\{0, 1\right\}$ фиксирован.
    Почему можно алфавит зафиксировать?
    Ну потому что можно доказать, что любой алфавит можно, путем кодирования его символов, заменить алфавитом из нулей из единиц.
    То есть если функция, которая работает на бинарных кодах, вычислима в одном алфавите, содержащем нули и единицы, то она вычислима некоторой машиной алфавите, содержащем только нули и единицы.
    Поэтому выбор алфавита не очень существеннен, а ограничение на наличие в нем $0$ и $1$ довольно слабое.
    Любую машину с таким $\Sigma$ можно кодировать в $\{0, 1\}$, то есть положим $\Gamma = \{\code(a)~|~a \in \Gamma\}$, $Q = \{q_{0}, \ldots, q_{n}\} = \left\{\$b(i)\$~|~i \in \{0, \ldots, n\}\right\}$, $\delta$ --- cписок конечных инструкций $qa \mapsto q^{\prime}a^{\prime}S^{\prime}$, каждую из которых можно закодировать, то есть каждая такая инструкция --- слово конечного алфавита $\implies$ его можно закодировать двоичным кодом.
    Поэтому каждая $\MM \mapsto \code(\MM) \in \left\{0, 1\right\}^{*} \implies$ любая $\MM$ --- это конечный объект.
    \begin{theorem}
        Существует МТ $\UU$ с входным алфавитом $\Sigma$ такая, что для любой МТ $\MM$ с входным алфавитом $\Sigma$
        $$
            \forall \vec{x} \in \Sigma^{*} \quad \UU\left(\left\langle \code\left(\MM\right), \vec{x}\right\rangle\right) \simeq \MM(\vec{x}).
        $$
    \end{theorem}
    Здесь \enquote{$\simeq$} означает, что либо обе машины приходят к одной завершающей конфигурации, либо обе не останаливаются (никогда не приходят к завершающей конфигурации).
    \item[\enquote{Возможность выполнять по шагам}:] Можно доказать, что существует машина $\TT$ такая, что для любой машины $\MM$ и для любых $\vec{x}, \vec{y}$, а также для любого $k \in \N$
    $$
        \TT\left(\left\langle \code(\MM), \vec{x}, \vec{y}, u(k)\right\rangle\right) = \begin{cases}
            1, & \text{если $M$ на входе $\vec{x}$ останавливается за $k$ шагов и выдает $\vec{y}$,} \\
            0, & \text{иначе}.
        \end{cases}
    $$
\end{description}


