\ProvidesFile{lect-11.tex}[Лекция 11]

\section{Лекция 11}

Давайте перейдем теперь к идейному содержанию построенной нами теории.
Зачем нам все это нужно?
Логика, по сути дела, изучает, что делает математика.
В частности, нам хочется понять, что вообще с этой логикой можно делать.
Давайте, например, попробуем поговорить о группах.
Мы можем смотреть на группу как на некоторую структуру
$$
    \GG = (A; =; +^{(2)}, -^{(1)}; 0),
$$
но ведь не всякая интерпретация такой структуры является группой.
Мы хотим, что бы наша бинарная операция была ассоциативна, то есть
$$
    \GG \models \forall x \forall y \forall z ~ (x + y) + z = x + (y + z).
$$
Таким образом мы получили полугруппу.
Какие еще свойства должны выполняться в группе?
Нам нужно, чтобы для каждого элемента существовал обратный к нему, то есть
$$
    \GG \models \forall x~ (x + (-x) = 0).
$$
Будем конструировать абелеву группу, поэтому потребуем еще и коммутативность:
$$
    \GG \models \forall x~ (x + (-x) = 0 \land (-x) + x = 0).
$$
Осталось сказать, что 0 является нейтральным элементом, то есть
$$
    \GG \models \forall x~ (x + 0 = x \land 0 + x = x).
$$

Всего этого достаточно, чтобы $\GG$ была группой, однако, с другой стороны, у нас получилось три формулы.
И вот эти формулы являются аксиомами группы.
Назовем три предложения, которые обозначают наши аксиомы, соответственно $\varphi_{\text{ass}}$, $\varphi_{\text{inv}}$ и $\varphi_{\text{neut}}$ и рассмотрим множество
$$
        T = \{\varphi_{\text{ass}}, \varphi_{\text{inv}}, \varphi_{\text{neut}}\}.
$$
Мы, по сути дела, сказали, что
$$
    \forall \text{ нормальной } \GG \quad (\GG \text{ --- группа } \iff \forall \varphi \in T~ \GG \models \varphi.)
$$
Нормальность значит, что символ равенства обозначает равенство в нашем привычном понимании.
Поскольку любая группа удовлетворяет предложениям из $T$, можно сказать, что $T$ --- теория групп.

Давайте теперь подойдем с другой стороны.
Рассмотрим структуру $\MM = (A; =; <)$.
Когда такая штука является частично упорядоченным множеством?
Можем ли мы написать формулы, которые выполняются тогда и только тогда, когда она является порядком\footnote{тоже самое, что и частично упорядоченное множество.}?
\begin{description}
    \item[Антирефлексивность] $\MM \models \varphi_{\text{antireflex}} \eqcirc \forall x~ \neg (x < x)$;
    \item[Транзитивность] $\MM \models \varphi_{\text{transit}} \eqcirc \forall x \forall y \forall z~ (x < y \land y < z \implies x < z)$.
\end{description}
Утверждается, что любая нормальная $\MM$ является частично упорядоченным множеством $\iff \forall \varphi \in T_{\text{ord}} = \{\varphi_{\text{antireflex}}, \varphi_{\text{transit}}\}~ \MM \models \varphi$.
Что мы видим?
Разные классы математических структур можно задавать просто записав множество предложений, которые выполняются тогда и только тогда, когда наша структура является тем, что мы описываем (группой, порядком).

\subsection{Логическое (семантическое) следствие}

\begin{definition}
    {\it Теория} (в сигнатуре $\sigma$) --- это любое множество предложени в сигнатуре $\sigma$.
\end{definition}

\begin{definition}
    $\MM \models T \iff \forall \varphi \in T~ \MM \models \varphi$.
\end{definition}

\begin{definition}
    Теория $T$ {\it выполнима} (или {\it совместна}) $\iff \exists \MM~ \MM \models \TT$.
\end{definition}

\begin{statement}[антисимметричность порядка]
    Для любой модели $\MM$
    $$
        \MM \models T_{\text{ord}} \implies \MM \models \forall x \forall y~ (x < y \land y < x \implies x = y).
    $$
\end{statement}

\begin{proof}
    Мы знаем, что $\MM \models \forall x~ \neg x < x$ и $\MM \models \forall x \forall y \forall z~ (x < y \land y < z \implies x < z)$.
    Хотим, чтобы $\MM \models \forall x \forall y~ (x < y \land y < x \implies x = y)$.
    Зафиксируем $a, b \in M$, тогда $a <^{\MM} b$ и $b <^{\MM} a$.
    Тогда, по транзитивности, $a <^{\MM} a$, а по антирефлексивности $\neg a <^{\MM} a$.
    Получили противоречие, то есть $\forall a, b \in \MM$
    $$
        \MM \models \neg(a <^{\MM} b \land b <^{\MM} a),
    $$
    тогда $\MM \models (x < y \land y < x \implies x = y)(a, b)$, потому что импликация истинна для произвольных $a$, $b$.
\end{proof}

В некотором смысле, $\varphi_{\text{antisym}}$ следует из $\varphi_{\text{antireflex}}$ и $\varphi_{\text{transit}}$, то есть, следует из $T_{\text{ord}}$.

\begin{definition}
    Предложение $\varphi$ {\it логически следует} из теории $T$, если $\forall \MM~ (M \models T \implies \MM \models \varphi)$.
\end{definition}

\paragraph{Обозначение}
$T \models \varphi$ (из теории $T$ следует $\varphi$)

Что значит логическое следование?
Оно значит, что в любой структуре, где верна теория $T$, верно и предложение $\varphi$.

Вернемся к теории групп.
Выразим, что нейтральный элемент единственный.
Для начала, нам потребуется выразить условие нейтральности элемента:
$$
    \psi \eqcirc \forall y\forall x~ (x + y = x \land y + x = x) \implies y = 0.
$$
Такая формула является фактом из теории групп, но как это доказать?

\begin{statement*}
    $T_{\text{гр}} \models \psi$ ($\iff \forall \text{ нормальной } \MM~(\MM \models T_{\text{гр}} \implies \MM \models \psi)$).
\end{statement*}

\begin{proof}
    Нам дано, что у нас есть нормальная структура $\MM$ и $\MM \models T_{\text{гр}}$, то есть
    \begin{align}
        \MM &\models \varphi_{\text{ass}}, \\
        \MM &\models \varphi_{\text{inv}}, \\
        \MM &\models \varphi_{\text{neut}}.
    \end{align}
\end{proof}

\begin{proof}
    Зафиксируем $b \in M$ и допустим, что $\MM \models (\forall x~ (x + y = x \land y + x = x))(b)$.
    Мы хотим, чтобы $b = 0^{\MM}$.
    Рассуждаем как в обычной математике.
    Нам нужно подставить в качестве $x$ $0^{\MM}$, тогда
    $$
        0^{\MM} + b = 0^{\MM},
    $$
    но из $\varphi_{\text{neut}}$ мы знаем, что $0^{\MM} + z = z$, тогда
    $$
        0^{\MM} + b = 0^{\MM} = b.
    $$
\end{proof}

Давайте напоследок введем еще одну теорию, которая нам понадобится --- теория $\DLO$\footnote{Dense Linear Order.}:
$$
    \DLO = T_{\text{ord}} \cup \{\forall x \forall y (x < y \lor y < x \lor x = y), \forall x \forall y (x < y) \implies \exists z (x < z \land z < y), \forall x \exists y~ x < y, \forall x \exists y~ y < x\}.
$$
Наш $\DLO$ отличается от общепринятого $\DLO$ отсутствием наибольшего и наименьшего элементов.
Для любой нормальной $\MM$ ($\MM \models \DLO \iff \MM$ --- плотный линейный порядок без максимума и минимума).

\paragraph{Примеры}
$(\R, <) \models \DLO$; $(\Q, <) \models \DLO$; $(\R \cap (0, 1), <) \models \DLO$.

\paragraph{Контрпримеры}
\begin{itemize}
    \item $(\N, <) \centernot\models \DLO$ --- присутствует минимум;
    \item $(\Z, <) \centernot\models \DLO$ --- нет максимума и минимума, но порядок не плотный;
    \item $(R \cap [0, 1], <) \centernot\models \DLO$ --- присутствует и минимум, и максимум.
\end{itemize}

\subsection{Теорема компактности (без доказательства)}

До сих пор мы рассматривали примеры, в которых логика нам не давала ничего кроме лишних мучений.
Рассмотрим следующее утверждение:

\begin{statement*}
    Предположим, что мы знаем, что какая-то формула логически следует из теории (т.е. $T \models \varphi$), всегда ли существует конечное подмножество $T^{\prime} \subseteq T$ такое, то $T^{\prime} \models \varphi$?
\end{statement*}

Учитывая тот факт, что теория может быть бесконечной, такое утверждение сложно доказать.
Тем не менее, оно верно и называется {\it теоремой о компактности}.
Выглядит оно может быть и не очень презентабельно, но из него следует куча всяких вещей.
Например, мы можем с вами доказать, что не существует теории, которая имеет модели сколь угодно большой конечной мощности, но не имеет бесконечной модели.
Пока мы просто поверим в эту теорему, а потом ее докажем.

\begin{statement}
    Можем ли мы придумать такую теорию $T_{\text{infin}}$, что для любой нормальной $\MM$ ($\MM \models T_{\text{fin}} \iff M \text{ бесконечно}$)?
\end{statement}

\begin{proof}
    Поскольку теорию выбираем мы, то и сигнатуру выбираем тоже мы.
    Пусть в $\sigma$ есть равенство \enquote{$=$} и есть различные константы $\{c_{0}, c_{1}, c_{2}, \ldots \} \sim \N$.
    Определим формулу $\Diff_{n}(\vec{c})$, которая говорит, что $n$ элементов попарно не равны, индуктивно:
    \begin{description}
         \item[$k = 2 \colon$] $\Diff_{2}(c_{0}, c_{1}) \eqcirc \neg x_{1} = x_{2}$;
         \item[$k = n + 1 \colon$] $\Diff_{n + 1}(c_{0}, \ldots, c_{n}) \eqcirc \Diff_{n} \land \bigwedge\limits_{i = 0}^{n}\neg c_{n + 1} = c_{i}$.
    \end{description}
    Положим $T_{\text{infin}} = \{\exists x_{0} \exists x_{1} \Diff_{2}(x_{0}, x_{1}), \exists x_{0} \exists x_{1} \exists x_{2} \Diff_{3}(x_{0}, x_{1}, x_{2}), \ldots, \exists x_{0} \ldots \exists x_{n} \Diff_{n + 1}(x_{0}, \ldots, x_{n}), \ldots\}$.
    Если в какой-то структуре данная теория верна, то структура автоматически является бесконечной, то есть
    $$
        \MM \models T_{\text{infin}} \implies \text{ Для любого $n$ в $M$ есть $n$ попарно различных элементов } \implies M \text{ бесконечно}.
    $$
    Обратно, допустим, что $M$ бесконечно, тогда $\forall n$ в $M$ можно найти $n$ различных элементов для любого $n$, тогда $\forall n~ \MM \models \exists x_{1} \ldots \exists x_{n}~ \Diff_{n}(x_{1}, \ldots, x_{n}) \implies \MM \models T_{\text{infin}}$.
\end{proof}

И так, мы получили теорию, которая выполняется тогда и только тогда, когда структура бесконечна.

\subsection{Невозможность аксиоматизации класса конечных нормальных структур}

Докажем следующее следствие из теоремы компактности:

\begin{corollary}
    Если для любого конечного подмножества $T^{\prime} \subseteq T$ $T^{\prime}$ выполнимо, то выполнима и вся $T$.
\end{corollary}

\begin{proof}
    Давайте рассмотрим тождественную ложь $\bot$, равную любой не выполнимой формуле (например, $\exists x (Px \land \neg Px)$).

    \paragraph{Замечание}
    $T \models \bot \iff T$ не выполнима

    \begin{proof}
        $T \models \bot \iff \forall \MM (\MM \models T \implies M \models \bot)$, но $\MM \models \bot$ не может быть истинна, отсюда $\forall \MM~ \MM \centernot \models \iff T$ не выполнима.
    \end{proof}

    Будем доказывать исходное утверждение от противного.
    Пусть $T$ не выполнима, тогда $T \models \bot$.
    Тогда, по теореме о компактности, существует конечное $T^{\prime} \subseteq T$ такое, что $T^{\prime} \models \bot \implies T^{\prime}$ не выполнимо, что противоречит утверждению.
\end{proof}

\begin{statement}
    $\nexists T_{\text{fin}}$ такой, что $\forall \MM~ (\MM \models T_{\text{fin}} \iff \MM \text{ конечна})$.
\end{statement}

\begin{proof}
    Пусть такая теория $T_{\text{fin}}$ есть в сигнатуре $\sigma$.
    Добавим к $\sigma$ новые константные символы $\{c_{0}, c_{1}, \ldots\} \sim \N$.
    Рассмотрим бесконечную теорию
    $$
        \hat{T} = T_{\text{fin}} \cup \{\neg c_{i} = c_{j} \mid i, j \in \N\}.
    $$
    Утверждается, что для любого конечного $T^{\prime} \subseteq \hat{T}$ $T^{\prime}$ выполнима.
    Почему?
    Заметим, что $T^{\prime} \subseteq T_{\text{fin}} \cup \{ \neg c_{i} = c_{j} \mid i, j \leqslant K \}$
    Рассмотрим конечное множество $M$ мощности не меньше $K$.
    Тогда мы видим, что в структуре $(M; =; \ldots) \models T_{\text{fin}}$ так как она выполняется в любой конечной структуре.
    С другой стороны, мы можем сопоставить $c_{0}, \ldots, c_{K}$ какие-то элементы из $M$, поскольку мощность $M$ не меньше $K$.
    Поэтому $(M; =; \ldots) \models \{\neg c_{i} = c_{j} \mid i, j \leqslant K\}$.
    Таким образом $T^{\prime}$ выполнимо для любого конечного подмножества.
    Тогда, по теореме компактности, $\hat{T}$ выполнимо $\implies \exists \hat{\MM}~ \hat{\MM} \models \hat{T} \implies \hat{\MM} \models T_{\text{fin}} \cup \{c_{i} = c_{j} \mid i, j \in \N\} \implies \hat{\MM}$ бесконечна, но $\hat{\MM} \models T_{\text{fin}}$, а мы предполагали, что любая такая модель конечна.
\end{proof}

Что говорит нам этот результат?
Он нам говорит, что свойство \enquote{быть конечным} не может быть выражено в логике первого порядка.
То есть мы с вами находим некоторые границы применимости логики первого порядка.

\subsection{Сколемизация}

Давайте рассмотрим формулу предварённую:
$$
    \forall x \forall y \exists z~ Pxyz.
$$
Что значит, что $\MM \models \forall x \forall y \exists z~Pxyz$?
Это значит, что для любых $a$, $b$ $\in M$ найдется $c = c(a, b)$ такое, что $P^{\MM}(a, b, c)$.
Рассмотрим формулу
$$
    \forall x \forall y \exists z~ x + y = z.
$$
Здесь $z$, по сути дела, является функцией от $x$ и $y$.
Мы можем эту функцию некоторым образом сделать явной.
Для этого нам нужно преоброазовать формулу:
$$
    \forall x \forall y \exists z~ Pxyz \leadsto \forall x \forall y~ Pxy(fxy),
$$
где $f$ --- ранее не встречавшийся функциональный символ.
Мы помним, что формула
$$
    \varphi(t/x) \implies \exists x \varphi
$$
общезначима, если $t$-$x$-$\varphi$.
Значит, в предположении, что подстановка корректна (в данном случае она корректна, потому что в формуле нет кванторов по $x$ и $y$), формула
$$
    Pxy(fxy) \implies \exists z Pxyz
$$
общезначима.
Отсюда, в свою очередь, следует, что формула
$$
    \forall x \forall y Pxy(fxy) \implies \forall x \forall y \exists z Pxyz
$$
общезначима.
То есть, по сути дела из формулы
$$
    \forall x \forall y Pxy(fxy)
$$
следует формула
$$
    \forall x \forall y \exists z.
$$
А наоборот?
Наоборот, конечно же, нет, ведь в этой формуле $f$ вообще не упоминалась.
Поэтому, если такая формула истинна в какой-то структуре, то совершенно не обязательно, что в той же структуре будет истинна формула $\forall x \forall y Pxy(fxy)$, потому что не понятно, что такое $f$.
Тем не менее, мы можем $f$ как-то проинтерпретировать и подобрать значения таким образом, что это будет функция, которая по любым $x$, $y$ выбирает $z$.
Тогда мы получим равносильность.

Рассмотрим следующую формулу в предположении, что в ней нет фиктивных кванторов:
$$
    \exists w \forall x \forall y \exists z \forall u \exists v~\varphi \leadsto \forall x \forall y \exists z \forall u \exists v~\varphi(c/w) \leadsto \forall x \forall y  \forall u \exists v~\varphi(c/w, fxy/z) \leadsto \forall x \forall y \forall u~\varphi(c/w, fxy/z, gxyu/v),
$$
что можно сказать про такую формулу?
Если считать, что $\varphi$ безкванторная, то у нас получилась формула, в которой нет кванторов существования.
Утверждается, что исходная формула выполнима тогда и только тогда, когда выполнима формула, которая получается в результате этих шагов.
Ну и на каждом шаге сохраняется выполнимость.
Вообще говоря, формулы не эквивалентны, но они равновыполнимы.

Содержательно, мы уже проговаривали, почему это так.
Ну действительно, если выполнима формула
$$
    \forall x \forall y \exists z~ Pxyz,
$$
то где-то, в какой-то структуре, для любых $x$ и $y$ находится значение $z$.
Добавим к сигнатуре символ $f$ и проинтерпретируем его функцией (которая должна существовать в силу аксиомы выбора) которая каждому $x$ и $y$ ставит в соответствие подходящий $z$.
Такая процедура называется {\it сколемизацией}.

\paragraph{Формально}
У нас есть отношение $\to_{S}$, устроенное следующим образом:
$$
    \forall \vec{x}~\exists y~ \varphi \to_{S} \forall \vec{x}~\varphi(f\vec{x}/y).
$$
При этом, мы предполагаем, что формула $\forall \vec{x}~\exists y \varphi$ явялется предварённой.

\begin{definition}
    Предварённая формула $\varphi^{\prime}$ называется {\it сколемовской нормальной формой} формулы $\varphi$, если
    \begin{enumerate}
        \item В $\varphi^{\prime}$ нет квантора $\exists$;
        \item $\varphi \to_{S} \varphi_{1} \to_{S} \ldots \to_{S} \varphi_{n} \to_{S} \varphi^{\prime}$ (обозн. $\varphi \twoheadrightarrow_{S} \varphi^{\prime}$).
    \end{enumerate}
\end{definition}

\begin{statement}
    Если $\varphi \twoheadrightarrow_{S} \varphi^{\prime}$, то $\varphi$ выполнима $\iff$ $\varphi^{\prime}$ выполнима.
\end{statement}